
\chapter{Automatic Generation of Code}\label{sec:codegen}

It is possible to generate Java code\index{code generation} for a
large subset of VDM-SL and VDM++ models. In addition to Java, a C++
code generator is currently being developed, but this work is in its
early stages of development, and it is not included with releases of
Overture yet. For comparison, code generation of VDM-SL and VDM++
specifications to both Java and C++ is a feature that is available in
VDMTools~\cite{Java2VDMMan,CGMan,CGManPP}. The sections below focus
solely on the Java code generator available in Overture.

\section{Use of the Java Code Generator}
\label{sec:javacg_use}

\begin{figure}[htbp]
\begin{center}
\includegraphics[width=10cm]{screenDumps/javacg_menu}
\caption{Launching the Java code generator.\label{fig:javacg_menu}}
\end{center}
\end{figure}

\begin{figure}[htbp]
\begin{center}
\includegraphics[width=\linewidth]{screenDumps/javacg_output}
\caption{The status of code generating the \texttt{AlarmPP} example.\label{fig:javacg_output}}
\end{center}
\end{figure}

The Java code generator can be launched via the context menu as shown
in Figure~\ref{fig:javacg_menu}. Alternatively, this can be done by
highlighting the project in the VDM explorer and typing one of the
shortcuts associated to this plugin.\\

\noindent The Java code generator operates in two different modes:

\begin{itemize}

\item \emph{Regular mode:} In this mode the Java code generator
  produces an Eclipse project with all the generated code. Java code
  generation in this mode can also be initiated using the
  \texttt{Ctrl+Alt+C} shortcut.

\item \emph{Launch Configuration mode:}. Is currently limited to
  VDM++. This mode is like regular code generation except that the
  Java code generator also prompts the user for a launch configuration
  as input for the code generation process. Based on this launch
  configuration the Java code generator constructs an entry point (a
  \texttt{main} method really) that serves as an entry point for the
  generated code. Launch configuration based code generation can be
  initiated using the \texttt{Ctrl+Alt+B} shortcut.

\end{itemize}

\noindent Upon completion of the code generation process the status is
output to the console as shown in Figure~\ref{fig:javacg_output}. In
particular this figure shows the status of code generating the
\texttt{AlarmPP} model available in the Overture standard examples. As
indicated by the console output, the generated code is available as an
Eclipse project in the \texttt{<workspace>/<project>/generated/java}
folder.

\section{Configuration of the Java Code Generator}

The Java code generator can be configured via a preference page as
shown in Figure~\ref{fig:javacg_config}. The preference page can be
accessed in the way you would normally access an Eclipse preference
page or via the context menu shown above in
Figure~\ref{fig:javacg_menu}. The Java code generator provides a few
options that allows the user to configure the code generation process
(see Figure~\ref{fig:javacg_config}). The sub-sections below treat
each of these configuration parameters individually, in the order they
appear in the preference page.

\begin{figure}[htbp]
\begin{center}
\includegraphics[width=13cm]{screenDumps/javacg_config}
\caption{Configuration of the Java code generator.\label{fig:javacg_config}}
\end{center}
\end{figure}

\subsection{Disable cloning}

In order to respect the value semantics of VDM the Java code generator
sometime needs to perform deep copying of certain objects that is uses
to represent composite value types (records, tuples, tokens, sets,
sequences and maps). For example, in VDM a record is a value type,
which means that occurrences of the record must be copied when it
appears in the right-hand side of an assignment, it is passed as an
argument or returned as a result. However, Java does not support
composite value types like structs and records, and as a consequence
record types must be represented using classes, which use reference
semantics. This means that an object reference, which is used to
represent a composite value type in the generated Java code must be
deep copied when it appears in the right-hand side of an assignment,
it is passed as an argument or returned as a result. For arbitrarily
complex value types (such as records composed of record or sets of
composed of sets) deep copying may introduce a significant overhead in
the generated code. If the specification subject to code generation
does not truly rely on value semantics the user may wish to disable
deep copying of value types in the generated code in order to remove
this overhead. The user should, however, be aware that disabling of
cloning may lead to code being generated that does not preserve the
semantics of the input specification and in general disabling of
cloning is discouraged. By default cloning is enabled.

\subsection{Generate character sequences as strings}

In VDM a string is a sequence of characters and there is no notion of
a string type. Java in particular works differently since it uses a
separate type to represent a string. The default behaviour of the Java
code generator is to code generate sequences of characters as strings
and subsequently do the necessary conversion between between strings
and sequences in the generated code. Another possibility is to treat a
string literal for what it truly is, namely a sequence of characters,
and thereby avoid any conversion between strings and sequences. In
order to do that, i.e.\ \textit{not} generating character sequences as
strings, the corresponding option must be unchecked.

\subsection{Generate concurrency mechanisms}

If the user does not rely on the concurrency mechanisms of VDM++ and
does not want to include support for them in the generated code the
corresponding option in the preference page must be unchecked. By
default the behaviour of the Java code generator is to not include
support for the concurrency mechanisms of VDM++ in the generated code.

\subsection{Generate Java Modeling Language (JML) annotations}

When a VDM model is code generated to Java all the contract-based
elements of the model, i.e.\ the pre conditions, post conditions and
invariants, are ignored by default. This feature does, however,
provide limited support for translation of the contract-based elements
of a VDM-SL model. When the corresponding option is checked the
contract-based elements are translated to JML~\cite{Burdy&05}
annotations which are added on top of the generated Java code as
source code comments. This allows the system properties, expressed in
terms of pre conditions, post conditions and invariants, to be checked
against the generated code.\\

\noindent Please note that this feature is still experimental and only
has limited support for checking of named type invariants.

\subsection{Choose output package}

The Java code generator allows the output package of the generated
code to be specified. If the user does not specify a package, the code
generator outputs the generated Java code to a package with the same
name as the VDM project. If the name of the project is not a valid
java package, then the generated code is output to the default Java
package.

\subsection{Skip classes during the code generation process}

It may not always make sense to code generate every class in a
project. Classes that can often be skipped include classes that act as
execution entry points or classes that are used to load input for the
specification. Classes that the user wants to skip can be specified in
the text box in the Java code generator preference page by separating
the class names by a semicolon. As an example, \texttt{World;Env}
makes the code generator skip code generation of the \texttt{World}
and \texttt{Env} classes, while generating code for any other
class. For convenience the output of the Java code generator will also
inform the user about what classes are being skipped.

\section{Limitations of the Java Code Generator}

In case the Java code generator encounters a construct that it cannot
code generate it will report it as unsupported to the user and the
user can then try to rewrite that part of the specification using
other (supported) constructs. Reporting of unsupported constructs is
done via the console output and using editor markers. In order to
demonstrate this, Figure~\ref{fig:javacg_unsupported} shows the
console output of the Java code generator when it encounters a type
bind, which is an example of an unsupported language construct. Note
the small marker appearing in the editor in order to point out where
use of the construct appears. By hovering the corresponding marker the
user will see the reason for why the construct is unsupported. For the
type bind example in Figure~\ref{fig:javacg_unsupported} the reason
is:\\

\noindent \texttt{Following constructs are not supported:\\ \{b |
b:(bool * bool)\} (ASetCompSetExp) at 6:11. Reason:\\Generation of a
set comprehension is only supported for\\multiple set binds. Got:
b:(bool * bool)}\\

\begin{figure}[htbp]
\begin{center}
\includegraphics[width=\linewidth]{screenDumps/javacg_unsupported}
\caption{Reporting of unsupported constructs in the
console.\label{fig:javacg_unsupported}}
\end{center}
\end{figure}

The user will get similar messages and markers for other unsupported
VDM constructs. To summarise, the Java code generator currently does
not support code generation of multiple inheritance and neither does
it support traces, type bind, invariant checks and pre and post
conditions. Furthermore, let expressions appearing on the right-hand
side of an assignment will also be reported as unsupported. Code
generation of patterns is also fairly limited and includes the
identifier, ignore, record and tuple patterns as well as the
character, boolean, integer, nil, quote, real and string match values.

\section{The Code Generation Runtime Library}

The generated code relies on a runtime library used to represent some
of the types available in VDM (tokens, tuples etc.) as well as
collections and support for some of the complex operators such as
sequence modifications. For simplicity every Eclipse project generated
by the Java code generator contains the runtime library. More
specifically, there is a copy of the runtime library containing only
the binaries (\texttt{lib/codegen-runtime.jar}) as well as a version
of the runtime library that has the source code attached
(\texttt{lib/codegen-runtime-sources.jar}). The runtime library is
imported by every code generated class using the Java import statement
\texttt{\textbf{import} org.overture.codegen.runtime.*;} and in order
to compile the generated Java code the runtime library must be visible
to the Java compiler.

Similar to VDMTools the runtime library also provides implementation
for subset of the functionality available in the standard libraries:
The runtime library provides a full implementation of the
\texttt{MATH} library, support for conversion of values into character
sequences as provided by the \texttt{VDMUtil}, and finally
functionality to write to the console as available in the \texttt{IO}
library.
