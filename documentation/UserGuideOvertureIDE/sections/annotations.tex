
\chapter{Model Annotations}\label{sec:annotations}

\section{Overview}

Annotations were introduced in Overture version 2.7.0 as a means to
allow a specifier to affect the tool’s behaviour without affecting the
meaning of the specification. The idea is similar to the notion of
annotations in Java, which can be used to affect the Java compiler,
but do not alter the runtime behaviour of a program. Overture provides
some standard annotations, but the intent is that specifiers can
create new annotations and add them to the Overture system
easily. Note that this feature is currently \emph{limited to the
  Overture command-line interface}.

\section{Syntax}\label{sec:annotations-syntax}

Annotations are added to a specification either as block comments or
one-line comments. This is so that other VDM tools will not be
affected by the addition of annotations, and emphasises the idea that
annotations do not alter the meaning of a specification. An annotation
must be present at the start of a comment, and has the following
default syntax:

\begin{description}
\item `@', identifier, [ `(', expression list, `)' ]
\end{description}

\noindent So for example, an operation in a VDM++ class could be
annotated as follows:

\begin{lstlisting}[language=VDM++, frame=tlbr,escapechar=!]
class A
operations
  -- @Override
  public add: nat * nat ==> nat
  add(a, b) == !\ldots!
\end{lstlisting}

\noindent Or the value of variables can be traced during execution as
follows:

\begin{lstlisting}[language=VDM++, frame=tlbr, label={lst:annotations-trace-ex}]
functions
  add: nat * nat +> nat
  add(a, b) ==
    /* @Trace(a, b) */ a + b;
\end{lstlisting}

\section{Location}

Annotations are always located next to another syntactic category,
even if they do not affect the behaviour or meaning of that
construct. In the examples above, the
\lstinline[language=VDM++]|@Override| annotation applies to the
definition of the add operation, and the
\lstinline[language=VDM++]|@Trace| annotation applies to the
expression \lstinline[language=VDM++]|a + b|.

Specific annotations may limit where they can be applied (for example,
\lstinline[language=VDM++]|@Override| only makes sense for operations
and functions in VDM++ specifications), but in general annotations can
be applied to the following:

\begin{itemize}
  \item classes or modules.
  \item definitions within a class or module.
  \item expressions within a definition.
  \item statements within an operation body.
\end{itemize}

In each case, the annotation must appear in a comment, by itself,
before the construct concerned. Multiple annotations can be applied to
the same construct, and may be interleaved with other textual
comments, but each annotation must appear in its own comment.

Note that an annotation in an expression effectively acts as an
operator which has a high binding precedence. So in the
\lstinline[language=VDM++]|@Trace| example above, the annotation binds
to the \lstinline[language=VDM++]|a| variable sub- expression, not
\lstinline[language=VDM++]|a + b|. This makes no difference with
\lstinline[language=VDM++]|@Trace|, but
\lstinline[language=VDM++]|@NoPOG| and
\lstinline[language=VDM++]|@OnFail| apply to a specific sub-expression
and they should be used with bracketed expressions to make that
clear. For example:

\begin{lstlisting}[language=VDM++, frame=tlbr]
divide: nat * nat +> real
divide(p, q) ==
/* @NoPOG */ (p/q);
\end{lstlisting}

Without the brackets around \lstinline[language=VDM++]|p/q| you
would get the warning shown below, and the
\lstinline[language=VDM++]|@NoPOG| would only apply to the
\lstinline[language=VDM++]|p|, which would therefore
still generate a PO for the division operator.\\

\noindent \texttt{Warning 5030: Annotation is not followed by
  bracketed\\ sub-expression}\\

\section{Tool Effects}

Annotations can be used to affect the following aspects of Overture's
operation:

\begin{itemize}
  \item The parser (for example) to enable or disable new language
  features.
  \item The type checker (for example) to check for overrides or
  suppress warnings
  \item The interpreter (for example) to trace the execution path or
  variables' values
  \item The PO generator to (for example) skip obligations for an area
  of specification.
\end{itemize}

Note that none of these examples affect the meaning of the
specification, only the operation of the tool. Although it would be
possible to create an annotation to affect a specification's
behaviour, this is strongly discouraged.

\section{Loading and Checking}

In order to load and check annotations, the jar file containing the
annotations must be added to the classpath. As an example of this, we
will assume the existence of a file \lstinline[style=tool]|test.vdmsl|
that contains the \lstinline[language=VDM++]|@Trace| annotated
\lstinline[language=VDM++]|add| function in
Section~\ref{sec:annotations-syntax}. Now, to load annotations and
trace the parameters of this function we do as follows:

\begin{lstlisting}[style=tool,breaklines=true,escapechar=!,frame=tb,label={lst:annotations-cli-ex}]
$ !\textbf{java -cp "Overture.jar:annotations.jar" org.overture.interpreter.VDMJ -vdmsl -i test.vdmsl}!

!Parsed 1 module in 0.055 secs. No syntax errors!
!Type checked 1 module in 0.055 secs. No type errors!
!Initialized 1 module in 0.048 secs.!
Interpreter started
> p add(1,2)
!Trace: in 'DEFAULT' (test.vdmsl) at line 4:9, a = 1!
!Trace: in 'DEFAULT' (test.vdmsl) at line 4:9, b = 2!
= 3
!Executed in 0.028 secs.!
$ !\ldots!
\end{lstlisting}

An annotation will be checked whenever it is available on the
classpath. So the easiest way to disable checking of annotations is
simply to remove the annotations jar from the classpath.

\section{Standard Annotations}\label{sec:standard-annotations}

Overture includes some standard annotations. They are provided in a
separate jar file which needs to be on the classpath when Overture is
started. If the jar is not on the classpath, annotations will be
silently ignored. The standard annotations perform the following
processing:

\subsection{@Override}

This annotation is similar to the Java
\lstinline[language=Java]|@Override| annotation, which is used to
verify that a Java method overrides a superclass method, raising an
error if not. In Overture, the override applies to operations or
functions in a VDM++ class. The typecheck of the annotation verifies
that the dialect is VDM++, that the annotation has no arguments, and
that it is applied to either an operation or a function
definition. Lastly, if there is no ``inherited'' definition that this
definition overrides, an error is
raised, e.g.:\\

\noindent \texttt{Error 6004: Definition does not @Override superclass
  in 'A'\\ (test.vpp) at line 3:9}\\

\noindent The annotation has no effect on the interpreter or the PO
generator.

\subsection{@Trace}

The \lstinline[language=VDM++]|@Trace| annotation is intended to trace
the flow of control in the interpreter, either to note that a
particular point in the execution has been reached, or to log the
values of variables at that location. The typechecker checks that the
annotation is applied to a statement or an expression only. The check
also makes sure that any arguments supplied are simple variable names
and refer to variables that are in scope. When the interpreter is
running it either just prints out the current location to stderr, or
it prints the location and
``\textless{}name{}\textgreater{}={}\textless{}value{}\textgreater''
for each argument (i.e.\ each variable name traced). For example, the
specification in Section~\ref{sec:annotations-syntax} produces the
following:

\begin{lstlisting}[style=tool,escapechar=!]
$ p !\textbf{add(1,2)}!
Trace: !in! 'A' (test.vdm) at line 9:13, A`a = 1
Trace: !in! 'A' (test.vdm) at line 9:13, A`b = 2
= 3
Executed !in! 0.007 secs.
\end{lstlisting}

\noindent The annotation has no effect on the execution values or the
PO generator.

\subsection{@NoPOG}

The \lstinline[language=VDM++]|@NoPOG| annotation is intended to
suppress PO generation over a region of the specification. It can be
applied to a class/module, a definition, a statement or an
expression. The typechecker checks
that the annotation has no arguments passed, e.g.:\\

\noindent \texttt{Error 6000: @NoPOG has no arguments in 'A'
  (test.vdm) at line 9:13}\\

The PO generator clears the list of proof obligations generated for
the annotated element (class/module, definition, statement or
expression). The annotation has no effect on the interpreter.

\subsection{@Printf}

The \lstinline[language=VDM++]|@Printf| annotation is almost identical
to the \lstinline[language=VDM++]|IO`printf| library operation, except
that as an annotation it can be used in functions as well as
operations, and the arguments do not have to be presented as a
sequence literal. The typecheck checks that the annotation has a
string as its first argument. Execution of the annotation evaluates
the arguments and then passes the values generated to
\lstinline[language=Java]|System.out.printf|. Note that, as with
\lstinline[language=VDM++]|IO`printf|, the format string can only
contain \%s converters, even if the values being printed are numeric.

\subsection{@OnFail}

The \lstinline[language=VDM++]|@OnFail| annotation is virtually the
same as the \lstinline[language=VDM++]|@Printf| annotation, except
that it can only be used to annotate boolean expressions --- and the
expression that follows should be bracketed to make it clear. The
annotation will only print the output if the evaluation of the
expression is false. This is useful to add at various points in
complex boolean expressions, such as large pre/postconditions or type
invariants. For example:

\begin{lstlisting}[language=VDM++, frame=tlbr, escapechar=!]
inv !\textbf{mk\_}!R(p, q) ==
  -- @OnFail("p=%s, should be <10", p)
  (p < 10)
  -- @OnFail("p=%s, should be in PSET", p)
  and (p in set PSET)
  -- @OnFail("q=%s, should be >10", q)
  and (q > 10)
  -- @OnFail("q=%s, should be in QSET", q)
  and (q in set QSET);
\end{lstlisting}

If this type invariant is violated, the error message indicates that
the error is where the invalid value is generated (the console, here),
rather than where in the invariant that the boolean expression
fails. The \lstinline[language=VDM++]|@OnFail| annotations catch the
failing sub-clause and print a helpful message:

\begin{lstlisting}[style=tool,escapechar=!, breaklines=true]
> p mk_R(10,2)
p=10, should !be! <10 !in! 'DEFAULT' (test.vdm) at line 10:13
Error 4079: Type invariant violated !by! mk_R arguments !in! 'DEFAULT' (console)

> p mk_R(5,10)
p=5, should !be! !in! PSET !in! 'DEFAULT' (test.vdm) at line 13:13
Error 4079: Type invariant violated !by! mk_R arguments !in! 'DEFAULT' (console)

> p mk_R(1,2)
q=2, should !be! >10 !in! 'DEFAULT' (test.vdm) at line 16:13
Error 4079: Type invariant violated !by! mk_R arguments !in! 'DEFAULT' (console)

> p mk_R(1,15)
q=15, should !be! !in! QSET !in! 'DEFAULT' (test.vdm) at line 19:13
Error 4079: Type invariant violated !by! mk_R arguments !in! 'DEFAULT' (console)
\end{lstlisting}

\section{Creating New Annotations}

It is possible for users to add their own annotations although this
requires some knowledge of the Overture internals. Specifically how to
traverse Overture's internal representation of the model, referred to
as the abstract syntax tree. For more information about tree traversal
using visitors, the reader is referred to the Overture Visitor
Guide~\cite{VisitorGuide}. Moreover, Overture provides several
examples to support the creation of new annotations such as the
standard annotations described above~\cite{AnnotationsProvided} and
other more ``exotic'' examples that (say) require extending the VDM
parser~\cite{AnnotationsExamples}.

%%% Local Variables:
%%% mode: latex
%%% TeX-master: "../OvertureIDEUserGuide"
%%% End:
