% Overture Combinatorial Testing Guidelines
\documentclass{overturerepchap}
\usepackage{url}
\usepackage{graphics}
\usepackage{times}
\usepackage{listings}
\usepackage{color}
\usepackage{graphicx}
\usepackage{latexsym}
\usepackage{longtable} % ,multirow
\usepackage{vdmsl-2e}
\usepackage{makeidx}
\usepackage{hyperref}
\usepackage{ifthen}
\usepackage{makeidx}
\usepackage{fancyhdr}

\newcommand{\nonstandard}[1]{}
\newcommand{\vdmtools}{VDMTools}
\newcommand{\vdmstyle}[1]{\texttt{#1}}
\newcounter{exerciseno}
\newcommand{\vdmsl}{VDM-SL}
\newcommand{\vdmpp}{VDM++}
\newcommand{\Index}[1]{#1\index{#1}}

%\newcommand{\Exercise}[1]{%
%    \textbf{Exercise \thechapter.\theexerciseno}
%   \refstepcounter{exerciseno} #1 $\Box$\\ }%}
%\newcommand{\initexercise}{\setcounter{exerciseno}{0}}
%\newcounter{exerciseno}
\newcommand{\MYEQUIV}{$\equiv$}
\newenvironment{TypeSemantics}{\begin{longtable}[r]{|p{3.5cm}|p{9cm}|}\hline%
  Operator Name & Semantics Description \\ \hline\hline \endhead}%
  {\hline\end{longtable}}

\newcommand\thebookexercise{\thechapter.\arabic{exerciseno}}
\newenvironment{myexercise}{\par
  \refstepcounter{exerciseno}%
  \indent\textbf{Exercise\ \thebookexercise}\enskip}{$\Box$\\
}
\newenvironment{myhardexercise}{\par
  \refstepcounter{exerciseno}%
  \indent\textbf{Exercise\ \thebookexercise $\star$}\enskip}{$\Box$\\
}
\newcommand{\initexercise}{\setcounter{exerciseno}{0}}
%\newenvironment{mysolution}{
%} %% This will be replaced by a perl script extracting the solutions
  %% and inserting them automatically into the solutions chapter.
\newcommand{\insertfig}[4]{ % Filename, epsheight, epswidth, caption,  label
\begin{figure}[htb]
\begin{center}
\includegraphics[width=#2]{#1}
\end{center}
\caption{{\em #3}} #4
\end{figure}
}

\newcommand{\insertfignw}[3]{ % Filename, caption,  label
\begin{figure}[h]
\begin{center}
\includegraphics{#1}
\end{center}
\caption{{\em #2}} #3
\end{figure}
}

\newcommand{\insertfignwrot}[3]{ % Filename, caption,  label
\begin{figure}[h]
\begin{center}
\scalebox{1}%
{\rotatebox{270}{%
\includegraphics{#1}}}
\end{center}
\caption{{\em #2}} #3
\end{figure}
}

%\newcommand{\insertcommentedvdm*}[2]{}
  %% This macro is identified by perl script which will move the
  %% parameter to the solutions chapter.
% definition of VDM++, JavaCC, JJTree, JTB, ANTLR and SableCC for listings
\include{customlangdef}
% define the layout for listings
\lstdefinestyle{tool}{basicstyle=\ttfamily,
                         frame=trBL,
			 showstringspaces=false,
			 frameround=ffff,
			 framexleftmargin=0mm,
			 framexrightmargin=0mm}
\lstdefinestyle{mystyle}{basicstyle=\ttfamily,
                         frame=trBL,
%                         numbers=left,
%			 gobble=0,
			 showstringspaces=false,
%			 linewidth=\textwidth,
			 frameround=fttt,
			 aboveskip=5mm,
			 belowskip=5mm,
			 framexleftmargin=0mm,
			 framexrightmargin=0mm}
%\lstdefinestyle{mystyle}{basicstyle=\sffamily\small,
%			 frame=tb,
%                         numbers=left,
%			 gobble=0,
%			 showstringspaces=false,
%			 linewidth=345pt,
%			 frameround=ffff,
%			 framexleftmargin=8mm,
%			 framexrightmargin=8mm,
%			 framextopmargin=1mm,
%			 framexbottommargin=1mm,
%			 aboveskip=7mm,
%			 belowskip=5mm,
%			 xleftmargin=10mm,}

\lstset{style=mystyle}
\lstset{language=VDM++}
\lstset{alsolanguage=Java}
% The command below enables you to escape into normal LaTeX mode inside your 
% VDM chunks by starting with a `!\UTF{FFFD}\UTF{FFFD}? character and ending with a `!\UTF{FFFD}\UTF{FFFD}?
\lstset{escapeinside=!!}

%\newcommand{\kw}[1]{{\tt #1}}

%%%%%%%%%%%%%%%%%% Commands for bibtex %%%%%%%%%%%%%%%%%%%%%%%
%************************************************************************
%                                                                       *
%       Bibliography and Terminology supporting commands                *
%                                                                       *
%************************************************************************

\newcommand{\bthisbibliography}[1]{\chapter*{References}%
   \begin {list} {}%
     {\settowidth {\labelwidth} {[#1]XX}%
      \setlength {\leftmargin} {\labelwidth}%
      \addtolength{\leftmargin} {\labelsep}%
      \setlength {\parsep} {1ex}%
      \setlength {\itemsep} {2ex}%
     }
  }
\newcommand{\ethisbibliography}{\end{list}}
\newcommand{\refitem}[2]
  {\bibitem[#1]{#2}}

% Requirements environment
\newenvironment{reqs}{%
\begin{enumerate}
%\renewcommand{\labelenumi}{\textbf{R\theenumi}}
\renewcommand{\theenumi}{\textbf{R\arabic{enumi}}}
}{%
\end{enumerate}}

%\newcommand{\Exercise}[1]{%
%    \textbf{Exercise \thechapter.\theexerciseno}
%   \refstepcounter{exerciseno} #1 $\Box$\\ }%}
%\newcommand{\initexercise}{\setcounter{exerciseno}{0}}
\newcommand{\Lit}[1]{`{\tt #1}\Quote}
\newcommand{\Rule}[2]{
  \begin{quote}\begin{tabbing}
    #1\index{#1}\ \ \= = \ \ \= #2  ; %    Adds production rule to index

  \end{tabbing}\end{quote}
  }
\newcommand{\RuleTarget}[1]{\hypertarget{rule:#1}{}}
\newcommand{\Ruledef}[2]
{
  \RuleTarget{#1}\Rule{#1}{#2}%
  }
\newcommand{\Ruleref}[1]{
  \hyperlink{rule:#1}{#1}}
\newcommand{\SeqPt}[1]{\{\ #1\ \}}
\newcommand{\lfeed}{\\ \> \>}
\newcommand{\dsepl}{\ $|$\ }
\newcommand{\dsep}{\\ \> $|$ \>}
\newcommand{\Lop}[1]{`{\bf\ttfamily #1}\Quote}
\newcommand{\blankline}{\vspace{\baselineskip}}
\newcommand{\Brack}[1]{(\ #1\ )}
\newcommand{\nmk}{\footnotemark}
\newcommand{\ntext}[1]{\footnotetext{{\bf Note: } #1}}
\newlength{\kwlen}
\newcommand{\Keyw}[1]{\settowidth{\kwlen}{\tt
    #1}\makebox[\kwlen][l]{{\bf\ttfamnily #1}}}
\newcommand{\keyw}[1]{{\bf\ttfamily #1}}
\newcommand{\id}[1]{{\tt #1}}
\newcommand{\metaiv}[1]{\begin{lstlisting}\input{#1}\end{lstlisting}}

\newcommand{\OptPt}[1]{[\ #1\ ]}
\newcommand{\MAP}[2]{\keyw{map }#1\keyw{ to }#2}
\newcommand{\INMAP}[2]{\keyw{inmap }#1\keyw{ to }#2}
\newcommand{\SEQ}[1]{\keyw{seq of }#1}
\newcommand{\NSEQ}[1]{\keyw{seq1 of }#1}
\newcommand{\SET}[1]{\keyw{set of }#1}
\newcommand{\SETONE}[1]{\keyw{set1 of }#1}
\newcommand{\PROD}[2]{#1 * #2}
\newcommand{\TO}[2]{#1 $\To$ #2}
\newcommand{\FUN}[2]{#1 $\To$ #2}
\newcommand{\PUBLIC}{\ifthenelse{\boolean{VDMpp}}{public\mbox{}}{\mbox{}}}
\newcommand{\PRIVATE}{\ifthenelse{\boolean{VDMpp}}{private}{\mbox{}}}
\newcommand{\PROTECTED}{\ifthenelse{\boolean{VDMpp}}{protected}{\mbox{}}}

\newcommand{\experience}[1]{%
\begin{center}
\fbox{
\begin{minipage}[t]{.8\textwidth}
#1
\end{minipage}}
\end{center}}


\pagestyle{fancy}
\fancyhead{}
\fancyhead[LO]{\leftmark}
\fancyhead[RE]{The VDM Combinatorial Testing Guidelines}
\fancyhead[RO,LE]{\resizebox{0.05\textwidth}{!}{\includegraphics{overture}}}
\fancyfoot[C]{\thepage}
\makeindex

\begin{document}

\title{Guidelines for using VDM Combinatorial Testing Features}
\author{Nick Battle\\
            Peter Gorm Larsen}

\date{December 2020}
\reportno{TR-005}

\pagenumbering{roman}
\maketitle


{\textbf{Document history}}

\begin{tabular}{|l|l|l|l|l|}\hline
Month   & Year & Version & Version of Overture.exe & Comment \\ \hline
December   & 2010 &    0.1     & 3.0.2   & Initial version\\ \hline
\end{tabular}

%\addtocounter{page}{2}
\tableofcontents
\newpage
\mbox{}
\newpage
\pagenumbering{arabic}
\setcounter{page}{1}

\chapter{Introduction}

This manual is a complete guide the combinatorial testing of VDM models. It
assumes the reader has no prior knowledge of combinatorial testing, but a
working knowledge of VDM, in particular, the VDM++ and VDM-SL dialects.

\section{What is Combinatorial Testing?}

Creating a comprehensive set of tests for VDM specifications can be a time
consuming process. To try to make the generation of test cases simpler, Overture
provides a VDM language extension (for all dialects) called Combinatorial
Testing \cite{Nie&11,Larsen&10c}.

In general, specifications are tested to verify that certain properties or
behaviours are met, as specified by constraints in the specification and
validation conjectures in the tests.

The simplest way to test a specification is to write ad-hoc tests, starting from
a known system state and proceeding with a sequence of operation calls that
should move to a new state or produce some particular result or error response.
The problem with this kind of testing is that it can be very laborious to produce
the number of tests needed to cover the complete system behaviour. It is also
expensive to maintain a large test suite as the specification evolves.

The most complete way to test a specification is to produce a formal
mathematical proof that it will never violate its constraints, and always meet
its validation conjectures if presented with a legal sequence of operation
calls. This provides the highest level of confidence in the correctness of a
specification, but it can be unrealistic to produce a complete formal proof for
complex specifications, even with tool support \cite{Paulson97,Bicarregui&94}.

Model checking provides an approach to formal testing that is considerably
better than ad-hoc testing but not as complete as formal proof \cite{Clarke&99}. This approach
uses a formal specification of the system properties desired, often written in a
temporal calculus, and the model checker symbolically executes the
specification searching for execution paths that violate the constraints. Since
the execution is symbolic, extremely large state spaces can be searched
(billions of cases is not uncommon), and failed cases can produce a ``counter
example'' that demonstrates the failure. This is a very powerful technique, but
in practice, realistic specifications often produce a state space explosion
that is too great for model checkers.

Combinatorial testing is an approach that is far more powerful than ad-hoc
testing, but not as complete as model checking. Tests are produced automatically
from ``\texttt{\textbf{traces}}'' that are relatively simple to define. The approach allows
specifications to be tested with perhaps millions of test cases, but cannot
guarantee to catch every corner case in the way that a model checker can.
Therefore the technique is useful for specifications that are too complex for
model checking or formal proof.

A combinatorial trace is a pattern that describes the construction of
argument values and the sequences of operation calls that will exercise the
specification. A specification may contain several traces, each designed to test
a particular aspect. Traces are automatically expanded into a (potentially
large) number of tests, each of which is a particular sequence of operation
calls and argument values. The execution of tests is performed
automatically, starting each in a known state; a test is considered to pass if
it does not violate the specification's constraints, or the test's validation
conjectures. Individual failed tests can be executed in isolation to find out
why they failed, which is similar to a model checker's counter example. 

\section{The Structure of this Document}

The document is intended to be read sequentially, but readers who are familiar
with the basics of VDM traces can skip Chapter \ref{chap:traces} and go straight
to the patterns in Chapter \ref{chap:patterns}.

\begin{itemize}
\item Chapter \ref{chap:traces}, \nameref{chap:traces} introduces the concept of
a combinatorial test and explains how to work with traces in VDM.
\item Chapter \ref{chap:patterns}, \nameref{chap:patterns} looks at common ways
of using traces that are useful in many different situations.
\item Chapter \ref{chap:examples}, \nameref{chap:examples} uses two more
significant specifications to demonstrate typical usage of traces in real life.
\item Appendix \ref{chap:CTsyntax}, \nameref{chap:CTsyntax} defines the formal
grammar of traces.
\item Appendix \ref{chap:OvertureShots}, \nameref{chap:OvertureShots} includes
screen shots of the Combinatorial Testing perspective in Overture.
\item Appendix \ref{chap:listings}, \nameref{chap:listings} contains the full
listings of the models explored in Chapter \ref{chap:examples}.
\end{itemize}

\chapter{Working with Traces}
\label{chap:traces}

\section{Basic Trace Constructs}
Combinatorial tests are embedded within a VDM specification using a section
called ``\texttt{\textbf{traces}}''. Typically, one or more traces are added to a separate class
or module that is intended for testing rather than the main specification,
though you can add traces to any class you wish. In this chapter, we will use
the example classes below:

\small
\begin{lstlisting}
class Counter
instance variables
  total:int := 0;

operations
  public inc: () ==> int
  inc() == ( total := total + 1; return total; )

  public dec: () ==> int
  dec() == ( total := total - 1; return total; )

end Counter

class Tester
instance variables
  obj:Counter := new Counter();

traces
  T1: obj.inc();

end Tester
\end{lstlisting}
\normalsize

Notice that there are two classes, \texttt{Counter} and \texttt{Tester}. The \texttt{Counter} class defines a
simple operation that increments and decrements a total state value that is
initially zero. The \texttt{Tester} class creates an instance of Counter and defines a
single trace called \texttt{T1}. Trace names are simple identifiers, optionally separated
by slashes (e.g.\ \texttt{item456/interface/all}). This example is the simplest
trace possible and indicates that the trace should expand to a single test that
just calls \texttt{obj.inc()}.

This trace can either be executed in Overture in the Combinatorial Testing
perspective (see Appendix \ref{chap:OvertureShots}), or it can be executed from
the command line using the \texttt{runtrace} command. The command line output is
illustrated here for simplicity:

\lstset{style=tool,language=}
\begin{lstlisting}[escapechar=@]
> runtrace Tester`T1
Generated 1 tests in 0.004 secs. 
Test 1 = obj.inc()
Result = [1, PASSED]
Executed in 0.007 secs. 
All tests passed
\end{lstlisting}
\lstset{style=mystyle}
\lstset{language=VDM++}

The first line of output indicates that one test has been generated from the
trace. With more complex examples, this generation could expand to thousands or
millions of tests, and consequently it may take a few seconds.

The next line of the output describes the test that was generated. Note that
this is called ``Test 1'', and consists of a single call to \texttt{obj.inc()}.

The line below the test gives the result of executing that test. There is a
single return value from the call to \texttt{obj.inc()}, 1, which is listed
along with the word ``PASSED'' that indicates that there were no constraint violations
in the test execution.

Lastly the time taken to execute all of the tests is given, and an indication of
whether any tests failed.

The reason that this trace only expands to a single test is that the trace, when
considered as a pattern, only matches a single operation call. But if we change
the trace to the following:

\small
\begin{lstlisting}
traces
  T1: obj.inc() | obj.dec();
\end{lstlisting}
\normalsize

The trace is now saying that it would match either a call to \texttt{obj.inc()}
or a call to \texttt{obj.dec()}. Therefore the test expansion produces the
following:

\lstset{style=tool,language=}
\begin{lstlisting}[escapechar=@]
> runtrace Tester`T1
Generated 2 tests
Test 1 = obj.inc()
Result = [1, PASSED]
Test 2 = obj.dec()
Result = [-1, PASSED]
Executed in 0.033 secs. 
All tests passed
\end{lstlisting}
\lstset{style=mystyle}
\lstset{language=VDM++}

This time two tests are generated. The first calls \texttt{obj.inc()}, the
second \texttt{obj.dec()}. Notice that the decrement test is completely separate
from the increment test. It produces -1 as its result, because the \texttt{Counter}
object is re-created for each test. It does not decrement the counter back to
zero after the first test incremented it.

If we want to test an increment followed by a decrement, that would be expressed
using a semi-colon separator:

\small
\begin{lstlisting}
traces
  T1: obj.inc(); obj.dec();
\end{lstlisting}
\normalsize

This produces the output:

\small
\lstset{style=tool,language=}
\begin{lstlisting}[escapechar=@]
Generated 1 tests
Test 1 = obj.inc(); obj.dec()
Result = [1, 0, PASSED]
Executed in 0.027 secs. 
All tests passed
\end{lstlisting}
\lstset{style=mystyle}
\lstset{language=VDM++}
\normalsize

This generates a single test again, but you can see that the test involves two
calls and that they return 1 and 0, respectively. So this time the second call
is operating on the same object instance as the first.

If the increment and decrement operations are independent, it makes sense to
test calls to them in either order, which would be expressed as:

\small
\begin{lstlisting}
traces
  T1: || ( obj.inc(), obj.dec() );
\end{lstlisting}
\normalsize

Notice that the separator has changed to a comma. That produces the output:

\small
\lstset{style=tool,language=}
\begin{lstlisting}[escapechar=@]
Generated 2 tests
Test 1 = obj.inc(); obj.dec()
Result = [1, 0, PASSED]
Test 2 = obj.dec(); obj.inc()
Result = [-1, 0, PASSED]
Executed in 0.029 secs. 
All tests passed
\end{lstlisting}
\lstset{style=mystyle}
\lstset{language=VDM++}
\normalsize

So now both orderings of the two calls are produced. This is because there are
two orderings that match the pattern \texttt{|| ( \ldots, \ldots )}. This
particular trace construct naturally expands to an arbitrary number of calls and
produces a test for every permutation of the calls in brackets.

But what if some tests are a pair of calls and some are not? If we want to make
a call optional, the \texttt{?} operator can be added to any operation call
(i.e.\ not just within \texttt{||} operators) to indicate that this will match
tests where the call is made and where it is not. For example:

\small
\begin{lstlisting}
traces
  T1: || ( obj.inc(), obj.dec()? );
\end{lstlisting}
\normalsize

This means that the decrement call is optional and so although it is included in
the orderings of the pair, it should also be absent in some cases. This
example produces the following:

\small
\lstset{style=tool,language=}
\begin{lstlisting}[escapechar=@]
Generated 4 tests
Test 1 = obj.inc(); skip
Result = [1, (), PASSED]
Test 2 = obj.inc(); obj.dec()
Result = [1, 0, PASSED]
Test 3 = skip; obj.inc()
Result = [(), 1, PASSED]
Test 4 = obj.dec(); obj.inc()
Result = [-1, 0, PASSED]
Executed in 0.043 secs. 
All tests passed
\end{lstlisting}
\lstset{style=mystyle}
\lstset{language=VDM++}
\normalsize

You see that the decrement call is sometimes present and sometimes replaced by
\texttt{\textbf{skip}}, which indicates the absence of an optional call. Notice also that
the \texttt{||} operator and the \texttt{?} operator work together to combine their
effects in this example, though \texttt{?} can be used for any operation call.

Along the same lines as \texttt{?}, it is possible to add \texttt{*} and
\texttt{+} operators to any call, which indicate that it should be called zero
or more times, and one or more times. The maximum number of times is a tool preset
value that defaults to 5, though it can be changed. So for example:

\small
\begin{lstlisting}
traces
  T1: obj.inc()*;
  T2: obj.dec()+;
\end{lstlisting}

\lstset{style=tool,language=}
\begin{lstlisting}[escapechar=@]
> runtrace Tester`T1
Generated 6 tests
Test 1 = skip
Result = [(), PASSED]
Test 2 = obj.inc()
Result = [1, PASSED]
Test 3 = obj.inc(); obj.inc()
Result = [1, 2, PASSED]
Test 4 = obj.inc(); obj.inc(); obj.inc()
Result = [1, 2, 3, PASSED]
Test 5 = obj.inc(); obj.inc(); obj.inc(); obj.inc()
Result = [1, 2, 3, 4, PASSED]
Test 6 = obj.inc(); obj.inc(); obj.inc(); obj.inc(); obj.inc()
Result = [1, 2, 3, 4, 5, PASSED]
Executed in 0.046 secs. 
All tests passed

> runtrace Tester`T2
Generated 5 tests
Test 1 = obj.dec()
Result = [-1, PASSED]
Test 2 = obj.dec(); obj.dec()
Result = [-1, -2, PASSED]
Test 3 = obj.dec(); obj.dec(); obj.dec()
Result = [-1, -2, -3, PASSED]
Test 4 = obj.dec(); obj.dec(); obj.dec(); obj.dec()
Result = [-1, -2, -3, -4, PASSED]
Test 5 = obj.dec(); obj.dec(); obj.dec(); obj.dec(); obj.dec()
Result = [-1, -2, -3, -4, -5, PASSED]
Executed in 0.042 secs. 
All tests passed
\end{lstlisting}
\lstset{style=mystyle}
\lstset{language=VDM++}
\normalsize

The important difference between these two is that \texttt{T1} includes an extra
\texttt{\textbf{skip}} case, whereas \texttt{T2} does not.

Lastly, it is possible to indicate a specific number of repetitions of a call or
a range of repetitions. For example:

\small
\begin{lstlisting}
traces
  T1: obj.inc(){3};
  T2: obj.dec(){2, 4};
\end{lstlisting}

\lstset{style=tool,language=}
\begin{lstlisting}[escapechar=@]
> runtrace Tester`T1
Generated 1 tests
Test 1 = obj.inc(); obj.inc(); obj.inc()
Result = [1, 2, 3, PASSED]
Executed in 0.026 secs. 
All tests passed

> runtrace Tester`T2
Generated 3 tests
Test 1 = obj.dec(); obj.dec()
Result = [-1, -2, PASSED]
Test 2 = obj.dec(); obj.dec(); obj.dec()
Result = [-1, -2, -3, PASSED]
Test 3 = obj.dec(); obj.dec(); obj.dec(); obj.dec()
Result = [-1, -2, -3, -4, PASSED]
Executed in 0.01 secs. 
All tests passed
\end{lstlisting}
\lstset{style=mystyle}
\lstset{language=VDM++}
\normalsize

The \texttt{T1} trace now produces a single test with precisely three repetitions, while
the \texttt{T2} trace gives three tests with 2, 3 and 4 repetitions respectively.

If you combine a \texttt{||} operator with a repetition, the result is to repeat
all of the possibilities of the permutation with the given number of repetitions.
For example:

\small
\begin{lstlisting}
traces
  T1: || ( obj.inc(), obj.dec() ) {2};
\end{lstlisting}

\lstset{style=tool,language=}
\begin{lstlisting}[escapechar=@]
> runtrace Tester`T1
Generated 4 tests
Test 1 = obj.inc(); obj.dec(); obj.inc(); obj.dec()
Result = [1, 0, 1, 0, PASSED]
Test 2 = obj.dec(); obj.inc(); obj.inc(); obj.dec()
Result = [-1, 0, 1, 0, PASSED]
Test 3 = obj.inc(); obj.dec(); obj.dec(); obj.inc()
Result = [1, 0, -1, 0, PASSED]
Test 4 = obj.dec(); obj.inc(); obj.dec(); obj.inc()
Result = [-1, 0, -1, 0, PASSED]
Executed in 0.038 secs. 
All tests passed
\end{lstlisting}
\lstset{style=mystyle}
\lstset{language=VDM++}
\normalsize

Here, the \texttt{||} operator produces (\texttt{inc}, \texttt{dec}) and (\texttt{dec}, \texttt{inc}); then the
repetition doubles this, but it doubles every combination of the two rather than
simply repeating each one twice.

\section{Using Variables}

So far, the trace examples have called operations that do not include any
arguments. Arguments can be passed as literals, but traces also provide the
means to define variables that can change value as tests are generated from a
trace.

If we overload the example increment and decrement operations with versions that
take an integer parameter, by which to change the counter, we can write traces
like this:

\small
\begin{lstlisting}
...
  public inc: int ==> int
  inc(i) == ( total := total + i; return total; );

  public dec: int ==> int
  dec(i) == ( total := total - i; return total; )

traces
  T1:
      let a in set {1, ..., 10} be st a mod 2 = 0 in
          obj.inc(a);
\end{lstlisting}

\lstset{style=tool,language=}
\begin{lstlisting}[escapechar=@]
> runtrace Tester`T1
Generated 5 tests
Test 1 = obj.inc(2)
Result = [2, PASSED]
Test 2 = obj.inc(4)
Result = [4, PASSED]
Test 3 = obj.inc(6)
Result = [6, PASSED]
Test 4 = obj.inc(8)
Result = [8, PASSED]
Test 5 = obj.inc(10)
Result = [10, PASSED]
Executed in 0.04 secs. 
All tests passed
\end{lstlisting}
\lstset{style=mystyle}
\lstset{language=VDM++}
\normalsize

In a standard VDM specification, the \texttt{\textbf{let}\ldots \textbf{be st}} expression would
choose an arbitrary element from the set that meets the \texttt{st} clause. But
in a trace context, this looseness is used as a pattern that expands to a test
covering each possible set value that would match. Notice that the tests list
the actual value of the argument passed, rather than the symbolic name, ``a''.

A trace can include multiple \texttt{\textbf{let}} clauses, but if these are nested, then
the trace expands to the \emph{combination} of the variables. For example:

\small
\begin{lstlisting}
traces
  T1:
      let a in set {1, 2, 3} in
          let b in set {4, 5, 6} in
              ( obj.inc(a); obj.dec(b) );
\end{lstlisting}

\lstset{style=tool,language=}
\begin{lstlisting}[escapechar=@]
> runtrace Tester`T1
Generated 9 tests
Test 1 = obj.inc(1); obj.dec(4)
Result = [1, -3, PASSED]
Test 2 = obj.inc(1); obj.dec(5)
Result = [1, -4, PASSED]
Test 3 = obj.inc(1); obj.dec(6)
Result = [1, -5, PASSED]
Test 4 = obj.inc(2); obj.dec(4)
Result = [2, -2, PASSED]
Test 5 = obj.inc(2); obj.dec(5)
Result = [2, -3, PASSED]
Test 6 = obj.inc(2); obj.dec(6)
Result = [2, -4, PASSED]
Test 7 = obj.inc(3); obj.dec(4)
Result = [3, -1, PASSED]
Test 8 = obj.inc(3); obj.dec(5)
Result = [3, -2, PASSED]
Test 9 = obj.inc(3); obj.dec(6)
Result = [3, -3, PASSED]
Executed in 0.066 secs. 
All tests passed
\end{lstlisting}
\lstset{style=mystyle}
\lstset{language=VDM++}
\normalsize

This example produces a test for every combination of ``\texttt{a}'' and ``\texttt{b}'' values,
which is therefore nine tests. The round brackets are needed around the pair of
operation calls because a call binds tightly to the \texttt{\textbf{let}}. Without the
brackets, you get the following scope error, referring to the ``a'' in the
second call to \texttt{obj.dec(a)}:

\small
\begin{lstlisting}
traces
  T1:
      let a in set {6, 7, 10} in
          obj.inc(a); obj.dec(a)
\end{lstlisting}
\scriptsize
\lstset{style=tool,language=}
\begin{lstlisting}[escapechar=@]
Error 3182: Name 'Tester`a' is not in scope in 'Tester' (example.vpp) at line 28:29
Type checked 2 classes in 0.12 secs. Found 1 type error
\end{lstlisting}
\lstset{style=mystyle}
\lstset{language=VDM++}
\normalsize

Note also that the variables defined are in scope
throughout the clauses below, so the ``\texttt{a}'' variable could be used to define the
set of ``\texttt{b}'' values:

\small
\begin{lstlisting}
traces
  T1:
      let a in set {1, 2, 3} in
          let b in set {a, ..., a + 2} in
              ( obj.inc(a); obj.dec(b) );
\end{lstlisting}

\lstset{style=tool,language=}
\begin{lstlisting}[escapechar=@]
> runtrace Tester`T1
Generated 9 tests
Test 1 = obj.inc(1); obj.dec(1)
Result = [1, 0, PASSED]
Test 2 = obj.inc(1); obj.dec(2)
Result = [1, -1, PASSED]
Test 3 = obj.inc(1); obj.dec(3)
Result = [1, -2, PASSED]
Test 4 = obj.inc(2); obj.dec(2)
Result = [2, 0, PASSED]
Test 5 = obj.inc(2); obj.dec(3)
Result = [2, -1, PASSED]
Test 6 = obj.inc(2); obj.dec(4)
Result = [2, -2, PASSED]
Test 7 = obj.inc(3); obj.dec(3)
Result = [3, 0, PASSED]
Test 8 = obj.inc(3); obj.dec(4)
Result = [3, -1, PASSED]
Test 9 = obj.inc(3); obj.dec(5)
Result = [3, -2, PASSED]
Executed in 0.059 secs. 
All tests passed
\end{lstlisting}
\lstset{style=mystyle}
\lstset{language=VDM++}
\normalsize

If two variables should take values from the same set of values, it is possible
to use a \texttt{multiple bind} in a trace, but not a \texttt{bind list}. For
example:

\small
\begin{lstlisting}
traces
  T1:
      let a, b in set {1, 2, 3} in
          ( obj.inc(a); obj.dec(b) );
\end{lstlisting}

\lstset{style=tool,language=}
\begin{lstlisting}[escapechar=@]
> runtrace Tester`T1
Generated 9 tests
Test 1 = obj.inc(1); obj.dec(1)
Result = [1, 0, PASSED]
Test 2 = obj.inc(2); obj.dec(1)
Result = [2, 1, PASSED]
Test 3 = obj.inc(3); obj.dec(1)
Result = [3, 2, PASSED]
Test 4 = obj.inc(1); obj.dec(2)
Result = [1, -1, PASSED]
Test 5 = obj.inc(2); obj.dec(2)
Result = [2, 0, PASSED]
Test 6 = obj.inc(3); obj.dec(2)
Result = [3, 1, PASSED]
Test 7 = obj.inc(1); obj.dec(3)
Result = [1, -2, PASSED]
Test 8 = obj.inc(2); obj.dec(3)
Result = [2, -1, PASSED]
Test 9 = obj.inc(3); obj.dec(3)
Result = [3, 0, PASSED]
Executed in 0.061 secs. 
All tests passed
\end{lstlisting}
\lstset{style=mystyle}
\lstset{language=VDM++}
\normalsize


As well as defining a variable value from a set, variables can be used to
simplify calculations that would otherwise have to be made in the arguments to
operation calls. These simpler \texttt{\textbf{let}} definitions do not increase the
number of tests generated from the trace, they just introduce new names in the
scope that follows. Multiple variable definitions can be declared in 
one \texttt{\textbf{let}} expression. For example:

\small
\begin{lstlisting}
traces
T1:
    let a in set {1, 2, 3} in
        let b = a + 1, c = a - 1 in
            ( obj.inc(b); obj.dec(c) );
\end{lstlisting}

\lstset{style=tool,language=}
\begin{lstlisting}[escapechar=@]
> runtrace Tester`T1
Generated 3 tests
Test 1 = obj.inc(2); obj.dec(0)
Result = [2, 2, PASSED]
Test 2 = obj.inc(3); obj.dec(1)
Result = [3, 2, PASSED]
Test 3 = obj.inc(4); obj.dec(2)
Result = [4, 2, PASSED]
Executed in 0.054 secs. 
All tests passed
\end{lstlisting}
\lstset{style=mystyle}
\lstset{language=VDM++}
\normalsize

If repetitions are added to a clause within a \texttt{\textbf{let}} body, they bind
tightly to the operation call rather than the entire \texttt{\textbf{let}} clause.
If you want to repeat the entire \texttt{\textbf{let}}, you have to
bracket the whole clause and add a repetition to that. For example:

\small
\begin{lstlisting}
traces
  T1:
      let a in set {1, 2, 3} in
          obj.inc(a){1, 2}
  T2:
      ( let a in set {1, 2, 3} in
          obj.inc(a) ){1, 2}
\end{lstlisting}

\lstset{style=tool,language=}
\begin{lstlisting}[escapechar=@]
> runtrace Tester`T1
Generated 6 tests
Test 1 = obj.inc(1)
Result = [1, PASSED]
Test 2 = obj.inc(1); obj.inc(1)
Result = [1, 2, PASSED]
Test 3 = obj.inc(2)
Result = [2, PASSED]
Test 4 = obj.inc(2); obj.inc(2)
Result = [2, 4, PASSED]
Test 5 = obj.inc(3)
Result = [3, PASSED]
Test 6 = obj.inc(3); obj.inc(3)
Result = [3, 6, PASSED]
Executed in 0.044 secs. 
All tests passed

> runtrace Tester`T2
Generated 12 tests
Test 1 = obj.inc(1)
Result = [1, PASSED]
Test 2 = obj.inc(2)
Result = [2, PASSED]
Test 3 = obj.inc(3)
Result = [3, PASSED]
Test 4 = obj.inc(1); obj.inc(1)
Result = [1, 2, PASSED]
Test 5 = obj.inc(2); obj.inc(1)
Result = [2, 3, PASSED]
Test 6 = obj.inc(3); obj.inc(1)
Result = [3, 4, PASSED]
Test 7 = obj.inc(1); obj.inc(2)
Result = [1, 3, PASSED]
Test 8 = obj.inc(2); obj.inc(2)
Result = [2, 4, PASSED]
Test 9 = obj.inc(3); obj.inc(2)
Result = [3, 5, PASSED]
Test 10 = obj.inc(1); obj.inc(3)
Result = [1, 4, PASSED]
Test 11 = obj.inc(2); obj.inc(3)
Result = [2, 5, PASSED]
Test 12 = obj.inc(3); obj.inc(3)
Result = [3, 6, PASSED]
Executed in 0.03 secs. 
All tests passed
\end{lstlisting}
\lstset{style=mystyle}
\lstset{language=VDM++}
\normalsize

The difference may seem subtle, but the effect is significant. T1 behaves like a
simple ``\{1, 2\}'' repetition for each of the \texttt{\textbf{let}} values, whereas T2
produces either one or two cases from the \emph{entire set} created by the \texttt{\textbf{let}}
clause.

Although the example above uses a \texttt{let <set bind>} expression, it is also
possible to use a \texttt{\textbf{let} <seq bind>} or \texttt{\textbf{let} <type bind>}. In a
trace context, the ordering that a sequence bind carries does not affect the
generation of tests; if the example had \texttt{\textbf{let} a \textbf{in seq} [1,2,3]}, the
same tests would be generated, and since tests are independent their order is not
meaningful. So sequence binds are not particularly useful in traces. However
type binds (of finite types) are a shorthand for ``all values of this type'',
which can be useful in some circumstances. This is covered later in Chapter~\ref{chap:patterns}.

\section{Tests with Errors}

The examples so far have only includes tests that \textbf{PASSED}. This means that they
completed the sequence of operation calls without violating any pre-conditions,
post-conditions, state invariants, type invariants, recursive measures or dynamic
type checks.

If a sequence of operations causes a post-condition failure, then it
is certain that there is a problem with the specification –- it should not be
possible to provoke a post-condition failure with a set of legal calls (ie. ones
which pass the pre-conditions and type invariants). On the other hand, if a 
sequence of operations violates a pre-condition, or a type or class invariant, 
then it is \emph{possible} that the specification has a problem, but it is also 
possible that the test itself is at fault (passing illegal values).

The combinatorial testing environment indicates the exit status of the test  in
the verdict returned in the last item of the results (all \texttt{PASSED} above). So if 
pre/post/invariant conditions are violated during a test, this may be set to 
\texttt{FAILED} or \texttt{INDETERMINATE}\footnote{The tools sometimes call this \texttt{INCONCLUSIVE}}. If
a test fails, then any subsequent test which starts \emph{with the same
sequence of calls} as the failed sequence will also fail. These tests are
filtered out of the remaining test sequence automatically, and not executed.

For example, if we introduce a pre- and postcondition into our example, we see
this behaviour:

\small
\begin{lstlisting}
...
  public inc: int ==> int
  inc(i) == ( total := total + i; return total; )
  pre i < 10
  post total < 20;

traces
  T1:
      let a in set {6, 7, 10} in
          obj.inc(a){1, 5}
\end{lstlisting}

\scriptsize
\lstset{style=tool,language=}
\begin{lstlisting}[escapechar=@]
> runtrace Tester`T1
Generated 15 tests
Test 1 = obj.inc(6)
Result = [6, PASSED]
Test 2 = obj.inc(6); obj.inc(6)
Result = [6, 12, PASSED]
Test 3 = obj.inc(6); obj.inc(6); obj.inc(6)
Result = [6, 12, 18, PASSED]
Test 4 = obj.inc(6); obj.inc(6); obj.inc(6); obj.inc(6)
Result = [6, 12, 18, Error 4072: Postcondition failure: post_inc in 'Counter' at line 15:16, FAILED]
Test 5 = obj.inc(6); obj.inc(6); obj.inc(6); obj.inc(6); obj.inc(6)
Test 5 FILTERED by test 4
Test 6 = obj.inc(7)
Result = [7, PASSED]
Test 7 = obj.inc(7); obj.inc(7)
Result = [7, 14, PASSED]
Test 8 = obj.inc(7); obj.inc(7); obj.inc(7)
Result = [7, 14, Error 4072: Postcondition failure: post_inc in 'Counter' at line 15:16, FAILED]
Test 9 = obj.inc(7); obj.inc(7); obj.inc(7); obj.inc(7)
Test 9 FILTERED by test 8
Test 10 = obj.inc(7); obj.inc(7); obj.inc(7); obj.inc(7); obj.inc(7)
Test 10 FILTERED by test 8
Test 11 = obj.inc(10)
Result = [Error 4071: Precondition failure: pre_inc in 'Counter' at line 14:11, INCONCLUSIVE]
Test 12 = obj.inc(10); obj.inc(10)
Test 12 FILTERED by test 11
Test 13 = obj.inc(10); obj.inc(10); obj.inc(10)
Test 13 FILTERED by test 11
Test 14 = obj.inc(10); obj.inc(10); obj.inc(10); obj.inc(10)
Test 14 FILTERED by test 11
Test 15 = obj.inc(10); obj.inc(10); obj.inc(10); obj.inc(10); obj.inc(10)
Test 15 FILTERED by test 11
Executed in 0.075 secs. 
Some tests failed or indeterminate
\end{lstlisting}
\lstset{style=mystyle}
\lstset{language=VDM++}
\normalsize

The \texttt{inc} operation now has a pre-condition that the argument must be
less than 10 and a postcondition that the resulting total must be less than 20.
The trace makes 1 to 5 calls to the \texttt{inc} operation with arguments 6, 7
and 10, respectively.

The first three tests are fine, but \texttt{Test 4} fails because the fourth call to
\texttt{inc(6)} pushes the total over the limit. This is therefore a
post-condition \texttt{FAILED} test, and the error message is listed along with the
results of the earlier operation calls. Test 5 then tries to do the same, but
adds a further call. This must fail in the same place as \texttt{Test 4}, because\texttt{Test 4}
is the ``stem'' of \texttt{Test 5}. Therefore this test is ``\texttt{FILTERED} by \texttt{Test 4}''.
Similarly, \texttt{Test 8} fails and Tests 9 and 10 are filtered by this failure.

Test 11 fails on the first call to \texttt{inc(10)}, since the argument must be
less than 10. This produces an \texttt{INDETERMINATE} error because we are not
sure whether this is a problem with the trace or the specification being tested.
Lastly, Tests 12 to 15 are filtered by Test 11, since they would behave the same
way.

At the end of the run, the \texttt{runtrace} command indicates that some tests
failed or were indeterminate, just to remind you.

\section{Trace Reduction}

A trace may expand to many millions of tests and these may take many hours to
execute. This can make large traces difficult to work with, both while the trace
is being developed and subsequently when traces are used to check that a change
to a specification is sound.

Therefore the trace system provides the means to reduce the number of tests that
are generated. This limited number can then be checked more quickly, which
naturally does not provide the confidence of a full execution, but is convenient
to work with when a sample of tests is sufficient.

Trace reduction can be achieved in four different ways:

\begin{itemize}
\item \emph{RANDOM} reduction. This is the simplest kind of reduction, and is
used to randomly select tests from the full set. For example, a 1\% RANDOM 
reduction of a trace that expands to a million tests would select 10,000 tests
from the set. The pseudo-random selection is seeded, so that a consistent subset
of the tests can be selected.
\item \emph{SHAPES\_NOVARS} reduction. The ``shape'' of a test means the sequence
of operation names (regardless of arguments passed), and the idea of shaped
reduction is to preserve \emph{at least one test} of every shape in the full
set. So for example, a test that calls \texttt{[opA(1), opB(2), opC(3)]} would
be considered the same shape as a test that calls \texttt{[opA(111), opB(222),
opC(333)]}, but different to a test that calls \texttt{[opX(1), opY(2)]}. So a
shaped reduction of 1\% of one million tests would try to select 10,000 tests,
but it also guarantees to include at least one test of every shape within the
million, even if that means the reduction is (say) 1.5\%.
\item \emph{SHAPES\_VARNAMES} reduction. The simple interpretation of a
shape above only looks at the names of the operations called. This second kind
of shaped reduction looks at the variable names used in the \texttt{let}
bindings and constants as well as the names of the operations. So with this kind
of reduction, \texttt{[opA(a), opA(b)]} is different to \texttt{[opA(x),
opA(y)]}, even if the values of ``\texttt{a}'' and ``\texttt{x}'' can sometimes be the same.
Typically, different variable names are used in different parts of a complex
trace and so this reduction method is trying to select all of the different
parts of a trace, even if the sequence of operation names produced is the same
as another part of the trace.
\item \emph{SHAPES\_VARVALUES} reduction. The third type of shaped reduction is
even more specific about what constitutes a shape, taking into account the
\emph{value} of the variables used as well as their names. So \texttt{[opA(a)]}
will be considered a different shape to another \texttt{[opA(a)]} elsewhere, as
long as ``\texttt{a}'' is bound to a different value.
\end{itemize}

The following simple trace, using the \texttt{Counter} example from previous
chapters, illustrates RANDOM reduction:

\small
\begin{lstlisting}
traces
  T1:
      let a in set {1, ..., 1000} in
      let b in set {1, ..., 1000} in
          ( obj.inc(a); obj.dec(b) );
\end{lstlisting}

\lstset{style=tool,language=}
\scriptsize
\begin{lstlisting}[escapechar=@]
> filter
Usage: filter %age | RANDOM | SHAPES_NOVARS | SHAPES_VARNAMES | SHAPES_VARVALUES | NONE
Trace filter currently 100.0% NONE (seed 0)
 
> filter random
Trace filter currently 100.0% RANDOM (seed 0)

> filter 1%
Trace filter currently 1.0% RANDOM (seed 0)

> seedtrace 1234
Trace filter currently 1.0% RANDOM (seed 1234)

> runtrace Tester`T1
Generated 1000000 tests, reduced to 10000, in 1.571 secs. 
Test 66 = obj.inc(1); obj.dec(66)
Result = [1, -65, PASSED]
Test 155 = obj.inc(1); obj.dec(155)
Result = [1, -154, PASSED]
Test 323 = obj.inc(1); obj.dec(323)
Result = [1, -322, PASSED]
Test 419 = obj.inc(1); obj.dec(419)
Result = [1, -418, PASSED]
...
Test 999815 = obj.inc(1000); obj.dec(815)
Result = [1000, 185, PASSED]
Test 999894 = obj.inc(1000); obj.dec(894)
Result = [1000, 106, PASSED]
Test 999992 = obj.inc(1000); obj.dec(992)
Result = [1000, 8, PASSED]
Executed in 4.932 secs. 
\end{lstlisting}
\lstset{style=mystyle}
\lstset{language=VDM++}
\normalsize

The ``filter'' and ``seedtrace'' commands are used to set the filtering
required, then the trace is executed as normal. But we see that the million
tests have been reduced to 10,000 (taking a few seconds). Then instead of trying
every combination of ``\texttt{a}'' and ``\texttt{b}'', the filtering selects them at random,
starting with \texttt{a = 1, b = 65} and ending with \texttt{a = 1000, b = 992}.

The next example illustrates shaped reduction. The trace joins together two
\texttt{\textbf{let}} bindings, repeating each call once and twice. This would normally
produce 36 tests.

\small
\begin{lstlisting}
traces
  T1:
      let a in set {1, 2, 3} in obj.inc(a){1,2};
      let b in set {1, 2, 3} in obj.dec(b){1,2}
\end{lstlisting}

\lstset{style=tool,language=}
\begin{lstlisting}
> filter 1
Trace filter currently 1.0% RANDOM (seed 0)

> filter shapes_novars
Trace filter currently 1.0% SHAPES_NOVARS (seed 0)

> rt Tester`T1
Generated 36 tests, reduced by SHAPES_NOVARS, in 0.001 secs. 
Test 1 = obj.inc(1); obj.dec(1)
Result = [1, 0, PASSED]
Test 2 = obj.inc(1); obj.inc(1); obj.dec(1)
Result = [1, 2, 1, PASSED]
Test 7 = obj.inc(1); obj.dec(1); obj.dec(1)
Result = [1, 0, -1, PASSED]
Test 8 = obj.inc(1); obj.inc(1); obj.dec(1); obj.dec(1)
Result = [1, 2, 1, 0, PASSED]
Executed in 0.006 secs. 
All tests passed
>
\end{lstlisting}
\lstset{style=mystyle}
\lstset{language=VDM++}
\normalsize

Here, we try to reduce this to 1\%, but using the SHAPES\_NOVARS option. A
reduction of 1\% of 36 tests ought to produce a single test (reduction will
never produce zero tests), but in fact the shaped reduction produces four. You
can see that there is one case of each ``shape'': one \texttt{inc} and one \texttt{dec}; two \texttt{inc}'s
and one \texttt{dec}, and so on. So the idea is that the shaped reduction has given a
representative sample of all of the possible shapes, disregarding variable names
or values.

Reducing this trace using SHAPES\_VARNAMES produces the same number of shapes,
since the two calls always use the same variable names. But reduction using
SHAPES\_VARVALUES regards all of the tests as different shapes, because the
variable/value/operation combinations are different in all of the tests.

In practice, the most useful reductions are RANDOM and SHAPES\_NOVARS. Random
reductions give a simple way to cut down a large number of tests. Shaped
reduction is choosing one example of every ``path'' that the trace is taking the
specification through, which is often closely related to the different use cases
that the system has.


\section{How does Trace Expansion Work?}

The sections above have given an overview of all the trace operators, and there
are some examples of combinations of operators. But to see how traces are
expanded in general, we need to look at traces from a different point of view.
The syntax of traces is deliberately made similar to the syntax of VDM-SL, but
to understand how operators combine to produce multiple tests, it helps to look
at operators as though they followed a separate ``expansion'' grammar. In the
description that follows, a \texttt{set} is a set of tests:

\begin{itemize}
\item \texttt{set = object.opname(args)}. The simplest form of a trace is a set
that comprises a single call to an operation or function with arguments. The
arguments can be symbolic, and bound to various values by the \texttt{\textbf{let}}
operator described below.
\item \texttt{set = set1; set2; ...; setn}. A set of tests may be formed from
an ordered sequence of sets. This expands to all possible selections of one test
from each of the sets. In its simplest form, this could be a sequence of
operation calls which therefore just expands to one test. But a combination of
sets of tests results in a set of the product of the sizes of those sets.
\item \texttt{set = set1 ?}. A set of tests may be formed from another set with
a \texttt{?} operator. This produces the same set, but includes a ``skip'' step.
\item \texttt{set = set1 \{n[, n]\}}. A set of tests may be formed from another
set with a \texttt{\{n\}} or \texttt{\{n1, n2\}} operator. This produces a set
with every member of the original set repeated \texttt{n} times, or between
\texttt{n1} and \texttt{n2} times (inclusive).
\item \texttt{set = set1 *|+}. A set of tests may be formed from another set
with a \texttt{*} or \texttt{+} operator. This produces another set with every member
of the original set repeated from 0 to N times (with \texttt{*}) or 1 to N times
(with \texttt{+}). The value of N is tool dependent, but defaults to 5.
\item \texttt{set = set1 | set2 | ... | setn}. A set of tests may be formed by
combining a number of other test sets with a \texttt{|} operator. This produces
a set with the union of the other sets.
\item \texttt{set = || (set1, set2, ..., setn)}. A set may be formed from the
permutations of a number of other sets. This produces a set with each
permutation of each selection of one test from each set.
\item \texttt{set = let <multiple bind> [be st <cond>] in set1}. A set of
tests may be formed from a \texttt{multiple bind}, which expands to the
substitution of all the possible the bound values in the original set.
\item \texttt{set = let <name> = <exp> [, <name2> = <exp2>, ...] in set1}. A set
of tests may be evaluated in a scope that defines name/value pairs. This does
not increase the number of tests in the set, but just binds free variables.
\end{itemize}

For example, if (for brevity) we say that a test with a single call to
\texttt{obj.opA()} is written as ``[A]'', and similarly ``[B]'' and ``[C]'' 
for other operation calls, and ``[-]'' for a skip, then we can say the 
following trace operators produce these sets of tests:

\small
\lstset{style=tool,language=}
\begin{lstlisting}[escapechar=@]
A? = { [A], [-] }
A;B = { [AB] }
A;B? = { [AB], [A] }
A* = { [-], [A], [AA], [AAA], [AAAA], [AAAAA], ... }
A+ = { [A], [AA], [AAA], [AAAA], [AAAAA], ... }
A{3} = { [AAA] }
A{1,3} = { [A], [AA], [AAA] }
A | B = { [A], [B] }
A | B? = { [A], [B], [-] )
|| (A, B, C) = { [ABC], [ACB], [BAC], [BCA], [CAB], [CBA] }
|| (A, (B;C)) = { [ABC], [BCA] }
|| (A, B+) = { [AB], [BA], [ABB], [BBA], [ABBB], [BBBA], ... }
let a in set {1,2,3} in A(a) = { [A(1)], [A(2)], [A(3)] }
let b : bool * bool in B(b) = {
    [B(mk_(true, true))], [B(mk_(true, false))],
    [B(mk_(false, true))], [B(mk_(false, false))]
}
let z = 1 in B(z) = { [B(1)] }
\end{lstlisting}
\lstset{style=mystyle}
\lstset{language=VDM++}
\normalsize

Note that the repeat limits in a trace (like \{1,3\}) must be numeric literals.
But values in a multiple bind set or sequence can be variables, either bound
earlier in the trace or other fields within scope of the trace inside the object
or module where it is defined. Similarly, the values in the right hand side of
let definitions can be variables within the trace or the object/module scope.

\section{Language Considerations}

Combinatorial tests are available for both VDM-SL and VDM++/VDM-RT. The process 
of trace expansion and execution is very similar in all cases, but there are
some differences that are described below.

\subsection{Traces in VDM-SL}

Traces are added in a ``traces'' section within a VDM-SL specification. This can
either be within one or more modules or within a flat specification. The name of
the traces in a module are implicitly exported, so they are referred to as
\texttt{<modulename>`<tracename>}. You can omit the module name if it is the
default module.

The VDM-SL specification that is equivalent to the example used above is like
this:

\small
\begin{lstlisting}
module Counter
exports all
definitions

state S of
    total:int
init s == s = mk_S(0)
end

operations
    inc: int ==> int
    inc(i) == ( total := total + i; return total; );

    dec: int ==> int
    dec(i) == ( total := total - i; return total; )
end Counter

module Tester
imports from Counter all
definitions

traces
    T1: Counter`inc(1)*;

end Tester
\end{lstlisting}
\normalsize

And in the VDM-SL command line, that would be executed as follows. Note that
Tester is not the default module, so the trace name is qualified:

\scriptsize
\lstset{style=tool,language=}
\begin{lstlisting}[escapechar=@]
> modules
Counter (default)
Tester
> runtrace Tester`T1
Generated 6 tests in 0.002 secs. 
Test 1 = skip
Result = [(), PASSED]
Test 2 = inc(1)
Result = [1, PASSED]
Test 3 = inc(1); inc(1)
Result = [1, 2, PASSED]
Test 4 = inc(1); inc(1); inc(1)
Result = [1, 2, 3, PASSED]
Test 5 = inc(1); inc(1); inc(1); inc(1)
Result = [1, 2, 3, 4, PASSED]
Test 6 = inc(1); inc(1); inc(1); inc(1); inc(1)
Result = [1, 2, 3, 4, 5, PASSED]
Executed in 0.019 secs. 
All tests passed
>
\end{lstlisting}
\lstset{style=mystyle}
\lstset{language=VDM++}
\normalsize

This trace is very similar to the VDM++ example. The \texttt{Counter} module has a
single state that is equivalent to the VDM++ ``total'' instance variable. Note
that this is reset to zero automatically before each test is executed. This is
because each test re-initialises the specification, and the module
state has an ``\texttt{\textbf{init}}'' clause that sets the total to zero.

Notice also that the operation calls are not applied to a Counter object, unlike
VDM++.

\subsection{Traces in VDM++ and VDM-RT}

Traces are added in a ``traces'' section within a VDM++ or VDM-RT
specification, inside one or more classes. In effect, the name of the trace is
a public static symbol, so it is referred to as
\texttt{<classname>`<tracename>}, as we have seen in the examples above. You
can omit the class name if that is the default class.

Although a VDM++ trace is effectively a static scope, and can call static
operations directly (similar to a VDM-SL trace), every test execution occurs in
a \emph{new instance} of the containing class -- in our examples, in a new Tester
instance. This means that objects created within the Tester's construction will
be freshly initialised and ready for use in each test run. In the example, the
Counter object \texttt{obj} is created for each test, because the instance
variable is initialised at construction.

\subsection{Expansion and Execution Considerations}

The process of running a combinatorial test has two phases: expanding the trace
to a number of test definitions; and subsequently executing those definitions.
The trace expansion typically does not take very long, since it is only
constructing a tree of iterators that are capable of generating the tests one
after another. The subsequent execution of those tests can obviously take a long
time, depending on how many there are.

We have seen (above) how the specification is initialised before test execution,
and the state of the module or class is available to the trace, but care must be
taken if operations or functions within the environment are used as part of a
trace. This is because some expressions are evaluated during trace expansion and
some during test execution. For example:

\small
\begin{lstlisting}
...
functions
  private static range: int * int -> set of int
  range(a, b) == {a, ..., b};

values
  Z = 100;

traces
  T1: let x in set range(3, 5) in
      let y in set range(x, x+2) in
          obj.inc(Z + x + y);
\end{lstlisting}
\normalsize

In this case, the \texttt{range} function is used to create a set for the
multi-binds, and this is executed during \emph{expansion}, once for ``\texttt{x}'' and three
times for ``\texttt{y}''. Similarly, the right hand side of simple let definitions are
executed during trace expansion. But the addition of \texttt{Z + x + y} in the
argument to \texttt{inc} is called during \emph{execution} (once for each test).

Trace generation starts inside a fresh object instance of the class (or
initialised module) that contains the trace. So if operations are called during
trace \emph{expansion}, these can modify state and so affect subsequent
operation calls elsewhere in the expansion. This can become very confusing, and
it is not a recommended trace design strategy! On the other hand, calling
functions as part of the trace expansion can make traces easier to understand
and can provide the means to build complex sets that would be difficult to
construct directly within the trace statements.

When a test is listed in the trace output, the arguments that are passed to
operation calls are shown as literals, if possible. As seen in the examples
here, a call to \texttt{obj.inc(1)} is shown, rather than \texttt{obj.inc(a)}. This is
possible whenever arguments can be easily evaluated. For example the \texttt{Z
+ x + y} case above would produce \texttt{Test 1 = obj.inc(106)}, which is 100 +
3 + 3. But if the argument is a more complex expression involving operation
applies or new object creation, these cannot be evaluated and so the argument
expression is listed ``as is''. For example, if the trace above is changed to
call \texttt{obj.inc(max(x, y))}, the test would be listed with ``\texttt{x}''
and ``\texttt{y}'' rather than their current values:

\small
\lstset{style=tool,language=}
\begin{lstlisting}[escapechar=@]
> runtrace Tester`T1
Generated 9 tests in 0.006 secs. 
Test 1 = inc(max(x, y))
Result = [3, PASSED]
Test 2 = inc(max(x, y))
Result = [3, PASSED]
...
\end{lstlisting}
\lstset{style=mystyle}
\lstset{language=VDM++}
\normalsize

\chapter{Combinatorial Testing Patterns}
\label{chap:patterns}

\section{Data and Process Traces}
\label{chap:dataprocess}

Traces fall into two broad categories, although you can have a mixture
of both in a specification. One category is focussed on the behaviour of a
single function or operation when presented with a large variety of different
data structures; the other category is focussed on the process behaviour of a
system when exercised by a large number of different operation call paths.

In one sense this distinction is artificial. You can create a trace that
explores many process paths and also creates a wide variety of data values to
pass to the operations called on the way. But these aspects of a specification's
behaviour are often separable, and little advantage is gained by trying to test
everything at once. 

The patterns described in this chapter are generally useful in one or other of
the trace styles. The styles are also illustrated in more detail by the
examples in Chapter \ref{chap:examples}.

\section{Traces and Test Operations}
\label{chap:testops}

After expanding a trace to all of the call sequences that match, a
test is ultimately just a sequence of operation or function calls. But these
calls do not have to be the primary operations that drive the specification. In
some cases, it makes sense for traces to expand to a set of tests that call
\emph{test operations} that maintain their own state and exercise the main
specification, checking the responses and the main state. Checks like this, that
do not directly form part of the constraints of the specification, are usually
called \emph{validation conjectures}. The task of the testing operations is
therefore to check that, whatever the call sequence made by the trace, the
validation conjectures for the specification are maintained.

For example, a specification may describe how a sequence of parts are produced
as calls are made to \texttt{newPart(), adjustPart(), completePart()}. The
process of creation of individual parts may well involve preconditions,
postconditions, type invariants and so on. But there may also be a requirement
that (say) over time, the total number of parts of type A and type B never
differ by more than a tolerance. This is a validation conjecture: it is not
directly stated in the constraints of the specification, but it is a behaviour
that must be manifest by the system over time. Therefore the trace(s) for such
a system can call the main operations via test operations, like
\texttt{testNewPart()} and so on, and the \texttt{testCompletePart()} operation
can check the history of parts created to validate that the tolerance is always
respected, regardless of the sequence of operations that the trace tries. Note
that these test operations can maintain their own state that is private and
separate from that of the main specification.

In such cases, it makes sense to add all of the test operations to a separate
class or module, to make it clear that they are not part of the main
specification. This is illustrated in the Basket Service example in Chapter
\ref{chap:examples}.

\section{Common Patterns}

Experience has shown that the testing of many specifications with traces
requires several common ``patterns'' to create data selections or sequences of
operations. These are presented in this section.

\subsection{Sets for ``let'' bindings}

The set of values that is used by a \texttt{\textbf{let} multiple-bind} is usually
shown as an enumeration of literals in examples. But the set value can use any
VDM expression that yields a set. The following cases are generally useful:

\begin{itemize}
  \item Set comprehensions can be used to select values from a larger set that
  meet the membership predicate. This is similar to the use of the \texttt{\textbf{be
  st}} clause, but the filter acts on the members of the set rather than the
  values that are bound:
\small
\begin{lstlisting}
  let pair in set {mk_(a, b) | a, b in set VALUES & a > b} in ...
\end{lstlisting}
\normalsize
  \item The \texttt{power} operator can be used to produce all possible subsets
  from another set, though you often have to eliminate the empty set, which is
  produced by the operator:
\small
\begin{lstlisting}
  let options in set power ALL_OPTIONS \ {{}} in ...
\end{lstlisting}
\normalsize
  \item If you are using finite types (i.e. types that have a finite number of
  values), then you can use a multiple type bind to conveniently choose all of
  the possible values of that type:
\small
\begin{lstlisting}
  let p1, p2 : Product in ...
\end{lstlisting}
\normalsize
  \item Often elements have to be chosen from a set such that they are different
  to previous selections from the same set. This can sometimes be done in a
  single \texttt{\textbf{let}} bind by using a \texttt{\textbf{be st}} clause that states that the
  values are different. But a common usage is to make a selection from a set
  that has had the first choice(s) eliminated:
\small
\begin{lstlisting}
  let a, b in set S be st a <> b in ...
  let c in set S \ {a, b} in ...
  let d in set S be st d not in set {a, b, c} in ...
\end{lstlisting}
\normalsize
  \item A common requirement is to select permutations of a set. This can be
  done by using the looseness of a set bind. Note the cardinality check in the
  k-permutation example (k=3), and the use of the set pattern and the
  set-of-set \texttt{\{S\}} in the full permutation example - in this case you
  have to know the size of the set to match the pattern:
\small
\begin{lstlisting}
-- Every k-permutation of 3 values from S
  let p1, p2, p3 in set S be st card {p1, p2, p3} = 3 in ...
-- Permutations of all values from S
  let {p1, p2, p3, p4, p5} in set {S} in ...
\end{lstlisting}
\normalsize
  \item It is sometimes useful to be able to select all k-combinations from
  a set, rather than k-permutations. This is similar, but with a \texttt{\textbf{be
  st}} discriminator that selects one unique combination from the possible orderings
  (here, \texttt{c1 < c2 \textbf{and} c2 < c3}):
\small
\begin{lstlisting}
-- Every k-combination of 3 values from S
  let c1, c2, c3 in set S be st card {c1, c2, c3} = 3
      and c1 < c2 and c2 < c3 in ...
\end{lstlisting}
\normalsize
\end{itemize}

\subsection{Bracketing operations with headers and trailers}

A very common requirement is to make a number of fixed setup calls,
followed by a large number of different test calls, followed by a number of
fixed closedown calls. This pattern emerges naturally from a semi-colon
separated trace sequence with a complex ``middle'':
\scriptsize
\begin{lstlisting}
traces
  T:
    SetSystemDate(221220);
    LoadCertificates(RefData);
        test1() |
        test2() |
        test3() |
        test4() |
        test5() |
        test6() |
        test7();
    EndTransaction();
\end{lstlisting}
\normalsize
This produces a set of seven tests, calling \texttt{test1} to \texttt{test7},
each of which is sandwiched by calls that set up the system and close down the
transaction. Clearly the body containing the alternative tests can be
arbitrarily complicated, for example using variable binds to generate many
possibilities.

\subsection{Graph searching by traces}

Traces that explore all of the possible uses cases of a system frequently
need to search a graph that describes the possible paths through the use case.
This is translated into a trace by generating a call for each step through the
graph, where the trace expands to the possibilities at each step. For example:
\scriptsize
\begin{lstlisting}
values
  TESTSTATES : map TestState to map Event to TestState =
    {
        <READY> |->
        {
            <SEND>    |-> <SENT>
        },

        <SENT> |->
        {
            <RX_NACK> |-> <RESEND1>,
            <RX_ACK>  |-> <END>,
            <TIMEOUT> |-> <RESEND1>,
            <PARTIAL> |-> <SENT>
        },
        ...
    }

traces
  AllTransitions:
      let s1 = <READY> in
      let ev1 in set dom TESTSTATES(s1) in
      let s2 = TESTSTATES(s1)(ev1) in
      let ev2 in set dom TESTSTATES(s2) in
      let s3 = TESTSTATES(s2)(ev2) in
      let ev3 in set dom TESTSTATES(s3) in
      let s4 = TESTSTATES(s3)(ev3) in
      let ev4 in set dom TESTSTATES(s4) in
      let s5 = TESTSTATES(s4)(ev4) in
      let ev5 in set dom TESTSTATES(s5) in
          execute([ev1, ev2, ev3, ev4, ev5]);
 
\end{lstlisting}
\normalsize
Here the system starts in state \texttt{s1}, which must be \texttt{READY}. From
there, a number of events are possible given by the domain of a lookup of the
state in the \texttt{TESTSTATES} map. One of these events is selected as
\texttt{ev1}, which then moves us to state \texttt{s2}, and so on. After five
events have been generated, they are passed to an \texttt{execute} operation which
uses the events to test that particular path through the graph.

Note that a graph searching cascade like this can generate tens of thousands of
possibilities very quickly, even with comparatively simple graphs.

\subsection{Building data in stages}

A cascade of let bindings can be used to build a complex data structure in
stages, rather than each binding having to create a complete value. For example:
\scriptsize
\begin{lstlisting}
EPATest:
    let a, b, c, d, e in set    -- Pick five branches
    {
        mk_B([1],        0),
        mk_B([-1, 2],    1),
        mk_B([1, 2, 0],  2)
    }
    in

    let branches in set         -- Between one and five of them
    {
        [a],
        [a,b],
        [a,b,c],
        [a,b,c,d],
        [a,b,c,d,e]
    }
    in
 
    let epa = mk_InputFile(     -- EPA InputFile from those B values
        mk_Header(),
        mk_PaymentSummary(getPS(branches)),
        {
            mk_Branch(
                mk_SummaryOfCharge(mk_token(mk_("MID", B)), ...),
                {
                    mk_RecordOfCharge(... mk_("TxnNum", B, i) ...)
                    | i in set inds branches(B).ROCs
                },
                {
                    mk_Adjustment(mk_token(mk_(B, i)))
                    | i in set {1, ..., branches(B).ADJs}
                }
            )
            | B in set inds branches
        },
        mk_Trailer(getTR(branches) + 1)
    )
    in

    -- Finally, transform the EPA file into the various output formats.
    (
        transformAudit(epa);
        transformC4D(epa)
    );
\end{lstlisting}
\normalsize

This example creates \texttt{InputFiles}, which contain various records that
relate to electronic payments taken from a branch of a business. The objective of the
trace is to check a large number of different input files with different numbers
of branches and various record types.

Creating such files in a single trace step would be difficult, if not
impossible. But here, we see that the generation starts with a selection of
``B'' values from a set of possibilities. Then sequences of between one and five
of these values is created. Then these ``branch'' sequences are used in nested
set comprehensions to create an \texttt{InputFile}. Lastly, every InputFile
created is processed by a couple of operations.

\subsection{Oracle functions}

In many cases, the expected result of an operation or function call is too
complicated to predict simply in a trace - assuming you have a support function
that checks the result with (say) a post-condition:

\scriptsize
\begin{lstlisting}
  assert: seq1 of nat * seq1 of nat +> bool
  assert(data, expected) == data = expected
  post RESULT = true;
\end{lstlisting}
\normalsize

Here, we assume that the ``\texttt{data}'' argument passed comes from some processing in
the specification, but where does the ``\texttt{expected}'' value come from? It could be
a literal in the trace, but this is not practical for non-trivial examples, so
it is common to create support functions that produce answers that are correct
\emph{by definition}. Such functions are called \emph{oracles}. For example:

\scriptsize
\begin{lstlisting}
Scenarios:
  let s = mk_Service(...) in
  let q in set ... in
  let mk_(first, second) in set ... in 
  let test = mk_Test(
      [s],
      keyStrokesFor(q, first, second),    -- Expected keystrokes
      basketFor(first, second))           -- Expected basket
  in
      run(test);
\end{lstlisting}
\normalsize

In this example, a \texttt{Test} record is created that includes a Service, the
expected keystrokes and the expected basket result for a retail application in
a given scenario (use case). The \texttt{run} operation uses the key strokes to
drive the specification under test and then check that the expected basket is
produced correctly. Both of these functions are oracles.

\subsection{Using positive and negative sense checks}

The natural way to think about testing a specification is to consider all of the
success paths and then design traces that exercise those paths. But in many
cases, a specification is also required to ``fail'' in the correct way; that is,
there certain inputs or call sequences that require a specific error condition
to be generated, even though that is a failure in some sense.

There is an example of this in the Luhn specification described in Chapter
\ref{chap:examples}. The Luhn algorithm is a kind of checksum. Therefore it is
\emph{required} to fail if a piece of data is corrupted in specific ways, which
shows that the checksum is doing its job, detecting the corruption. So the Luhn
specification tests deliberately corrupt a piece of data and then verify that
the algorithm generates a \emph{different} check digit -- i.e.\ they verify that a
check with the original check digit fails correctly:

\scriptsize
\begin{lstlisting}
  checkFail: seq1 of nat * nat * nat ==> bool
  checkFail(data, expected, base) ==
      return luhn(data, base) <> expected    -- Expect failure
  post RESULT = true;
\end{lstlisting}
\normalsize

\section{Avoiding Test Explosions}

It is extremely easy to write a comparatively simple looking trace definition
that expands into a collection of tests that is so large that it is not
practical to execute, either because it would take too long or because the
generation process takes up too much memory.

The following tips will help you to avoid this pitfall:

\begin{description}
  \item[Start small:] It is tempting to write traces as clearly as
  possible to start with, and that may lead to the binding of values from sets
  of data or types with many values. The combinatorial expansion process will
  then either multiply these data sizes together, or in some cases generate 
  combinations that depend on the product of the \emph{factorial} of the data
  sizes. So a modest set of 50 values might therefore generate of the order of
  $10^{64}$ tests (i.e.\ the factorial of 50). So start small: design and test 
  traces with small example sets and types, and only expand the data selections
  when you can see that the trace is expanding as you require.
  \item[Split up traces:] The alternation operator, \texttt{|}, will join
  together two sets of tests in a trace. It may therefore be tempting to write a
  single trace that is composed of many parts, testing different parts of the
  system, joined by alternation. This has the advantage that the whole
  specification can be tested by running one trace. But it also means that the
  trace expands to the sum of all of the tests within all of the parts. That is
  much better than a product or factorial combination, but if the parts are
  genuinely separate, you will be able to do more tests in the parts by
  separating them into multiple traces.
  \item[Be careful with multiple-binds and power sets:] As mentioned
  above, factorial scaling is extremely expensive. This occurs most commonly
  with multiple-binds that are used for k-permutations, and with the
  \texttt{\textbf{power}} operator on sets. As suggested above, in these case start with
  small sets and increase them with care!
\end{description}

\chapter{Combinatorial Testing Examples}
\label{chap:examples}

This Chapter looks at two significant specifications and their traces.
Extracts of the specifications are given, but the full listing of both can be
found in Appendix \ref{chap:listings}.

\section{The Luhn Check Digit Model}

\subsection{Background}

The Luhn\footnote{See https://en.wikipedia.org/wiki/Luhn\_algorithm} check digit
algorithm is commonly used in commercial systems to provide a simple means to
check the integrity of a number, such as a credit card number or a product barcode.

The simplest version of the algorithm is able to check strings of decimal
digits. A more complex version is defined for checking strings of digits in an
arbitrary numerical base and encoding. In both cases, the check digit produced
is a valid character in the numerical base of the input. The example presented
here is the more complex algorithm.

The Luhn algorithm can detect a wide range of common transcription errors, for
example when adjacent digits are swapped in the input. But its error detection
is not perfect. In particular, there are specific patterns of input corruption
that are \emph{not detected} by the algorithm. The testing of the specification
has to verify both the correct detection of most corruptions, and the the
correct failure to detect the known weaknesses.

The discussion below starts by describing the structure and traces used to test
a standard version of the Luhn algorithm. Then we look at a \emph{non-standard}
implementation from a real life example, and show how combinatorial tests not
only identify the problem, but also show that the level of error checking
provided by the non-standard implementation is weaker than the standard one.

\subsection{The Structure of the Specification}

The Luhn algorithm is defined in a singe class, called ``LUHN''.

The first part of the specification defines the encoding for the digit
strings that are to be checked. The algorithm is defined for a given
\emph{base}, and therefore a set of characters for the base must be defined. In
the simple base-10 case, the ten characters are usually '0', '1', \ldots, '9',
but they could be any selection of ten distinct characters. Similarly, other
bases require more or fewer characters, and the choice is arbitrary. A given
choice is defined by a \texttt{Mapping} type in the model:

\small
\begin{lstlisting}
public Mapping = inmap char to nat
inv m == rng m = {0, ..., card rng m - 1};
\end{lstlisting}
\normalsize

\noindent The top level Luhn algorithm is then a function which takes a
\texttt{Mapping} as well as a base and a string of characters, returning a
single check digit:

\small
\begin{lstlisting}
public luhns: seq1 of char * nat1 * Mapping -> char
luhns(string, base, mapping) ==
    let encoded = [ mapping(string(i)) | i in set inds string ] in
        (inverse mapping)(luhn(encoded, base))

pre (elems string subset dom mapping) and (card rng mapping = base)
post RESULT in set dom mapping;
\end{lstlisting}
\normalsize

The precondition checks that the string passed is only comprised of characters
in the domain of the mapping, and that the base is consistent with the size of
the mapping. The postcondition checks that the character returned is also a
member of the mapping.

The body of the function uses the mapping to encode the input string into a
sequence of numbers. These are then processed by a lower level \texttt{luhn}
function to produce a check value, which is finally mapped back to a character
using the inverse of the mapping.

The \texttt{luhn} core function is passed a sequence of numbers and a base,
returning another number of that base:

\small
\begin{lstlisting}
public luhn: seq1 of nat * nat1 -> nat
luhn(data, base) ==
    let remainder = total(data, base) mod base in
        (base - remainder) mod base
pre forall i in set inds data & data(i) < base
post RESULT < base and (total(data, base) + RESULT) mod base = 0;
\end{lstlisting}
\normalsize

The Luhn check value is defined in terms of a \texttt{total} function. The
Luhn result is difference between the total function and the base (modulo the
base). The precondition checks that all of the values in the input string are
within the base, and the postcondition checks that the result is within the base
and that adding the total value to the result gives zero (modulo the base).

Lastly, the \texttt{total} function is passed the string of input values and
produces a total, which is the sum of the digits of the value multiplied by a
factor which alternates between 1 and 2, such that the \emph{rightmost}
factor is always 2.

\subsection{Testing Approach}

The natural approach to test the algorithm is to produce a check digit for some
strings with known check digits, such as a credit or debit card
number\footnote{No, this isn't my real credit card number!}, where the Luhn
calculation is in base-10 and computed over the first 15 digits of the number:

\small
\begin{lstlisting}
> print Test`luhn10("492912341234999")
= '5'
Executed in 0.006 secs. 
\end{lstlisting}
\normalsize

This is a perfectly valid way to test, and quickly gives confidence that the
LUHN specification is roughly correct. But of course there may be many edge
cases and strange combinations of digits that cause issues. This is where
combinatorial tests give us more testing power.

The style of combinatorial testing needed for a specification like LUHN is the
``data'' style, where traces are used to generate a large number of input data
strings, which are then verified by a single function. See
\ref{chap:dataprocess}.

The first trace creates a large set of test strings to check whether the
specification produces a result that does not violate any of the constraints:

\small
\begin{lstlisting}
values
  charToCodeMap10 : LUHN`Mapping =
    let decimal = "0123456789" in
      { decimal(a) |-> (a-1) | a in set inds decimal };

traces
  FirstN:
    let a,b,c,d in set dom charToCodeMap10 in
    (
      luhn10([a])     |
      luhn10([a,b])   |
      luhn10([a,b,c]) |
      luhn10([a,b,c,d])
    );
\end{lstlisting}
\normalsize

\noindent This uses the domain of the \texttt{Mapping} for base 10 to produce
\emph{all possible} sequences of 1, 2, 3 and 4 digits, calculating the Luhn
check digit for each. Note the following:

\begin{itemize}
	\item Unlike the ad-hoc tests of various credit card examples, this trace does
	not include the correct answer for each test. Rather, it depends on the
	postcondition of the \texttt{luhn} function, which says that the \texttt{total}
	function plus the check digit must be zero (mod 10). This is generally the case
	for combinatorial tests, unless you produce an oracle function instead of a
	postcondition. In this case, such an oracle would have to reproduce the Luhn
	algorithm, and so this would add little value (i.e. the oracle could be wrong
	as well).
	\item The trace only includes strings of length 1 to 4, not 5 and beyond. Why
	not? The reason is a judgement, made by the tester. A string with a single value
	may well be a special case; similarly a string of two is the minimal example of
	a multi-character string. The Luhn algorithm treats alternate places in the
	string differently, so adding three and four values covers cases for the second
	pair. Tests of five or more values are probably adding less value, and at
	some point you simply have to stop. This trace produces 40,000 tests. If you
	have time, it would be easy to extend it to produce a million, but if the first
	40,000 work there is unlikely to be a problem in the remainder.
\end{itemize}

After the \texttt{FirstN} trace, there is a second one called
\texttt{FirstN\_16}, which does the same processing but in base 16 rather than
base 10. This naturally produces more test cases (because the alphabet is
larger), so to keep the trace size manageable, it covers strings of length 1 to
3. As before, this can be extended (with diminishing returns) if you have time.

So at this point, we have much greater confidence that the specification is
producing results that meet the internal constraints. But recall that the
correct Luhn specification is also \emph{unable} to detect certain changes in
its input. This is an unusual case where the algorithm is required to produce a
particular ``wrong'' behaviour, in a sense. So next we can design traces that
test this particular aspect.

The first trace checks that the algorithm correctly detects all single-digit
errors. This means all cases where a single digit is changed to another value:

\small
\begin{lstlisting}
  AllOneDigitErrors:
    let base = 16 in
    let input = conc [[a, a] | a in set {0, ..., base-1}] in
    let expected = LUHN`luhn(input, base) in
    let pos in set inds input in
    let replacement in set {0, ..., base-1} \ {input(pos)} in
    let corrupt = input(1, ..., pos-1) ^
                  [replacement] ^ input(pos+1, ..., len input)
    in
      checkFail(corrupt, expected, base);
\end{lstlisting}
\normalsize

\noindent This trace is mostly simple definitions. The test uses \texttt{base}
16; an \texttt{input} string is created formed from pairs of all digits from the base;
the \texttt{expected} Luhn check digit for the string is calculated. There are
duplicated pairs because the Luhn algorithm treats odd and even positions
differently, so both should be tried for every value. Then the \texttt{pos}
value considers every position in the string and \texttt{replacement} takes the
value of every other possible digit at that position. The \texttt{corrupt}
value is then the original input with the digit at \texttt{pos} replaced by
\texttt{replacement}. This corrupt string ought to produce a different Luhn
check digit to the \texttt{expected} value from the original input, which is
verified by \texttt{checkFail}.

So having verified the correct detection of all single digit errors, we have to
check for the cases that are \emph{not detected} by the algorithm. This happens
in two special cases, firstly for swapped pairs of digits in the input, where
one digit is zero and the other is base-1 (e.g. 'F' in hexadcimal or '9' in
base-10). The trace to prove this is as follows:

\small
\begin{lstlisting}
AllAdjacentTranspositions:
  let base = 16 in
  let a, b in set {0, ..., base-1} be st a <> b in
  let pass = (a = 0 and b = base-1) or (a = base-1 and b = 0) in
  let expected = LUHN`luhn([a, b], base) in
    checkResult([b, a], expected, base, pass);
\end{lstlisting}
\normalsize

\noindent This trace produces all \texttt{[a, b]} pairs for the given base,
where the two values are different. The expected Luhn check value is calculated for the pair
and then the result is checked for the swap of the two values, \texttt{[b, a]}.
The \texttt{pass} value is true for the special cases where the algorithm
explicitly should \emph{not} detect the swap - the \texttt{checkResult}
operation uses the pass argument to call \texttt{checkOK} or \texttt{checkFail}.

The second special case that is not detected by the Luhn algorithm is with the
exchange of some pairs of digits. Specifically it will not detect 22 $<->$ 55,
33 $<->$ 66 or 44 $<->$ 77. The trace to verify this is as follows:

\small
\begin{lstlisting}
AllTwinErrors:
  let base = 10 in
  let a, b in set {0, ..., base-1} be st a <> b in
  let pass = {a, b} in set {{2, 5}, {3, 6}, {4, 7}} in
  let expected = LUHN`luhn([a, a], base) in
    checkResult([b, b], expected, base, pass);
\end{lstlisting}
\normalsize

\noindent The trace generates all possible pairs of values for the base, and
then creates a pair from one value and checks that a substitution with a pair of the other is
detected as a change. Note that the \texttt{pass} value is true for the cases
that should not be detected.

The final trace to test the algorithm checks that a prefix of an arbitrary
number of zero characters does not affect the check digit value:

\small
\begin{lstlisting}
ZeroPadding:
  let base = 16 in
  let input = [x | x in set {0, ..., base-1}] in
  let expected = LUHN`luhn(input, base) in
  let number in set {1, ..., 8} in
  let padding = [p-p | p in set {1, ..., number}] in 
    checkOK(padding ^ input, expected, base);
\end{lstlisting}
\normalsize

The call to \texttt{checkOK} verifies that the check digit produced from the
original input is the same as that produced for the same with a varying prefix
of between 1 and 8 zeros.

\subsection{Real World Luhn, the ACME Example}

The sections above describe a formal specification of the standard Luhn
algorithm, and traces that can be used to verify that the specification is
correct, compared to the published algorithm.

It is always useful to have a formal specification of an algorithm, but in the
real world, implementations are often created by programmers (with the best of
intentions) on a rather less formal basis. In some cases, their implementations
are not correct, though they may still pass the simple ad-hoc levels of testing
that are usually performed.

The authors are familiar with a particular case of an invalid Luhn
implementation written by the ACME\footnote{Not their real name!} Company. ACME
is in the business of tracking the movement of a large number of parcels around
the country. Each parcel is marked with a (hexadecimal) barcode, and these have
a base-16 Luhn check digit. ACME employees use barcode scanners to check
parcels in and out of depots and delivery vans, and the software in the
scanners checks the Luhn check digit on each one. The same Luhn software is
used to create the barcodes for new parcels.

Unfortunately, the Luhn implementation used by ACME had a fault. The software
was originally written in Visual Basic for base-10 and subsequently extended to
cover base-16. But internally, a VB library to convert decimal characters to
numbers was silently ignoring hexadecimal characaters. This meant that although
the barcodes produced by the software could be checked by the scanners, the
check digits produced were not the standard check digits.

This went unnnoticed for some considerable time. But eventually the software
used to produce the barcodes had to be updated as part of a planned system
enhancement, leaving the barcode scanners unchanged. The informal specifications
of the system stated that it used the Luhn base-16 algorithm, and that was duly
re-implemented (correctly) in the upgrade.

Of course, when the new barcodes were checked with the old scanners, the check
digits \emph{sometimes} failed - not always, because the invalid algorithm was
only incorrectly handling hexadecimal characters, but it was still a serious
problem.

This caused a great deal of confusion about which implementation was correct! It
led to the creation of the formal model above and this was used to demonstrate
that the ACME implementation was at fault.

But what was the invalid algorithm? And what were its error detection
characteristics? This had to be discovered by careful reverse engineering of the
ACME Visual Basic code, and was captured in an ACME variant of the formal model,
which we present in this section.

\subsubsection{The ACME Total Function}

The error in the ACME implementation was tracked back to the \texttt{total}
function that is used to multiply and add the digits of the input. First we
show the standard algorithm for the core of the \texttt{total} function:

\small
\begin{lstlisting}
-- The standard algorithm:
let multipler = (len data) mod 2 + 1,
    code = hd data,
    product = code * multipler
in
    sumDigits(product, base)
\end{lstlisting}
\normalsize

\noindent The \texttt{sumDigits} function adds the digits of its argument in the
given base (e.g. the sum of 123 in base 10 is 1+2+3=6). Here is the equivalent
in the ACME implementation:

\small
\begin{lstlisting}
-- The ACME algorithm:
let multipler = (len data) mod 2 + 1,
    code = hd data,
    product = code * multipler
in
    if hasNonDecimals(product, base)
    then code
    else sumDigits(product, base)
\end{lstlisting}
\normalsize

\noindent Notice that the ACME algorithm has an extra test for
\texttt{hasNonDecimals}. This function is true if there are any A-F characters
in the product. If there are none, then the result of the ACME \texttt{total}
function will be exactly the same as the standard; but if there are non-decimal
characters, the function returns the \texttt{code} value - i.e. the
unmultiplied and unsummed input value. Obviously this is wrong.

Using this new specification, we could show that it seemed to produce ACME
check digits that agree with the old ACME software, both with and without
hexadecimal characters. But how could we test this specification further with
traces, if we were not sure of the error detection characteristics of the ACME
algorithm?

Clearly, without a complete analysis of the ACME algorithm we could not
\emph{predict} its error detection characteristics. But by using the traces for
the standard algorithm, we could easily \emph{discover} whether the ACME
version was detecting the same, or more, or fewer error cases. The results were
as follows:

\begin{description}
  \item[FirstN and FirstN\_16:] These traces still operate correctly under the
  ACME variant because they are testing for violation of constraints in the
  specification rather than checking individual check digit values. This tells
  us that the ACME algorithm does produce something that meets all of the pre-
  and postconditions defined.
  \item[AllOneDigitErrors:] This trace produces no errors with the standard
  algorithm, because it is capable of detecting all single digit errors.
  However, the ACME variant produces six errors. On closer examination, we find
  that it cannot dectect substitutions of 3 $<->$ 6, 5 $<->$ A, 7 $<->$ B.
  \item[AllAdjacentTranspositions:] This trace has two exceptions that are not
  detected by the standard algorithm, 0 $<->$ F and F $<->$ 0. The ACME variant
  has these two exceptions as well, but in addition it has 40 others.
  \item[AllTwinErrors and ZeroPadding:] These traces behave in the same way as
  the standard algorithm.
\end{description}

\noindent So, with the help of a formal model and combinatorial testing, we
could show the ACME company not only that their implementation was at fault, but
that the invalid implementation was significantly weaker at error detection
than the standard algorithm (and we could characterize its weaknesses).


\section{The Basket Service Model}

\subsection{Background}

The Basket Service model describes the behaviour of a simple client-server
system that manages a basket of retail items. Initially a basket is empty, and
can have multiple items added and cancelled (removed), before finally the entire
basket is settled (the customer pays for the items) or cancelled.

\subsection{The Structure of the Specification}

The Basket Service is modelled by a single class, called BasketService, which
includes the current basket in its state:

\small
\begin{lstlisting}
types
public Product = seq1 of char;

private BasketItem ::
  iid      : nat		-- Unique per basket
  product  : Product
  amount   : real;

private Basket ::
  items : [seq of BasketItem]
inv
  basket ==
    basket.items <> nil =>
      let ids = { item.iid | item in seq basket.items } in
        card ids = len basket.items;

instance variables
  basket : Basket := mk_Basket(nil);
\end{lstlisting}
\normalsize

\noindent The client-server messages are modelled by record types for each
transaction type, either adding or cancelling an item from the bsaket, or cancelling or
settling the entire basket. For example:

\small
\begin{lstlisting}
public AddItemRequest ::
  uid      : nat
  product  : Product
  amount   : real;

public CancelItemRequest ::
  uid      : nat
  iid      : nat;

public CancelBasketRequest ::
  uid      : nat;

public SettleBasketRequest ::
  uid      : nat;

public Response ::
  type     : <OK> | <FAILED> | <SEQERR>
  sid      : nat
  message  : [seq of char]
  iid      : [nat]
  total    : real;
\end{lstlisting}
\normalsize

\noindent In addition to the simple values that populate the basket, the
requests carry a \texttt{uid} (a \emph{user} message ID), and responses include
a \texttt{sid} (a \emph{server} message ID). The protocol requires that both
user IDs and server IDs are incremented for every message sent or received.
Responses also include the total amount of the basket so far (which may be
zero, if it is empty).

The processing of each message type is specified as a \emph{pair} of operations,
like \texttt{addItem} and \texttt{addItemImpl}. The first is the public
interface which checks the arguments passed and the message sequencing before
calling the second operation; the second is private, assumes the arguments are
correct, and performs the core processing of the request. The postcondition on
the first operation defines the overall behaviour \emph{in detail}; the
precondition on the second operation verifies that the arguments are valid. For
example, the \texttt{cancelItem} operations are as follows:

\scriptsize
\begin{lstlisting}
public cancelItem: CancelItemRequest ==> Response
cancelItem(request) ==
(
  if lastuid <> nil and request.uid <> lastuid + 1
  then return seqerrResponse("Invalid sequence")
  else
  (
    lastuid := request.uid;

    if basket.items = nil
    then return errorResponse("Basket is empty", nil)
    else if request.iid not in set basketItems(basket)
         then return errorResponse("No such item", request.iid)
         else return cancelItemImpl(request)
  )
)
post
  cases RESULT.type:
    <SEQERR> ->
      basket = basket~  -- sid changes, nothing else
      and lastuid = lastuid~
      and iid = iid~
      and sid = sid~ + 1,

    <FAILED> ->
      basket = basket~  -- lastuid and sid change
      and lastuid = request.uid
      and iid = iid~
      and sid = sid~ + 1,

    <OK> ->
      lastuid = request.uid
      and iid = iid~
      and sid = sid~ + 1
      and len basket.items = len basket~.items - 1
      and RESULT.total = basketTotal(basket.items)
      and RESULT.iid not in set basketItems(basket)
  end;

private cancelItemImpl: CancelItemRequest ==> Response
cancelItemImpl(request) ==
(
  basket.items := [ item | item in seq basket.items & item.iid <> request.iid ];
  return okResponse(request.iid)
)
pre lastuid = request.uid
  and basket.items <> nil
  and request.iid in set basketItems(basket);
\end{lstlisting}
\normalsize

\noindent Notice how the main public operation performs basic checks and then
delegates to the private ``Impl'' operation, which actually performs the update. The
postcondition on main operation has a detailed postcondition that considers
each possible result type and states the changes in the system state and
message counters that result. You can see that if the result is \texttt{<OK>},
then the basket is edited to remove a single item and return the new basket
total.

From a combinatorial testing perspective, this style of specification allows us
to make lots of experimental calls to the public interface operations, in the
confidence that the constraints will very precisely check that the operations
are modifying the state of the server correctly. Although note that the server
only checks the sequence of the user message IDs; it is still up to the
``caller'' to check that the server message IDs returned are in sequence.

\subsection{Testing Approach}

The obvious way to create a simple test of this model is to write a sequence of
public interface calls, and check that the return response from each is as
expected. For example:

\scriptsize
\begin{lstlisting}
  public testAdHoc: () ==> seq of bool
  testAdHoc() == let s = new BasketService() in return
  [
    s.cancelItem(mk_BasketService`CancelItemRequest(0, 99))
      = mk_BasketService`Response(<FAILED>, 1, "Basket is empty", nil, 0),
    s.cancelBasket(mk_BasketService`CancelBasketRequest(1))
      = mk_BasketService`Response(<FAILED>, 2, "Basket is empty", nil, 0),
    s.addItem(mk_BasketService`AddItemRequest(2, "apples", -99))
      = mk_BasketService`Response(<FAILED>, 3, "Amount invalid", nil, 0),
    s.addItem(mk_BasketService`AddItemRequest(3, "apples", 1.23))
      = mk_BasketService`Response(<OK>,     4, nil, 1, 1.23),
    s.addItem(mk_BasketService`AddItemRequest(99, "pears", 2.34))
      = mk_BasketService`Response(<SEQERR>, 5, "Invalid sequence", nil, 1.23),
    s.addItem(mk_BasketService`AddItemRequest(4, "pears", 2.34))
      = mk_BasketService`Response(<OK>,     6, nil, 2, 3.57),
    s.cancelItem(mk_BasketService`CancelItemRequest(5, 1))
      = mk_BasketService`Response(<OK>,     7, nil, 1, 2.34),
    s.cancelItem(mk_BasketService`CancelItemRequest(6, 99))
      = mk_BasketService`Response(<FAILED>, 8, "No such item", 99, 2.34),
    s.settleBasket(mk_BasketService`SettleBasketRequest(7))
      = mk_BasketService`Response(<OK>,     9, nil, nil, 2.34)
  ]
  post elems RESULT = {true};
\end{lstlisting}
\normalsize

\noindent This test operation returns a sequence of booleans, each one made from
an assertion that a call to a particular public operation will return a particular
Response message. The postcondition effectively asserts that the whole test
passes.

As discussed above, this type of ad-hoc testing is perfectly valid, but there
are very many paths through the server behaviour that are not exercised by such
tests. Here, combinatorial traces can give us a great deal more testing power.
The Basket Service model lends itself to the \emph{process} style of
combinatorial testing, discussed in \ref{chap:dataprocess}. But we also have to
consider creating supporting test operations that maintain the ``client'' state
of the system in order to make sensible sequences of interface calls on the
server with a trace. See \ref{chap:testops}.

To create test operations, we need to be able to track the server message ID (to
check that returned values are continuous) as well as creating a contiguous
client message ID for new requests. We also need to be able to track the item
IDs returned when new items are added to the basket, so that the cancelItem
operation can select items that are in the basket. Lastly, it would be useful to
be able to create errors in the IDs, so that we can check this behaviour as
well. This leads to a \texttt{Tests} class with state and operations as
follows:

\scriptsize
\begin{lstlisting}
class Tests

-- State for the combinatorial tests.
instance variables
  SID : nat              := 0;
  UID : nat              := 0;
  iids : set of nat      := {};
  server : BasketService := new BasketService();

operations
  private addItemTest: bool * BasketService`Product * real ==> BasketService`Response
  addItemTest(ok, product, amount) ==
  (
    if ok then  UID := UID + 1;

    let uid = if ok then UID else 0,
        response = server.addItem(mk_BasketService`AddItemRequest(uid, product, amount)) in
    (
      if response.type = <OK> then iids := iids union {response.iid};
      checkSID(response.sid);
      return response
    )
  );

  private checkSID: nat ==> ()
  checkSID(s) == SID := s
  pre s = SID + 1;

\end{lstlisting}
\normalsize

\noindent So the \texttt{addItemTest} operation (conditionally) increments the
client UID value and uses this to build a Request. If the result is
\texttt{<OK>}, the IID returned is added to the set. The server SID is checked
and updated, and the response is returned so that the tests from the trace show
the result details.

Similar test operations are created for the other server interface operations,
which similarly use the state data to prepare their requests. For example, the
\texttt{cancelItemTest} operation includes a line to select an arbitrary IID to
cancel, or return nil if there are no items in the basket (i.e. skip the
operation):

\scriptsize
\begin{lstlisting}
  if iids <> {} then
    let iid in set iids in
      ... -- cancel item iid
  else return nil
\end{lstlisting}
\normalsize

Given that all of the server interface operations check their actions in detail,
it is tempting to create a combinatorial test that just tries all of the
operations in every possible order, including with a \emph{false} argument to
cause error cases:

\small
\begin{lstlisting}
traces
  TryEverything: ||
  (
    addItemTest(true, "apples", 1.23),
    addItemTest(false, "pears", 4.56),
    cancelItemTest(true),
    cancelItemTest(false),
    cancelBasketTest(true),
    cancelBasketTest(false),
    settleBasketTest(true),
    settleBasketTest(false)
  );
\end{lstlisting}
\normalsize

\noindent This approach does have some value. It produces 8! tests (40,320),
like this, selecting test number 1234 to illustrate:

\scriptsize
\lstset{style=tool,language=}
\begin{lstlisting}
> runtrace TryEverything 1234
Generated 40320 tests in 0.001 secs. 
Test 1234 = addItemTest(true, "apples", 1.23);
            cancelItemTest(true);
            settleBasketTest(true);
            cancelItemTest(false);
            cancelBasketTest(true);
            cancelBasketTest(false);
            settleBasketTest(false);
            addItemTest(false, "pears", 4.56)
Result = [mk_Response(<OK>, 1, nil, 1, 1.23),
          mk_Response(<OK>, 2, nil, 1, 0),
          mk_Response(<OK>, 3, nil, nil, 0),
          nil,
          mk_Response(<FAILED>, 4, "Basket is empty", nil, 0),
          mk_Response(<SEQERR>, 5, "Invalid sequence", nil, 0),
          mk_Response(<SEQERR>, 6, "Invalid sequence", nil, 0),
          mk_Response(<SEQERR>, 7, "Invalid sequence", nil, 0), PASSED]
Excluded 40319 tests
Executed in 0.024 secs. 
All tests passed
\end{lstlisting}
\normalsize

\noindent Although this permutation approach might show up some corner cases,
most of the permutations do not represent a realistic call sequence of a
client (like this example!). At most a permutation will add one item to the
basket and only settle one basket. To be more realistic, we need more basket
activity before it is either settled or cancelled. That could be achieved by
adding more \texttt{addItemTest} or \texttt{settleBasketTest} calls to the
trace, but even if we only add two, that takes the total number of permutation
up to 10! (3,628,800), which will take a very long time to execute.

A more considered approach yields a larger number of realistic test sequences.
Here, we build a trace using a typical sequence of actions from a client, some
of which are alternated or optional (to cover varying possibilities). At the
end, the whole sequence is duplicated (\texttt{\{2\}}) to produce tests that
complete one basket and then another:

\small
\begin{lstlisting}
TwoBaskets:
  (
    addItemTest(true, "apples", 1.23) |
    addItemTest(false, "apples", 1.23);
    ( cancelItemTest(true) |
      cancelItemTest(false) )?;

    addItemTest(true, "pears", 2.34) |
    addItemTest(false, "pears", 2.34);
    ( cancelItemTest(true) |
      cancelItemTest(false) )?;

    ( cancelBasketTest(false) |
      settleBasketTest(false) )?;

    cancelBasketTest(true) |
    settleBasketTest(true)
  ) {2}
\end{lstlisting}
\normalsize

\noindent This trace means that we always start by adding, or failing to add an
item; then we either cancel or fail to cancel it or we skip this step; then we add or fail
to add a second item; then we cancel or fail to cancel or we skip; then we fail
to cancel the basket or fail to settle the basket; and lastly we successfully
cancel or settle the basket; and overall we repeat all \emph{combinations} of these
sequences twice.

This produces far more realistic test sequences and expands to 46,656 tests.
Here is one selected for illustration:

\scriptsize
\lstset{style=tool,language=}
\begin{lstlisting}
> runtrace TwoBaskets 4321
Generated 46656 tests in 0.001 secs. 
Test 4321 = addItemTest(true, "apples", 1.23);
            skip;
            addItemTest(true, "pears", 2.34);
            skip;
            skip;
            cancelBasketTest(true);
            addItemTest(true, "apples", 1.23);
            cancelItemTest(true);
            addItemTest(false, "pears", 2.34);
            cancelItemTest(true);
            skip;
            cancelBasketTest(true)
Result = [mk_Response(<OK>, 1, nil, 1, 1.23),
          (),
          mk_Response(<OK>, 2, nil, 2, 3.57),
          (),
          (),
          mk_Response(<OK>, 3, nil, nil, 0),
          mk_Response(<OK>, 4, nil, 1, 1.23),
          mk_Response(<OK>, 5, nil, 1, 0),
          mk_Response(<SEQERR>, 6, "Invalid sequence", nil, 0),
          nil,
          (),
          mk_Response(<OK>, 7, nil, nil, 0), PASSED]
Excluded 46655 tests Executed in 1.486 secs.
All tests passed
\end{lstlisting}
\normalsize

\noindent You can see that this is a more rational client call sequence (though
only just!). So by composing a trace that is focussed on more realistic behaviour, we
have managed to produce a comparable number of tests to the full permutation
approach (of the order of 40K), but each one should be adding more value
than a random permutation of interface calls. Of course, both traces can be
used in the full test suite.

\appendix
\chapter{Combinatorial Testing Syntax}
\label{chap:CTsyntax}

\Ruledef{traces definitions}{\Lop{traces},
   \OptPt{\Ruleref{named trace}, \SeqPt{\Lit{;}, \Ruleref{named trace}}}
}

\Ruledef{named trace}{\Ruleref{identifier},
  \SeqPt{\Lit{/}, \Ruleref{identifier}},
  \Lit{:},\Ruleref{trace definition list}
}

\Ruledef{trace definition list}{
  \Ruleref{trace definition term},
  \SeqPt{\Lit{;}, \Ruleref{trace definition term}}
}

\Ruledef{trace definition term}{
  \Ruleref{trace definition}, \SeqPt{\Lit{|}, \Ruleref{trace definition}}
}

\Rule{trace definition}{
  \Ruleref{trace binding definition} \dsep
  \Ruleref{trace repeat definition}
}

\Ruledef{trace binding definition}{
  \Ruleref{trace let def binding} \dsep
  \Ruleref{trace let best binding}
}

\Ruledef{trace let def binding}{
  \Lop{let}, \Ruleref{local definition},
  \SeqPt{ \Lit{,}, \Ruleref{local definition}}, \lfeed
  \Lop{in}, \Ruleref{trace definition}
}

\Ruledef{trace let best binding}{
  \Lop{let}, \Ruleref{multiple bind},
  \OptPt{\Lop{be}, \Lop{st}, \Ruleref{expression}}, \lfeed
  \Lop{in}, \Ruleref{trace definition}
}

\Ruledef{trace repeat definition}{
  \Ruleref{trace core definition},
  \OptPt{\Ruleref{trace repeat pattern}}
}

\Ruledef{trace repeat pattern}{
  \Lit{*} \dsep
  \Lit{+} \dsep
  \Lit{?} \dsep
  \Lit{\{}, \Ruleref{numeric literal}, \OptPt{\Lit{,}, \Ruleref{numeric literal}, \Lit{\}}}
}

\Rule{trace core definition}{
  \Ruleref{trace apply expression} \dsep
  \Ruleref{trace concurrent expression} \dsep
  \Ruleref{trace bracketed expression}
}

\Rule{trace apply expression}{
  \Ruleref{call statement}
}

\Rule{trace concurrent expression}{
  \Lit{||}, \Lit{(}, \Ruleref{trace definition}, \lfeed
  \Lit{,}, \Ruleref{trace definition}, \lfeed
  \SeqPt{\Lit{,}, \Ruleref{trace definition}}, \Lit{)}
}

\Ruledef{trace bracketed expression}{
  \Lit{(}, \Ruleref{trace definition list}, \Lit{)}
}

\chapter{Overture Screenshots}
\label{chap:OvertureShots}

\chapter{Example Model Listings}
\label{chap:listings}
The VDM models listed in this appendix are described in detail in Chapter
\ref{chap:examples}

\section{The LUHN Check Digit Model}
\section{The Basket Service Model}

\bibliographystyle{newalpha}

\bibliography{../bib/dan}

\end{document}


