\section{Translating \ac{vdmsl} contracts to \ac{jml}}
\label{sec:sl-dbc-gen}

In this section we present the rules used to translate the \ac{dbc}
elements of \ac{vdmsl} to \ac{jml} annotations that are added to the
generated Java code. For each of the elements, we describe the
approach used to translate the element to \ac{jml}. This is afterwards
generalised as a rule, which appears in a grey box.

\subsection{Allowing null values by default}
\label{sec:modules}

Overture's Java code-generator may sometimes introduce auxiliary
variables that are initialised to \kw{null} when it code-generates
some of the constructs of \ac{vdm}. To avoid having errors reported
when checking the generated code with a \ac{jml} tool, we allow
\kw{null} as a legal value by default for all references in the
generated code.

\transRule{Allowing null values by default} {Annotate every class output by the
  Java code-generator with the \kw{nullable\_by\_default} modifier to
  allow all references to use \kw{null} as a legal value.}
%
\newcounter{nullRule}
\setcounter{nullRule}{\arabic{transRule}}
%

As a consequence we also have to guard against \kw{null} values for
variables that originate from \ac{vdm} variables or
patterns\footnote{The generated code uses variables to represent the
  patterns (record pattern, tuple pattern, identifier pattern etc.)
  introduced by use of pattern matching in \ac{vdm}.}  that do not
allow \kw{nil}.

\subsection{Translating functional descriptions to \ac{jml}}
\label{sec:func-desc}

Recall that a \ac{vdmsl} function code-generates to a \kw{static} Java
method. In addition, a \ac{vdmsl} function does not have side-effects
and therefore the code-generated version of the method can be
annotated as \ac{jml} \kw{pure}.

\transRule{Translation of functions}{Any function -- whether it is
  defined by the user or derived, e.g.\ from a \kw{pre} or \kw{post}
  condition clause -- code-generates to a \kw{static} Java method that
  is annotated with the \kw{pure} modifier.}
%
\newcounter{funcRule}
\setcounter{funcRule}{\arabic{transRule}}
%

Operations, on the other hand, can read and manipulate the state of
the enclosing module, or invoke other operations that may have
side-effects. Therefore, the method that the operation code-generates
to cannot be annotated as \ac{jml} \kw{pure}.

When a \ac{vdmsl} definition (e.g.\ a functional description) is
code-generated to Java, the visibility of the corresponding Java
definition can, in principle, be set according to whether the
\ac{vdmsl} definition is exported (\kw{public}) or not
(\kw{private}). In the presentation of the translation rules following
this section, we omit explicit use of access specifiers in the rule
formulation as we do not consider it crucial to our work.

\subsection{Translating preconditions to \ac{jml}}

In terms of semantics there is no difference between a precondition in
\ac{vdmsl} and \ac{jml}. There are, however, interesting issues worth
mentioning regarding how the \ac{jml} translator implements the
translation. We start by covering preconditions of operations, and we
end this subsection by describing how they differ from those of
functions. As an example of how a \ac{vdmsl} precondition is
translated, consider the operation
in~\autoref{lst:vdmWithdrawPre}. This operation models withdrawal from
a bank account identified by the parameter \texttt{id}.

\begin{vdmsl}[style=customVdm,caption={\ac{vdmsl} operation for bank
account withdrawal guarded by a precondition.},label={lst:vdmWithdrawPre}]
Withdraw : AccountId * Amount ==> real
Withdraw (id, amount) ==
let newBalance =
    accounts(id).balance - amount
in (
 accounts(id).balance := newBalance;
 return newBalance;
)
pre
currentCard in set validCards and pinOk and
currentCard in set accounts(id).cards and
id in set dom accounts
\end{vdmsl}

In order to withdraw money from the account, we require that a valid
card has been inserted, the PIN code is accepted, and that the bank
account exists. Note that since \texttt{currentCard} is of the
optional type \texttt{[Card]} it can be \kw{nil}, which is not a valid
member of \texttt{validCards}. Therefore, the precondition is
\kw{false} when no debit card has been inserted into the \ac{atm}. The
\texttt{pre\_Withdraw} function, which is not a visible part of the
model, is derived from the \kw{pre} clause of the \texttt{Withdraw}
operation. In the generated code this function is represented using a
\kw{pure} method according to rule \arabic{funcRule} -- see
\autoref{lst:jmlPreWithdraw}. Note that for the method in
\autoref{lst:jmlPreWithdraw}, the Java code-generator uses extra
variables to perform the equivalent \ac{vdm} computation. These extra
variables are also type checked using \ac{jml} (although they are only
used to store intermediate results).

\begin{lstlisting}[float=*,style=customJml,caption={Code-generated version of
the \texttt{pre\_Withdraw} operation.},label={lst:jmlPreWithdraw}]
/*@ pure @*/
public static Boolean pre_Withdraw(
    final Number id, final Number amount, final atm.ATMtypes.St St) {
  //@ assert (Utils.is_nat(id) && inv_ATM_AccountId(id));
  //@ assert (Utils.is_nat1(amount) && inv_ATM_Amount(amount));
  //@ assert Utils.is_(St,atm.ATMtypes.St.class);
  Boolean andResult_3 = false;
  //@ assert Utils.is_bool(andResult_3);
  if (SetUtil.inSet(St.get_currentCard(), St.get_validCards())) {
    Boolean andResult_4 = false;
    //@ assert Utils.is_bool(andResult_4);
    if (St.get_pinOk()) {
      Boolean andResult_5 = false;
      //@ assert Utils.is_bool(andResult_5);
      if (SetUtil.inSet(St.get_currentCard(),
          ((atm.ATMtypes.Account) Utils.get(St.accounts, id)).get_cards())) {
        if (SetUtil.inSet(id, MapUtil.dom(St.get_accounts()))) {
          andResult_5 = true;
          //@ assert Utils.is_bool(andResult_5);
        }
      }
      if (andResult_5) {
        andResult_4 = true;
        //@ assert Utils.is_bool(andResult_4);
      }
    }
    if (andResult_4) {
      andResult_3 = true;
      //@ assert Utils.is_bool(andResult_3);
    }
  }
  Boolean ret_29 = andResult_3;
  //@ assert Utils.is_bool(ret_29);
  return ret_29;
}
\end{lstlisting}

The \texttt{Withdraw} operation is translated to the method shown
in~\autoref{lst:jmlWithdraw}. This method introduces several \ac{jml}
assertions that will be described in \autoref{sec:sl-types-gen}. Note
that the \texttt{pre\_Withdraw} method is invoked from the
\kw{requires} clause of the \texttt{Withdraw} method to check whether
the precondition is met. In addition to the input parameters of the
\texttt{Withdraw} method, the \texttt{pre\_Withdraw} method is also
passed the state \texttt{St}.

\begin{lstlisting}[float=*,style=customJml,caption={Code-generated version of
the \texttt{Withdraw} operation.},label={lst:jmlWithdraw}]
//@ requires pre_Withdraw(id,amount,St);
public static Number Withdraw(final Number id, final Number amount) {
 //@ assert (Utils.is_nat(id) && inv_ATM_AccountId(id));
 //@ assert (Utils.is_nat1(amount) && inv_ATM_Amount(amount));
 final Number newBalance =
     ((atm.ATMtypes.Account) Utils.get(St.accounts, id)).get_balance().doubleValue()
         - amount.longValue();
 //@ assert Utils.is_real(newBalance);
 {
   VDMMap stateDes_1 = St.get_accounts();
   atm.ATMtypes.Account stateDes_2 = ((atm.ATMtypes.Account) Utils.get(stateDes_1, id));
   //@ assert stateDes_2 != null;
   stateDes_2.set_balance(newBalance);
   //@ assert (V2J.isMap(stateDes_1) && (\forall int i; 0 <= i && i < V2J.size(stateDes_1); (Utils.is_nat(V2J.getDom(stateDes_1,i)) && inv_ATM_AccountId(V2J.getDom(stateDes_1,i))) && Utils.is_(V2J.getRng(stateDes_1,i),atm.ATMtypes.Account.class)));
   //@ assert Utils.is_(St,atm.ATMtypes.St.class);
   //@ assert \invariant_for(St);
   Number ret_7 = newBalance;
   //@ assert Utils.is_real(ret_7);
   return ret_7;
 }
}
\end{lstlisting}

\transRule{Translating the precondition of an operation} {
  Let \texttt{op} be a method code-generated from a \ac{vdmsl} user-defined operation and let the signature of \texttt{op} be:\\
  \kw{static} \texttt{R op(I\sub{1} i\sub{1},...,I\sub{n} i\sub{n})}\\
  Then \texttt{op} has a code-generated precondition method\\
  \texttt{pre\_op} that is \kw{pure} and which in addition to the
  pa\-ra\-met\-ers of \texttt{op} also takes the state component \texttt{s} as an argument, i.e.\ \\
  \texttt{/*@ \kw{pure} @*/} \kw{static} \texttt{ \kw{boolean}\\
    pre\_op(I\sub{1} i\sub{1},...,I\sub{n} i\sub{n},S s)}\\
  To ensure that the precondition is evaluated, we annotate \texttt{op} with the following \kw{requires} annotation:\\
  \texttt{//@ \kw{requires} pre\_op(i\sub{1},...,i\sub{n},s);}}
%
\newcounter{preOp}
\setcounter{preOp}{\arabic{transRule}}
%
Rule \arabic{preOp} assumes the existence of a state component
\texttt{s}. However, when the state of the module enclosing
\texttt{op} is not defined, rule \arabic{preOp} changes to not include
the state parameter in the definition of \texttt{pre\_op}.

The example above considers the case where the precondition is
guarding an operation (i.e.\ \texttt{Withdraw}). As described in
\autoref{sec:vdmjml}, a precondition is defined differently for a
function than it is for an operation. In particular, the precondition
of a function is not passed the state, so neither is the
code-generated version of it. We also note that the visibility of the
precondition function must be the same as that of the functional
description it guards. Otherwise it cannot be invoked from the
corresponding \kw{requires} clause.

\transRule{Translating the precondition of a function} {
  Let \texttt{f} be a method code-generated from a \ac{vdmsl} user-defined function and let the signature of \texttt{f} be:\\
  \kw{static} \texttt{R f(I\sub{1} i\sub{1},...,I\sub{n} i\sub{n})}\\
  Then \texttt{f} has a code-generated precondition method
  \texttt{pre\_f} that is \kw{pure} and which accepts the same parameters as \texttt{f}, i.e.\ \\
  \texttt{/*@ \kw{pure} @*/} \kw{static} \texttt{ \kw{boolean}\\
    pre\_f(I\sub{1} i\sub{1},...,I\sub{n} i\sub{n})}\\
  To ensure that the precondition is evaluated, we annotate \texttt{f} with the following \kw{requires} annotation:\\
  \texttt{//@ \kw{requires} pre\_f(i\sub{1},...,i\sub{n});}}
  
\subsection{Translating postconditions to \ac{jml}}
\label{sec:postcond}

Postconditions in \ac{vdmsl} and \ac{jml} are semantically similar,
although \ac{vdmsl} represents the postcondition function as a derived
function definition (as was done for preconditions). Furthermore, in
\ac{vdm} and \ac{jml} postconditions of operations and methods,
respectively, can access both the before and after states. Returning
to the \texttt{Withdraw} operation, one could specify a postcondition
requiring that exactly the value specified by the \texttt{amount}
parameter is withdrawn from the account -- see
~\autoref{lst:vdmWithdrawPost}.

\begin{vdmsl}[style=customVdm,caption={The \texttt{Withdraw} operation
guarded by a postcondition.},label={lst:vdmWithdrawPost}]
Withdraw : AccountId * Amount ==> real
Withdraw (id, amount) == ...
post
let accountPre = accounts~(id),
    accountPost = accounts(id)
in
 accountPre.balance =
 accountPost.balance + amount and
 accountPost.balance = RESULT;
\end{vdmsl}

The JML translator produces a \kw{pure} Java method to represent the
postcondition function. This Java method is invoked from the
\kw{ensures} clause to check that the postcondition holds. The
invocation of the postcondition method of the \texttt{Withdraw}
operation is shown in~\autoref{lst:jmlWithdrawPost}.

\begin{lstlisting}[style=customJml,caption={Code-generated version of
the \texttt{Withdraw} operation.},label={lst:jmlWithdrawPost}]
//@ requires pre_Withdraw(id,amount,St);
//@ ensures post_Withdraw(id,amount,\result,\old(St.copy()),St);
public static Number Withdraw(final Number id, final Number amount) {...}
\end{lstlisting}

Note in particular how the before and after states are passed to the
\texttt{post\_Withdraw} method. Reasoning about before state is
achieved using \ac{jml}'s \old expression. For the \texttt{Withdraw}
operation the before state is constructed as \oldexp{St.copy()}. Since
\texttt{St.copy()} is a deep copy of the state (as explained
in~\autoref{sec:sl2java}) the evaluation inside the \old expression
ensures that the result indeed is a representation of the before
state.

The \ac{jml} translator deep copies the state because Java represents
every composite data type using a class. So without deep copying the
state, only the address of the before state object reference is
copied. In effect, only a single object would exist to represent the
pre- and post states. This would never work, since state changes made
by the operation would affect what was intended to be a representation
of the before state. Therefore, the state is deep copied to get a
separate object to represent the before state.

\transRule{Translating the postcondition of an operation} {
  Let \texttt{op} be a method code-generated from a \ac{vdmsl} user-defined operation and let the signature of \texttt{op} be:\\
  \kw{static} \texttt{R op(I\sub{1} i\sub{1},...,I\sub{n} i\sub{n})}\\
  Then \texttt{op} has a code-generated postcondition method\\
  \texttt{post\_op} that is \kw{pure} and which in addition to the
  parameters of \texttt{op}, also takes the result and the before and after states of \texttt{op} as arguments, i.e.\ \\
  \texttt{/*@ \kw{pure} @*/} \kw{static} \texttt{ \kw{boolean}}\\
  \texttt{ post\_op(I\sub{1} i\sub{1},...,I\sub{n} i\sub{n},}\\
  \indent \texttt{R RESULT,S \_s,S s)}\\
  To ensure that the postcondition is evaluated we annotate \texttt{op} with the following \kw{ensures} annotation:\\
  \texttt{//@ \kw{ensures}
    post\_op(i\sub{1},...,i\sub{n},\textbackslash\kw{result},}\\
  \indent \texttt{\textbackslash\textbf{old}(s.copy()),s);}}
%
\newcounter{postOp}
\setcounter{postOp}{\arabic{transRule}}
%

While the primary concern is to preserve the behaviour of the
specification across the translation, deep copying values may
significantly affect system performance in a negative way. In
particular, because it is difficult (in general) to avoid deep copying
values unnecessarily when they are passed around in the generated
code. To address this issue, the Java code-generator (and hence the
\ac{jml} translator) offers an option that, when selected by the user,
omits deep copying of values (other than the old state). While the
purpose of this is to generate performance-efficient code, this option
is, however, only safe to use if Java objects that originate from
\ac{vdmsl} values are not modified via aliases.

Yi et al.\ identify and address a number of problems with the \old
expression~\cite{Yi&2013}. In particular, the authors of that work
conduct experiments showing that passing deep copies of the old state
may drastically increase a system's memory usage. To address this, Yi
et al.\ propose the \past expression as a more memory-efficient
alternative to the \old expression. Yi et al.\ further show that the
\past expression can be implemented as an extension of the OpenJML
runtime assertion checker by means of aspect-oriented programming
principles. However, OpenJML does not officially support the \past
expression yet, which is why the translation rules do not currently
rely on this expression. As we see it, the ideas proposed by Yi et
al.\ could potentially support the development of a more
performance-efficient way to handle old states in the \ac{jml}
translator.

Similar to rule \arabic{preOp}, rule \arabic{postOp} also assumes that
the state of the module enclosing \texttt{op} exists. If the state
component does not exist, rule \arabic{postOp} changes to not include
the state parameters in the definition of
\texttt{post\_op}. Furthermore, if \texttt{op} does not return a
result (the return type is void), then the definition of
\texttt{post\_op} does not include the \texttt{RESULT} parameter.

The example above considers the postcondition of an operation (i.e.\
\texttt{Withdraw}). As described in \autoref{sec:vdmjml}, the
postcondition of a function is not allowed to access state. Therefore, the
code-generated version of the postcondition function is not
passed the state.

\transRule{Translating the postcondition of a function} {
  Let \texttt{f} be a method code-generated from a \ac{vdmsl} user-defined function and let the signature of \texttt{f} be:\\
  \kw{static} \texttt{R f(I\sub{1} i\sub{1},...,I\sub{n} i\sub{n})}\\
  Then \texttt{f} has a code-generated postcondition method\\
  \texttt{post\_f} that is \kw{pure} and which in addition to the
  parameters of \texttt{f} also takes the result of \texttt{f} as an argument, i.e.\ \\
  \texttt{/*@ \kw{pure} @*/} \kw{static} \texttt{ \kw{boolean}\\
    post\_f(I\sub{1} i\sub{1},...,I\sub{n} i\sub{n},R RESULT)}\\
  To ensure that the postcondition is evaluated we annotate \texttt{f} with the following \kw{ensures} annotation:\\
  \texttt{//@ \kw{ensures}
    post\_f(i\sub{1},...,i\sub{n},\textbackslash\kw{result});}}

\subsection{Translating record invariants to \ac{jml}}
\label{sub:rec-inv}

A record can, like any other type definition in \ac{vdmsl}, be
constrained by an invariant. As an example,~\autoref{lst:vdmRec} shows
a record definition modelling a bank account.

\begin{vdmsl}[style=customVdm,caption={A \ac{vdmsl} record definition
modelling a bank account.},label={lst:vdmRec}]
Account :: 
 cards : set of Card
 balance : real
 inv a == a.balance >= -1000;
\end{vdmsl}

An \texttt{Account} comprises the available balance as well as the
debit cards associated with the account. We further constrain an
\texttt{Account} to not have a balance of less than -1000, which is
expressed using an invariant.

As described in~\autoref{sec:sl2java}, a record definition is
translated to a class that emulates the behaviour of a value type
using \texttt{copy} and \texttt{equals} methods.

Since a record invariant is required to hold for every record value,
or object instance in the generated code, we represent it using an
\kw{instance invariant} in \ac{jml} as shown
in~\autoref{lst:jmlRec}. Note that the \kw{instance} \kw{invariant} is
formulated as an implication such that invariant violations are not
reported when invariant checks are disabled. As we shall see in
\autoref{sec:atomic} this has to do with the way \ac{vdmsl} handles
atomic execution.

% PJ: Make listing a float to avoid unintentional page-break

\begin{lstlisting}[float,style=customJml,caption={Code-generated version
of the \texttt{Account} record.},label={lst:jmlRec}]
//@ nullable_by_default
final public class Account implements Record
{
 public VDMSet cards;
 public Number balance;
 //@ public instance invariant atm.ATM.invChecksOn ==> inv_Account(cards,balance);
  ...
 /*@ pure @*/
 public boolean equals(final Object obj)
 {...}
 /*@ pure @*/
 public atm.ATMtypes.Account copy(){...}
 /*@ pure @*/
 public VDMSet get_cards() {...}
 public void set_cards(final VDMSet _cards)
 {...}
 /*@ pure @*/
 public Number get_balance() {...}
 public void set_balance(final Number _balance) {...}
 /*@ pure @*/
 /*@ helper @*/
 public static Boolean inv_Account(final VDMSet _cards, final Number _balance){
   return _balance.doubleValue() >= -1000L;
 }
}
\end{lstlisting}

The code-generated record \texttt{Account} defines an \emph{invariant
  method} \texttt{inv\_Account} that takes all the record fields of
\texttt{Account} as input and evaluates the invariant predicate. This
method is invoked directly from the \ac{jml} invariant, as shown in
\autoref{lst:jmlRec}. Note that \texttt{inv\_Account} is a \kw{static}
method according to rule \arabic{funcRule}. In addition, this
method is annotated as a \kw{helper} to avoid the invariant check
triggering another invariant check, which eventually would cause a
stack-overflow.

\transRule{Translating a record invariant}{Let \texttt{D} be a
  code-generated record definition with fields
  \texttt{f\sub{1},...,f\sub{n}} of types
  \texttt{F\sub{1},...,F\sub{n}}, respectively, and let \texttt{D} be
  constrained by an invariant. Then \texttt{D} has an invariant method
  \texttt{inv\_D} that is annotated as a \kw{helper} to allow it to
  be invoked from the invariant clause of \texttt{D}. The invariant
  method can also be annotated as \kw{pure} since it originates from a
  function definition. The annotated signature of \texttt{inv\_D} thus
  becomes:\\
  \texttt{/*@ \kw{pure} @*/}\\
  \texttt{/*@ \kw{helper} @*/}\\
  \kw{boolean} \texttt{inv\_D(F\sub{1} f\sub{1},...,F\sub{n} f\sub{n})}\\
  Let further \texttt{invChecksOn} be a variable that is true if
  invariant checking is enabled and false otherwise. To represent the
  record invariant of \texttt{D} we annotate \texttt{D} with the
  \kw{invariant} annotation:\\
  \texttt{/*@ \kw{public} \kw{instance} \kw{invariant}\\invChecksOn
    \kw{==>} inv\_D(f\sub{1},...,f\sub{n}); @*/}}

As we shall later see in \autoref{sec:atomic}, atomic execution
sometimes requires extra assertions to be inserted into the generated
code in order to guarantee that the record invariant semantics of
\ac{vdmsl} are preserved.

All the methods inside a record class -- except for the constructor
and the ``setter'' methods -- do not modify the state of the record
class and therefore they are marked as \kw{pure}. Updates to a record
object in the generated code are made using the ``setter'' methods of
the generated record class, or by using the record modification
expression~\cite{Larsen&10b}. Use of ``setter'' methods instead of
direct field access to manipulate the state of a record (which is how
field access is achieved in \ac{vdmsl}) forces the record object into
a \emph{visible state} (as described in \autoref{sec:jml}) after it
has been updated, thus triggering the invariant check according to
the \ac{vdmsl} semantics. For example, in \ac{vdmsl} we could set
the balance of an account as shown in \autoref{lst:vdmAccountUpdate}.

\begin{vdmsl}[style=customVdm,caption={Updating the \texttt{Account}
balance in \ac{vdmsl}.},label={lst:vdmAccountUpdate},numbers=none]
acc.balance := newBalance;
\end{vdmsl}

\noindent This assignment produces the Java code shown in
\autoref{lst:jmlAccountUpdate}. Note that for this particular case
there is no need to generate any additional \ac{jml} assertions since
the state of \texttt{acc} becomes visible after the call to
\texttt{set\_balance}. This causes the invariant check of
\texttt{Account} to trigger.

\begin{lstlisting}[style=customJml,caption={Updating the
\texttt{Account} balance in the generated
code.},label={lst:jmlAccountUpdate},numbers=none]
acc.set_balance(newBalance);
\end{lstlisting}

% In \ac{vdmsl} an invariant is a function, which means that it cannot
% call operations -- only calls to functions are
% allowed.\footnote{Recent changes to \ac{vdm}10 now allows \kw{pure}
%   operations (different from the \kw{pure} concept in \ac{jml}) to be
%   called from functions.}. Java, on the other hand, uses methods to
% express functionality, but it does not distinguish between functions
% and operations as \ac{vdm} does. Therefore every functional
% description in a \ac{vdmsl} model generates to a Java method.

%\todo[inline]{Re-consider this paragraph}

% The record invariant method (e.g.\ \texttt{inv\_Account}) is defined
% within the record class (e.g.\ \texttt{Account}) and this class is
% generated as a top-level class by the current version of the
% \ac{jml} translator.\footnote{Alternatively, record classes could
% have been generated as inner classes of their defining
% module. However, since OpenJML currently provides limited support
% for loading of inner classes we have chosen to generate records as
% top-level classes. See:
% \url{https://sourceforge.net/p/jmlspecs/bugs/413/}} A user-defined
% functional description code-generates to a method in module
% class. Therefore this method cannot be accessed by a \kw{private}
% helper in a record class. Therefore we simply prohibit invocation of
% user-defined functions from a \ac{jml} invariant.

\subsection{Atomic execution}
\label{sec:atomic}

There are situations where multiple assignment statements in
\ac{vdmsl} need to be evaluated atomically in order to avoid
unintentional violation of a state invariant. In our example, this is
the case when the \ac{atm} returns the card to the owner, which is
done as the last step of a transaction. Returning the debit card also
requires us to invalidate the PIN code currently entered. These two
things have to be done atomically to avoid violating the state
invariant of the \texttt{ATM} module, which is checked using the
\texttt{inv\_St} function, derived from the state invariant shown in
\autoref{lst:vdmCase} in \autoref{sec:case}. Therefore the body of the
\texttt{ReturnCard} operation is executed inside an \kw{atomic}
statement block as shown in \autoref{lst:vdmReturn}. Note that the
invariant is evaluated internally by the interpreter, and therefore
the example in \autoref{lst:vdmReturn} makes no explicit mention of
the invariant.

\begin{vdmsl}[style=customVdm,caption={Removal of the debit card from
the \ac{atm} in \ac{vdmsl}.},label={lst:vdmReturn}]
ReturnCard : () ==> ()
ReturnCard () ==
atomic (
 currentCard := nil;
 pinOk := false;
)
pre currentCard <> nil
post currentCard = nil and not pinOk;
\end{vdmsl}

\ac{jml} does not include a syntactic construct similar to that of the
\kw{atomic} statement. Instead atomic execution must be achieved using
different means -- for example by manipulating state directly using
field access or \kw{helper} methods.

To be consistent with the way record state is updated, and
to reflect the way that \ac{vdmsl} handles atomic execution, we
believe a better approach is to use a flag that indicates if invariant
checks are enabled or not. Since this flag should not affect the
generated code, we make it a \kw{ghost} field such that it is only
visible at the specification level. Since this \kw{ghost} field must
be accessible everywhere in the translation, we make it a static field
of the class, as shown in \autoref{lst:invChecks}. The \kw{ghost}
field must be added to one of the generated Java classes since Java
does not really have global variables. Note that this flag does not
affect pre- and postconditions since these checks must always be
evaluated.

\begin{lstlisting}[style=customJml,caption={Ghost field used to
control invariant checking.},label={lst:invChecks}]
/*@ public ghost static boolean invChecksOn = true; @*/
\end{lstlisting}

The declaration of \texttt{invChecksOn} allows us to formulate
invariants such that violations are reported only if invariant
checking is enabled. An example of this is shown in
\autoref{lst:jmlRecInv} for the record state class of the \texttt{ATM}
module.

\begin{lstlisting}[style=customJml,caption={The invariant of the record state class.},label={lst:jmlRecInv}]
//@ public instance invariant atm.ATM.invChecksOn ==> inv_St(validCards,currentCard,pinOk,accounts);
\end{lstlisting}

The \texttt{invChecksOn} flag provides the means to emulate the
behaviour of atomic execution in a Java environment as shown in
\autoref{lst:jmlReturn}. Specifically, the \ac{jml} \kw{set} statement
is used to disable/enable invariant checking before/after executing
the body of the \texttt{ReturnCard} method.

\begin{lstlisting}[style=customJml,caption={Code-generated version
of the \texttt{ReturnCard} operation.},label={lst:jmlReturn}]
//@ requires pre_ReturnCard(St);
//@ ensures post_ReturnCard(\old(St.copy()),St);
public static void ReturnCard() {
 atm.ATMtypes.Card atomicTmp_1 = null;
 //@ assert ((atomicTmp_1 == null) || Utils.is_(atomicTmp_1,atm.ATMtypes.Card.class));
 Boolean atomicTmp_2 = false;
 //@ assert Utils.is_bool(atomicTmp_2);
 { /* Start of atomic statement */
   //@ set invChecksOn = false;
   //@ assert St != null;
   St.set_currentCard(Utils.copy(atomicTmp_1));
   //@ assert St != null;
   St.set_pinOk(atomicTmp_2);
   //@ set invChecksOn = true;
   //@ assert \invariant_for(St);
 } /* End of atomic statement */
}
\end{lstlisting}

\transRule{Enabling and disabling invariant checking}{Declare in a
  code-generated module \texttt{M} a globally accessible \ac{jml}
  \kw{ghost} field \texttt{invChecksOn} to control
  invariant checking:\\
  \texttt{/*@ \kw{public} \kw{ghost} \kw{static}\\
    \kw{boolean} invChecksOn = \kw{true}; @*/}\\
  Before executing a code-generated atomic statement (in any of the
  code-generated modules) invariant checking is disabled using the
  following \ac{jml} \kw{set}
  statement:\\
  \texttt{//@ \kw{set} M.invChecksOn = \kw{false};}\\
  After the code-generated atomic block has finished executing
  invariant checking is re-enabled using:\\
  \texttt{//@ \kw{set} M.invChecksOn = \kw{true};}}

When all the statements have been executed it must be ensured that no
invariants have been violated. For the example in
\autoref{lst:jmlReturn}, the only thing that needs to be checked is
that the state component of the \texttt{ATM} class, i.e.\ \texttt{St}
does not violate its invariant. This is checked by asserting that
\inv{St} holds.

\transRule{Resuming invariant checking}{Let
  \texttt{d\sub{1},...,d\sub{n}} be state designators of records that
  have been updated, or affected by an update, during execution of a
  code-generated atomic statement block. Further assume that
  \texttt{d\sub{1},...,d\sub{n}} have been updated in the given order,
  i.e.\ \texttt{d\sub{i}} was updated (for the first time) before
  \texttt{d\sub{i+1}} and that \texttt{d\sub{i}} may be of one of
  \texttt{m\sub{i}} record types\\
  \texttt{D\sub{i1},...,D\sub{im\sub{i}}}. Immediately after executing
  the code-generated atomic statement block, it is checked that the
  state designators \texttt{d\sub{1},...,d\sub{n}} do not violate any
  invariants using
  the following sequence of \kw{assert} statements:\\
  \texttt{//@ \kw{assert} \guardedinv{d\sub{1}}{D\sub{11}};}\\
  ...\\
  \texttt{//@ \kw{assert} \guardedinv{d\sub{1}}{D\sub{1m\sub{1}}};}\\
  ...\\
  \texttt{//@ \kw{assert} \guardedinv{d\sub{n}}{D\sub{n1}};}\\
  ...\\
  \texttt{//@ \kw{assert} \guardedinv{d\sub{n}}{D\sub{nm\sub{n}}};}}

%
\newcounter{resumeInv}
\setcounter{resumeInv}{\arabic{transRule}}
%

The \invariantfor construct is not currently implemented in
OpenJML. Instead the invariant check, for an object, can be inlined as
a method call (rather than explicitly using \invariantfor). However,
throughout this report we use \invariantfor to check record invariants
as we believe it makes the examples easier to understand.

The \ac{jml} translator keeps track of state designators of records
that potentially have been updated as part of executing the
code-generated atomic statement block. This is done by analysing the
left-hand sides of the assignment statements. Immediately after
invariant checking is re-enabled, i.e.\ the code-generated atomic
statement block has finished execution, it is checked that no record
violates its invariant.

There are a few things related to rule \arabic{resumeInv} that are
worth clarifying. First, for assignments to composite state
designators such as \texttt{a.b.c:=42}, the invariants of the
individual state designators \texttt{a}, \texttt{b} and \texttt{c},
have to be checked, if these are defined. For this particular example
we say that \texttt{c} was updated, and that \texttt{a} and \texttt{b}
were affected by the update. Second, the order in which the invariants
are checked follows that used by the Overture \ac{vdm} interpreter. Third,
regardless of how many times a state designator is updated, the
corresponding invariant is only checked once (for each state
designator) since this is how atomic execution works in \ac{vdm},
i.e.\ the update(s) are performed atomically, and afterwards the
constraints that the state designators are subjected to are
checked. Fourth, rule \arabic{resumeInv} includes all the state
designators that have been updated or affected by an update. No
particularly complex situations can arise that makes it difficult to
identify these state designators since all \ac{vdmsl}'s data types use
call-by-value semantics, and therefore no aliasing can
occur. Essentially this means that for the assignment statement
\texttt{a.b.c:=42}, the only invariants (if defined) that have to be checked are
those of the state designators \texttt{a}, \texttt{b} and \texttt{c}
 since aliases do not exist. Therefore, the \ac{jml}
translator can determine (using static analysis) that assertions only
have to be generated for these state designators. Naturally this
simplifies the translation process, since the \ac{jml} translator does
not have to identify additional state designators (other than those
that appear on the left-hand side of the assignment) that are affected
by the assignment.

A state designator can be ``masked'' as a union type and in such
situations it cannot always be statically determined what the runtime
type of a state designator will be. To demonstrate this, consider the
record types \texttt{R1} and \texttt{R2} and a state designator
declared as \texttt{\kw{dcl} r : R1 | R2 := ...}. Further assume that
\texttt{R1} and \texttt{R2} code-generate to classes
\texttt{R1\sub{c}} and \texttt{R2\sub{c}}. After updating \texttt{r}
atomically in the generated code, it is ensured that \inv{(R1\sub{c})
  r} holds if \texttt{r} is of type \texttt{R1\sub{c}}, and similarly
that the equivalent condition is true if \texttt{r} is of type
\texttt{R2\sub{c}}. Since rule \arabic{resumeInv} has to take all
possible types into account, the invariant checks are formulated as
implications.

Although the \ac{vdm} type system allows state designators to be
``masked'' as union types, most of the time it is possible to
statically determine the runtime type of a state de\-sig\-na\-tor. For
example, in \autoref{lst:jmlReturn} no \kw{instanceof} check is needed
since the static type of the state component is \texttt{St}. This is
an example where the \ac{jml} translator simplifies the checks proposed
by rule \arabic{resumeInv}.

There are more aspects to rule \arabic{resumeInv} worth discussing --
especially when state designators are based on arbitrarily complex
data structures such as nested records. These will be addressed in
\autoref{sub:complex-state}.

\subsection{Translating module state to \ac{jml}}

As described in \autoref{sub:vdmsl}, a module state invariant
constrains the record type used to represent the state component of
the enclosing module. Therefore, a module state invariant can
essentially be seen as a record invariant that can be translated into
\ac{jml}-annotated Java without introducing additional translation
rules. This subsection instead explains how a \ac{vdmsl} state
definition is translated into a form that allows the rules related to
record invariants to be applied (see \autoref{sub:rec-inv}).

In our example each account can be accessed from an \ac{atm} using one
of the debit cards associated with it. In addition to the bank
accounts, the state of the \ac{atm} also keeps track of the debit
cards that the system considers valid, the debit card that is
currently inserted into the \ac{atm}, and wheth\-er the PIN code entered
by the user is valid. The state (including the state invariant) as
specified in \ac{vdmsl} is shown in~\autoref{lst:vdmCase} and
described in \autoref{sec:case}. Based on the state definition, a
record class is generated that represents the state type as shown in
\autoref{lst:jmlStateRec}. Recall that the fields in this class are
nullable according to rule \arabic{nullRule}.

\begin{lstlisting}[style=customJml,caption={The record class used to represent the state type.},label={lst:jmlStateRec}]
final public class St implements Record {
 public VDMSet validCards;
 public atm.ATMtypes.Card currentCard;
 public Boolean pinOk;
 public VDMMap accounts;

 //@ public instance invariant atm.ATM.invChecksOn ==> inv_St(validCards,currentCard,pinOk,accounts);
 /* Record methods omitted */ 
}
\end{lstlisting}

In addition, an instance of the record class is created to represent
the state component as shown in \autoref{lst:jmlStateComp}. The state
component is annotated with the \kw{spec\_public} modifier so that it
can be referred to from the \kw{requires} and \kw{ensures} clauses of
\kw{public} methods. Also note that the module is not constrained by
an invariant. This is handled entirely by the record invariant shown
in \autoref{lst:jmlStateRec}.

\begin{lstlisting}[style=customJml,caption={The state component in the
\texttt{ATM} module.},label={lst:jmlStateComp}]
final public class ATM {
 /* Fields omitted */

 /*@ spec_public @*/
 private static atm.ATMtypes.St St = new atm.ATMtypes.St(SetUtil.~set~(),
            null, false, MapUtil.map());
 /* Module methods omitted */ 
}
\end{lstlisting}

\transRule{Translating the state component}{Annotate state components
  of module classes with the\\
  \kw{spec\_public} modifier to ensure that the state components can
  be referred to from the \kw{requires} and\\
  \kw{ensures} clauses of \kw{public} methods.}


%%% Local Variables:
%%% mode: latex
%%% TeX-master: "../../vdm2jml-tr"
%%% End:
