
\section{Translation assessment}
\label{sec:assess}

In this section we provide an assessment of the translation. We first
describe how the correctness of the translation was assessed, and
afterwards we discuss the scope and treated feature set in relation to
existing \ac{jml} tools.

\subsection{Translation correctness}

The translation rules have been validated by running examples through
the \ac{jml} translator and analysing the generated Java/\ac{jml}
using the OpenJML runtime assertion checker. Some of the examples used
to test the tool constitute \emph{integration tests} that have been
developed by the authors. In addition, we have used the tool to
analyse an \emph{external specification} (originally used as part of
an industrial case study) that the authors have not been involved in
the development of. A summary of the different examples used to test
the translation is given below. Additional details about the examples
can be found via the references provided.

The \emph{integration tests} currently consist of 85 examples that
cover testing of all the translation rules. Each test (typically)
forms a minimal example that exercises a small part of the entire
translation (such as a single rule). The workflow for running these
tests is as follows: First, the test model is translated to
\ac{jml}-annotated Java using the \ac{jml} translator. Next, the
generated Java/\ac{jml} is compiled and executed using the OpenJML
runtime assertion checker. Finally, the (actual) output reported by
the OpenJML runtime assertion checker is compared to the expected
output in order to confirm that the behaviour of the test model is
preserved across the translation. For example, if the execution of a
test model produces a precondition violation then the equivalent error
is expected to be produced when the generated Java/\ac{jml} is
executed using the OpenJML runtime assertion checker. All the examples
used to test the \ac{jml} translator are available via Overture's
GitHub page~\cite{OvertureGithub} or can be found in a technical
report that presents a more complete definition of the
translation~\cite{JmlTechReport16}.

Compared to the \emph{integration tests}, the \emph{external
  specification} is a large example that is rich in terms of \ac{dbc}
elements. The model was originally developed to study the properties
of an algorithm used to obfuscate \ac{fad} codes, which are six digit
numbers used to identify branches of a retailer. The customer required
that obfuscated \ac{fad} codes were still six digit numbers, remained
unique (per branch), and that the entire range of \ac{fad} codes
(0-999999) was still available. In addition, the obfuscation had to be
a light-weight calculation (rather than a look-up in a table). The
properties of the algorithm were described using \ac{vdm} contracts to
allow the algorithm to be validated using \ac{vdm}'s test automation
features~\cite{Larsen&10c}.

Investigating whether the algorithm met the requirements necessitated
the generation and execution of one million tests that initially could
not be handled by any of the \ac{vdm} tools (either due to intractable execution
times, or because the \ac{vdm} interpreter ran out of
memory). Motivated by this, the specification was translated into a
\ac{jml}-annotated Java program~\cite{Jorgensen&16}, and all one
million tests were executed using a code-generated version of the
\ac{vdm} specification. In that way, the properties of the obfuscation
algorithm could be validated by executing a code-generated version of
the \ac{vdm} specification using the OpenJML runtime assertion
checker.

\subsection{Translation scope and treated feature set}

As explained in \autoref{sec:recursive-types} it is possible to
formulate recursive types that currently are not supported by the
\ac{jml} translator. Aside from that, all \ac{vdmsl}'s types and
contract-based elements are supported. However, the \ac{jml}
translator does not currently support the object-oriented and
real-time dialects of \ac{vdm}, called \ac{vdm}++~\cite{Fitzgerald&05}
and \ac{vdm}-RT~\cite{Lausdahl&11}.

The Java code-generator that we extend currently only uses Java 7
features in the generated code. OpenJML is the only \ac{jml} tool that
we are aware of that supports this version of Java. Specifically, as
of December 2016, OpenJML version 0.8.5 was released with support for
Java 8, i.e.\ the latest official Java version (at the current time of
writing). Other \ac{jml} tools, on the other hand, lack support for
recent Java versions (in particular Java 7 and 8). Therefore, these
tools cannot currently be used to analyse the generated Java/JML.

The \ac{jml} translation is only valuable if the \ac{jml} features
that it relies on are supported by \ac{jml} tools. Specifically, we
have aimed to develop a translation that generates Java/\ac{jml} that
can be analysed using OpenJML. However, the translation would benefit
from the \invariantfor construct, which OpenJML does not currently
support. Instead we offer an alternative way to represent this
construct in order to achieve compatibility with OpenJML (see
\autoref{sec:atomic} for details).

%%% Local Variables: 
%%% mode: latex
%%% TeX-master: "../paper"
%%% End: 
