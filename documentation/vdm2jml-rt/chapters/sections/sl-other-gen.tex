\section{Other aspects of \ac{vdmsl} affecting the
\ac{jml}-generation}
\label{sec:sl-other-gen}

There are other aspects of \ac{vdmsl} that further complicate the
generation of \ac{vdmsl} models to \ac{jml}-annotated Java. In this
section we use examples to demonstrate these issues and explain how
they may be overcome.

\subsection{Complex state designators}
\label{sub:complex-state}

State designators may be composite data structures such as records
with fields that themselves are records. Such a data type forms
\emph{complex state designators} that when modified require careful
handling during the translation process. To demonstrate this, consider
the three \ac{vdmsl} record definitions \texttt{R1}, \texttt{R2} and
\texttt{R3} in \autoref{lst:vdmRecNesting}. Note in particular how the
invariants of \texttt{R1} and \texttt{R2} depend on the field of
\texttt{R3}. This transitive dependency complicates checking of
invariants in the generated code. To demonstrate this, the operation
in \autoref{lst:vdmRecNesting} instantiates \texttt{R1} as \texttt{r1}
and modifies it to violate the \texttt{R1} invariant, which causes a
runtime-error to be reported.

\begin{vdmsl}[style=customVdm,caption={Record nesting in
\ac{vdmsl}.},label={lst:vdmRecNesting}]
types
R1 :: r2 : R2
inv r1 == r1.r2.r3.x <> -1;
R2 :: r3 : R3
inv r2 == r2.r3.x <> -2;
R3 :: x : int
inv r3 == r3.x <> -3;

operations
op: () ==> nat
op () ==
(
 dcl r1 : R1 := @mk_@R1(@mk_@R2(@mk_@R3(5)));
 r1.r2.r3.x := -1;
 return 0;
)
\end{vdmsl}

The operation \texttt{op} in \autoref{lst:vdmRecNesting} produces the
method in \autoref{lst:jmlRecNesting}. For this example, \texttt{r1}
is the same in both listings, \texttt{r2} is the same as
\texttt{stateDes\_1} in \autoref{lst:jmlRecNesting}, and \texttt{r3}
is the same as \texttt{stateDes\_2}. Note that in
\autoref{lst:jmlRecNesting} we have removed fully qualified names of
record classes and other \ac{jml} checks that are not relevant.

\begin{lstlisting}[style=customJml,caption={Code-generated version of
the operation from
\autoref{lst:vdmRecNesting}.},label={lst:jmlRecNesting}]
public static Number op() {
 R1 r1 = new R1(new R2(new R3(5L)));
 R2 stateDes_1 = r1.get_r2();
 R3 stateDes_2 = stateDes_1.get_r3();
 stateDes_2.set_x(-1L);
 //@ assert \invariant_for(stateDes_1);
 //@ assert \invariant_for(r1);
 Number ret_1 = 0L;
 return ret_1;
}
\end{lstlisting}

Immediately after completing the state update, i.e.\ invoking
\texttt{stateDes\_2.set\_x(-1L)}, the following things happen:

\begin{enumerate}

\item The state of \texttt{stateDes\_2} becomes \emph{visible} thus
triggering the invariant check of \texttt{stateDes\_2}.

\item The invariant check of \texttt{stateDes\_1} is run by asserting
  \inv{stateDes\_1} and finally,

\item the invariant check of \texttt{r1} is run by asserting\\
  \inv{r1}, which causes a runtime-error to be reported.
\end{enumerate}

Strictly speaking the objects pointed to by \texttt{stateDes\_1} and
\texttt{r} are also in visible states after executing the update to
\texttt{stateDes\_2} and therefore the invariants of those objects
should also hold. In particular a state is \emph{visible} for an
object \texttt{o} \textit{``when no constructor, destructor,
  non-static method invocation with \texttt{o} as receiver, or static
  method invocation for a method in \texttt{o}\textquotesingle s class
  or some superclass of \texttt{o}\textquotesingle s class is in
  progress \cite{Leavens&13}''}. So in theory the invariant checks
should not have to be run explicitly (step 2 and step 3). The reason
that the \ac{jml} translator generates these checks anyway has to do
with the strategies \ac{jml} tools use to check invariants.

Tools such as \ac{jml} runtime checkers may assume no problems with
ownership aliasing to avoid having to keep track of what objects and
types are in visible states. Although this reduces the overhead of
checking invariants, it also means that some invariant violations might
go unnoticed. Alternatively, tools can check every applicable
invariant for classes and objects in visible states but this adds a
significant overhead to the program execution.

Since aliasing can never occur in \ac{vdmsl}, it becomes simpler to
keep track of what objects are in a visible state in the generated
code and thus generate \ac{jml} checks that explicitly trigger the
invariants checks. This has the advantage that invariant violations do
not go unnoticed even though a \ac{jml} tool adopts a more practical
approach to checking invariants.

For the example in \autoref{lst:jmlRecNesting}, the important thing is
to ensure that the violation of the invariant of \texttt{R1} is
reported after executing the state update. This is done by asserting
the entire chain of state designators. The \ac{jml} translator is able
to generate these checks since it keeps track of state designators of
records that may have been affected by updates to other state
designators.

\transRule{Checking transitive dependencies}{ Let \texttt{d\sub{n}} be
  a state designator of a record in the generated code that has been
  updated non-atomically, and let \texttt{d\sub{k},...,d\sub{1}}, for
  \texttt{k = n-1}, be state designators that were affected by the
  update to \texttt{d\sub{n}}. Further assume that \texttt{d\sub{i}}
  may be of one of \texttt{m\sub{i}} record types
  \texttt{D\sub{i1},...,D\sub{im\sub{i}}}. Immediately after executing
  the update to \texttt{d\sub{n}} the state of \texttt{d\sub{n}}
  becomes visible. To ensure that the invariant is evaluated
  for all affected
  state designators, execute the following sequence of assertions:\\
  \texttt{//@ \kw{assert} \guardedinv{d\sub{k}}{D\sub{k1}};}\\
  ...\\
  \texttt{//@ \kw{assert} \guardedinv{d\sub{k}}{D\sub{km\sub{k}}};}\\
  ....\\
  \texttt{//@ \kw{assert} \guardedinv{d\sub{1}}{D\sub{11}};}\\
  ...\\
  \texttt{//@ \kw{assert} \guardedinv{d\sub{1}}{D\sub{1m\sub{1}}};}}

%
\newcounter{dependency}
\setcounter{dependency}{\arabic{transRule}}
%

Note that the code in \autoref{lst:jmlRecNesting} omits the
\kw{instance} \kw{of} checks, proposed by rule \arabic{dependency},
since the types of the affected state designators can be determined
statically.

Regarding rule \arabic{resumeInv}, similar issues with transitive
dependencies may occur in the generated code when dealing with atomic
execution. Recall that invariant checking is disabled before a
code-generated atomic statement block is executed. Once the atomic
execution has completed, invariant checking is re-enabled, and
therefore rule \arabic{resumeInv} must also take into account all the
state designators that were affected by the atomic execution.

\subsection{Recursive types}
\label{sec:recursive-types}

It is possible to formulate recursive types for which the generated
\ac{jml} checks can only perform limited type checking. To demonstrate
this, consider the recursive \ac{vdm} type definition in
\autoref{lst:vdmRecType}. For this example, \texttt{S} represents an
infinite number of types including \kw{nat1} as well as all possible
dimensions of sequences that store elements of type \kw{nat1}, i.e.\
\kw{seq} \kw{of} \kw{nat1}, \kw{seq} \kw{of} \kw{seq} \kw{of}
\kw{nat1} and so on.

\begin{vdmsl}[style=customVdm,caption={Example of recursive type
definition in \ac{vdm}.},label={lst:vdmRecType}]
types
S = nat1 | seq of S;
\end{vdmsl}

\noindent The issue with this kind of type definition is that
\texttt{Is(v,S)} in theory becomes an expression of infinite
length. The \ac{jml} translator stops generating type checks whenever
it encounters type cycles. For the particular example in
\autoref{lst:vdmRecType} this means that a Java value or object
reference \texttt{v} is only considered to respect \texttt{S} if
\texttt{Utils.is\_nat1(v)} holds. For the rest of this section, we
discuss the current limitations of type checking recursive types, and
describe how these limitations may be addressed.

The approach used to check types could be changed to also
take the depth of the recursion \texttt{n} into account, i.e.\ use
\texttt{Is(v,T,n)} to generate the type checks. The current approach
used by the \ac{jml} translator thus corresponds to generating checks
using \texttt{Is(v,T,1)}. \texttt{Is(v,S,2)} then generates checks for
types \kw{nat1} and \kw{seq} \kw{of} \kw{nat}, whereas
\texttt{Is(v,S,3)} additionally generates a check for the type \kw{seq}
\kw{of} \kw{seq} \kw{of} \kw{nat1}.

Alternatively, checking a recursive type \texttt{T} (such as
\texttt{S} shown in \autoref{lst:vdmRecType}) can be done using a
code-generated recursive method that is constructed in a way that
allows a value \texttt{v} to be validated against \texttt{T}. Although
static provers may not be able to perform checking of such types it
should be possible using runtime assertion checking. However, in order
to enable this style of type checking, the \ac{jml} translator would
have to be extended with functionality that enables these methods to
be generated such that they can be invoked from the generated \ac{jml}
assertions.

The limitation of the \ac{jml} translator for the example shown in
\autoref{lst:vdmRecType} is a consequence of \texttt{S} being defined
using the union type constructor ``\texttt{|}''. However, it is
possible to check more practical examples of recursively defined types
such as the linked list \texttt{LL} shown in
\autoref{lst:vdmLinkedList}.

To demonstrate this, consider the construction of a linked list value
in \ac{vdm} that contains the numbers 1, 2 and 3 as shown in
\autoref{lst:vdmLinkedValue}. In the generated code this value is
represented using the code shown in \autoref{lst:jmlLinkedListValue}.

\begin{vdmsl}[style=customVdm,caption={Example of a linked list
defined using a record type.},label={lst:vdmLinkedList}]
types
LL ::
  element : nat
  tail : [LL]
\end{vdmsl}

\begin{vdmsl}[style=customVdm,caption={Example of a linked list value
in \ac{vdm}.},label={lst:vdmLinkedValue},numbers=none]
@mk_@LL(1, @mk_@LL(2, @mk_@LL(3, nil)))
\end{vdmsl}

\begin{lstlisting}[style=customJml,caption={Example of a linked list
value in Java.},label={lst:jmlLinkedListValue},numbers=none]
new LL(1L,new LL(2L,new LL(3L, null)));
\end{lstlisting}

Each time an object of type \texttt{LL} is instantiated in Java the
constructor checks the types of the current \texttt{element} and the
\texttt{tail} -- see \autoref{lst:jmlCheckLinkedList}. For this linked
list example, it is therefore possible to type check \texttt{LL} since
the \ac{vdm} type is represented using a recursively defined class in
the generated code.

\begin{lstlisting}[style=customJml,caption={Type checking a linked list using \ac{jml}.},label={lst:jmlCheckLinkedList}]
public LL(final Number _element, final LL _tail) {
 //@ assert Utils.is_nat(_element);
 //@ assert (_tail == null || Utils.is_(_tail,LL.class));
 ...
} 
\end{lstlisting}

\subsection{Detecting problems with the generated code}
\label{sec:detect-problem}

As explained in \autoref{sec:postcond} deep copying objects may
significantly affect the performance of the generated code. Therefore,
the user may not always want to have these copy calls
generated. However, from a general perspective this may result in code
that does not preserve the semantics across the translation. \ac{jml}
specifications can help detect such problems. To demonstrate this
consider the \ac{vdmsl} operation in
\autoref{lst:vdmVectorExample}. This operation assumes the existence
of a two-dimensional vector \texttt{Vector2D}, defined as a record (a
value type). In \autoref{lst:vdmVectorExample} \texttt{v2} is created
as a deep copy of \texttt{v1}, and therefore the assignment to
\texttt{v1} has no affect on \texttt{v2}, and \texttt{op} therefore
returns 1 (see the postcondition).

If this example is translated to Java with deep copying
\emph{disabled} the code shown in \autoref{lst:jmlVectorExample} is
produced. Note that this listing omits the generated \ac{jml}
assertions to focus on the postcondition.

\begin{vdmsl}[style=customVdm,caption={Use of value types in
\ac{vdm}.},label={lst:vdmVectorExample}]
op : () ==> nat
op () == (
dcl v1 : Vector2D := @mk_@Vector2D(1,2);
dcl v2 : Vector2D := v1; -- Copy value
v1.x := 2;
return v2.x;)
post RESULT = 1
\end{vdmsl}

\begin{lstlisting}[style=customJml,caption={Generated Java code
without copy calls.},label={lst:jmlVectorExample}]
//@ ensures post_op(\result);
public static Number op() {
  Vector2D v1 = new Vector2D(1L,2L);
  Vector2D v2 = v1;
  v1.set_x(2L);
  Number ret_1 = v2.get_x();
  return ret_1;
}
\end{lstlisting}

If this code is executed using the OpenJML runtime assertion checker
an error is reported because the method returns 2, which is different
from the result obtained by executing the corresponding \ac{vdmsl}
operation. Since deep copying is disabled only the \texttt{v1}
reference is copied, and therefore the update to \texttt{v1}, i.e.\
\texttt{v1.set\_x(2L)}, also affects \texttt{v2}.

The detection of the postcondition violation as reported by the
OpenJML runtime assertion checker is shown in
\autoref{lst:racOutputPost}. However, if the code is generated with
deep copying enabled (at the cost of performance) then \texttt{v2}
will be constructed as \texttt{Utils.copy(v1)} and the method will
change to return 1, as expected.

\begin{lstlisting}[style=racOutput,caption={Detection of a
postcondition violation.},label={lst:racOutputPost}]
Ex/DEFAULT.java:17: JML postcondition is false
    public static Number op() {
Ex/DEFAULT.java:16: Associated declaration: Ex/DEFAULT.java:17: 
    //@ ensures post_op(\result);
\end{lstlisting}



%%% Local Variables:
%%% mode: latex
%%% TeX-master: "../../vdm2jml-tr"
%%% End:
