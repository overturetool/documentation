\section{Checking \ac{vdm} types using \ac{jml}}
\label{sec:sl-types-gen}

In this section we describe how the translator uses \ac{jml} to check
the consistency of \ac{vdm} types when they are code-generated.

Throughout this section we construct a function called
\texttt{Is(v,T)} that takes as input a Java value \texttt{v} and a
\ac{vdm} type \texttt{T} and produces a \ac{jml} expression that can
be used to check whether \texttt{v} represents a value of type
\texttt{T}. We use \texttt{Is(v,T)} to check whether a Java value
remains consistent with the \ac{vdm} type that produces it. The
check produced by \texttt{Is(v,T)} can be added to the generated Java
code to ensure that no type violations occur.

This section covers some of the different classes of \ac{vdm} types
that the \ac{jml} translator supports, and explains using our case
study example, how \ac{jml} is used to check a Java value against the
\ac{vdm} type that produces it.  Finally, we summarise and provide the
complete definition of \texttt{Is(v,T)} in \autoref{fig:f-complete}.

%
% Constants used to control vertical space between cases
%
\newcommand\caseSpace{1pt}
\newcommand\caseSpaceSeqSet{14.0\caseSpace}
\newcommand\caseSpaceMap{20.0\caseSpace}

\begin{figure*}[ht]
  \tiny
\begin{mdframed}
\begin{equation*}
\texttt{Is(v,T)} =
\begin{cases}
  \texttt{Utils.is\_bool(v)} & \kw{if} \texttt{ T = \kw{bool}}\\[\caseSpace]

  \texttt{Utils.is\_nat(v)} & \kw{if} \texttt{ T = \kw{nat}}\\[\caseSpace]

  \texttt{Utils.is\_nat1(v)} & \kw{if} \texttt{ T = \kw{nat1}}\\[\caseSpace]

  \texttt{Utils.is\_int(v)} & \kw{if} \texttt{ T = \kw{int}}\\[\caseSpace]

  \texttt{Utils.is\_rat(v)} & \kw{if} \texttt{ T = \kw{rat}}\\[\caseSpace]

  \texttt{Utils.is\_real(v)} & \kw{if} \texttt{ T = \kw{real}}\\[\caseSpace]

  \texttt{Utils.is\_char(v)} & \kw{if} \texttt{ T = \kw{char}}\\[\caseSpace]

  \texttt{Utils.is\_token(v)} & \kw{if} \texttt{ T = \kw{token}}\\[\caseSpace]

  \texttt{Utils.is\_(v,String.\kw{class})} & \kw{if} \texttt{ T = \kw{seq of char}}\\[6.0\caseSpace]

  \texttt{Utils.is\_(v,S\sub{CG}.\kw{class})} &
  \!\begin{aligned}[b]
    & \kw{if} \texttt{ T} \text{ is a record or quote type } \texttt{S} \text{ that generates to}\\
    & \text{a Java class with the fully qualified name } \texttt{S\sub{CG}}
    \end{aligned}\\[10.0\caseSpace]

  \texttt{(v == \kw{null} || Is(v,S))} & \kw{if} \texttt{ T = [S]}\\[\caseSpace]

  \texttt{V2J.isTup(v,n) \&\& Is(v,T\sub{1}) \&\&...\&\& Is(v,T\sub{n})} & \kw{if} \texttt{ T = T\sub{1}*...*T\sub{n}}\\[\caseSpace]

  \texttt{Is(v,T\sub{1}) ||...|| Is(v,T\sub{n})} & \kw{if} \texttt{ T = T\sub{1}|...|T\sub{n}}\\[6.0\caseSpace]

  \!\begin{aligned}[b]
    & \texttt{V2J.isSet(v) \&\& (\kw{\textbackslash forall} \kw{int} i;}\\
    & \texttt{ 0 <= i \&\& i < V2J.size(v); Is(V2J.get(v,i),S))}
    \end{aligned}           & \kw{if} \texttt{ T = \kw{set} \kw{of} S}\\[\caseSpaceSeqSet]

  \!\begin{aligned}[b]
    & \texttt{V2J.isSeq(v) \&\& (\kw{\textbackslash forall} \kw{int} i;} \\
    & \texttt{ 0 <= i \&\& i < V2J.size(v); Is(V2J.get(v,i),S))}
    \end{aligned}           & \kw{if} \texttt{ T = \kw{seq} \kw{of} S}\\[\caseSpaceSeqSet]

  \!\begin{aligned}[b]
    & \texttt{V2J.isSeq1(v) \&\& (\kw{\textbackslash forall} \kw{int} i;} \\
    & \texttt{ 0 <= i \&\& i < V2J.size(v); Is(V2J.get(v,i),S))} \\
    \end{aligned}           & \kw{if} \texttt{ T = \kw{seq1} \kw{of} S}\\[\caseSpaceSeqSet]

  \!\begin{aligned}[b]
    & \texttt{V2J.isMap(v) \&\& (\kw{\textbackslash forall} \kw{int} i;} \\
    & \texttt{ 0 <= i \&\& i < V2J.size(v);} \\
    &\texttt{ Is(V2J.getDom(v,i),D) \&\& Is(V2J.getRng(v,i),R))}
    \end{aligned}           & \kw{if} \texttt{ T = \kw{map} D \kw{to} R}\\[\caseSpaceMap]

  \!\begin{aligned}[b]
    & \texttt{V2J.isInjMap(v) \&\& (\kw{\textbackslash forall} \kw{int} i;} \\
    & \texttt{ 0 <= i \&\& i < V2J.size(v);} \\
    &\texttt{ Is(V2J.getDom(v,i),D) \&\& Is(V2J.getRng(v,i),R))}
    \end{aligned}           & \kw{if} \texttt{ T = \kw{inmap} D \kw{to} R}\\[\caseSpaceMap]
    
  \texttt{Is(v,D) \&\& inv\_T(v)} &
  \!\begin{aligned}[b]
    & \kw{if} \texttt{ T} \text{ is a named invariant type with domain type }\\ 
    & \texttt{D} \text{ and invariant method} \texttt{ inv\_T}
    \end{aligned}\\[10.0\caseSpace]

\end{cases}
\end{equation*}
\end{mdframed}
\caption{Complete definition of \texttt{Is(v,T).}}
\label{fig:f-complete}
\end{figure*}

\subsection{Where to generate dynamic type checks}
\label{sec:where}

Most of the types available in \ac{vdm} are also present in Java in
some form or other. The \ac{vdm} and Java type systems do, however,
have some differences that require us to generate extra checks to
ensure that a Java value remains consistent with the \ac{vdm} type
that produces it.


In addition to producing the \ac{jml} expression needed to check the
consistency of a type, i.e.\ \texttt{Is(v,T)}, we also need to
consider where to add the check to the generated code. The description
below summarises the \ac{vdmsl} constructs that must be considered
when adding these checks to the generated Java code. We use the term
\emph{parameter} to refer to an identifier whose value does not
change. A parameter can be defined using a \kw{let} construct, which
is different from a state designator or variable that can be locally
defined using a \kw{dcl} statement or globally using a state
definition (see \autoref{sec:vdmjml}). The constructs to be considered
are:

\begin{itemize}

\item \kw{return} statement: If a functional description has a
  specified result type in its signature, then the returned value must
  be checked against the specified type.

\item Parameters of functions and operations: The arguments passed to
  a functional description must be checked against the specified types
  of the corresponding formal parameters upon entry to the functional
  description.

\item State designators: After updating a local or global state
  designator, the new value assigned must respect the type of the
  state designator.

\item Variable or parameter declaration: After initialising a variable
  or parameter it must be checked against its declared type.

\item Value definition: An explicitly typed value definition must
  specify a value consistent with its type.

\end{itemize}

All of the constructs in the list above -- with the exception of the
value definition -- can be checked using a \ac{jml} \kw{assert}
statement. The reason for this is that the code-generated versions of
these constructs appear inside methods in the generated code. Since a
\ac{vdm} value definition code-generates to a \kw{public} \kw{static}
\kw{final} field (a constant) it is checked using a \kw{static}
invariant.

\subsection{Translating basic types}
\label{sec:basic}

In our example we may wish to check that the amount being withdrawn
from an account is valid -- for example by requiring that it is a
natural number larger than zero, as shown in \autoref{lst:vdmBasic}.

\begin{vdmsl}[style=customVdm,caption={Use of explicit type annotation
to ensure that a valid amount is being
withdrawn.},label={lst:vdmBasic}]
let amount : nat1 = expense - profit
in
  Withdraw(accId, amount);
\end{vdmsl}

In the generated Java code, shown in \autoref{lst:jmlBasic}, this is
checked by analysing the value of the \texttt{amount} variable using
the \texttt{Utils.is\_nat1} method available from the Java
code-generator's runtime library. This method is invoked from a
\ac{jml} annotation in order to check that \texttt{amount} is
different from \kw{null} and that it represents an integer larger than
zero.

\begin{lstlisting}[style=customJml,caption={Use of \ac{jml} to check
that a valid amount is being withdrawn.},label={lst:jmlBasic}]
Number amount =
  expense.longValue() - profit.longValue();
//@ assert Utils.is_nat1(amount);
return Withdraw(accId, amount);
\end{lstlisting}

\transRule{Checking of the \kw{nat1} type}{Let \texttt{v} be a value
  or object reference in the generated code that originates from a
  variable or pattern of type \kw{nat1} and further define
  \texttt{Is(v,\kw{nat1}) = Utils.is\_nat1(v)}.\\
  To ensure that \texttt{v} represents a value of type \kw{nat1},
  generate a \ac{jml} check to ensure that \texttt{Is(v, \kw{nat1})}
  holds.}

The approach used to check other basic types follows the principles
demonstrated using \autoref{lst:vdmBasic} and \autoref{lst:jmlBasic}
-- the main difference being that each basic type uses a dedicated
method from the Java code-generator's runtime library. Therefore, we
omit the details of how other basic types of \ac{vdm} are checked
using \ac{jml}, and instead provide the complete set of rules in
\autoref{fig:f-complete}.

We note that a record type or a quote type can be checked in a way
similar to that of a basic type. The reason for this is that the Java
code-generator produces a Java class for each of the record definitions and
quote types in the \ac{vdm} model. Therefore, all there is to checking
whether an object reference represents a given record or quote class
is to check whether the object reference is an instance of said
class. The rules for checking record and quote types are included in
\autoref{fig:f-complete}.

\subsection{Translating optional types}
\label{sec:optional}

To demonstrate how the \ac{jml} translator handles optional types
consider the \texttt{GetCurrentCardId} operation in
\autoref{lst:vdmGetCurrentCardId}. This operation returns the
identification of the debit card currently inserted into the machine,
if any. Otherwise the operation returns \kw{nil} to indicate the
absence of a debit card. To allow \kw{null} as a return value, the
optional type operator is used to specify the return type of the
operation as \texttt{[\textbf{nat}]}.

\begin{vdmsl}[style=customVdm,caption={Operation for getting the id of
the debit card currently inserted into the
\ac{atm}.},label={lst:vdmGetCurrentCardId}]
GetCurrentCardId : () ==> [nat]
GetCurrentCardId () ==
 if currentCard <> nil then
   return currentCard.id
 else
   return nil;
\end{vdmsl}

Considering solely the signature of the code-generated version of this
operation, shown in \autoref{lst:jmlGetCurrentCardId}, there is no way
to tell that the return type represents a \texttt{[\textbf{nat}]}.

\begin{lstlisting}[style=customJml,caption={Signature of the
code-generated version of the \texttt{GetCurrentCardId}
operation.},label={lst:jmlGetCurrentCardId}]
public static Number GetCurrentCardId(){...}
\end{lstlisting}

\noindent The reason for this is that the Java code-generator uses the
\texttt{Number} class (which is part of the Java standard library) to
represent all numeric \ac{vdm} types. That the return type of the
operation is \texttt{[\textbf{nat}]} only becomes apparent when we
start using the corresponding method.

To demonstrate this, the Java fragment in
\autoref{lst:jmlOptionalCheck} uses the result of invoking the
\texttt{GetCurrentCardId} method to initialise a variable named
\texttt{id}. The initialisation of \texttt{id} is immediately followed
by a check that ensures that it represents either \kw{null} or a
natural number. The approach of allowing \kw{null} values like this is
the same for all optional types.

% The methods in the Java code-generator runtime library used for
% performing \ac{jml} checks (including those of \texttt{V2J}) all
% reject \kw{null} values, e.g.\ \texttt{Utils.is\_nat(\textbf{null})}
% yields \kw{false}.

\begin{lstlisting}[style=customJml,caption={Use of the
\texttt{GetCurrentCardId} method in the generated
code.},label={lst:jmlOptionalCheck}]
Number id = GetCurrentCardId();
//@ assert id == null || Utils.is_nat(id);
\end{lstlisting}

\transRule{Checking of optional types} {Let \texttt{v} be a value or
  object reference in the generated code that originates from a
  variable or pattern of the \ac{vdm} type \texttt{[T]} and further define\\
  \texttt{Is(v,[T]) = (v == \kw{null} || Is(v,T))}\\
  To ensure that \texttt{v} represents a value of type \texttt{[T]},
  generate a \ac{jml} check to ensure that \texttt{Is(v,[T])} holds.}

\subsection{Translating tuple types}
\label{sec:tuples}

In our case study example we use a tuple type to represent the status
of the \ac{atm}: the first field is a \kw{boolean} flag that indicates
if the \ac{atm} is currently awaiting a debit card to be inserted, and
the second field is a human-readable description of the current state
of the \ac{atm}, e.g. ``transaction in progress''. The signature of
the operation that retrieves the status of the \ac{atm} is shown in
\autoref{lst:vdmTuple}. Note in particular that the status returned is
represented using the tuple type \texttt{\kw{bool} * \kw{seq of
    char}}.

\begin{vdmsl}[style=customVdm,caption={The signature of the
\texttt{GetStatus} operation.},label={lst:vdmTuple}]
GetStatus : () ==> bool * seq of char
GetStatus () == ...
\end{vdmsl}

% In \ac{vdm} the status of the \ac{atm} can be retrieved as shown in
% \autoref{lst:vdmAtmStatus}. Note that the status flag and message are
% returned as fields of the same tuple value.

% \begin{vdmsl}[style=customVdm,caption={Retrieving the status of the
% \ac{atm}.},label={lst:vdmAtmStatus}]
% let status = GetStatus()
% in
%  ...
% \end{vdmsl}

In the generated Java code, every tuple value is represented as an
instance of the \texttt{Tuple} class available from the Java code-generator
runtime library. Since the \texttt{Tuple} class represents
tuple values in general, each instance of this class must be checked
against the specific tuple type that it originates from.

After the status of the \ac{atm} has been retrieved using the
\texttt{GetStatus} method in the generated code, the status is checked
as shown in \autoref{lst:jmlAtmStatus}. First it is checked that
\texttt{status} is a tuple of size two. Afterwards it is checked that
the first field is a \kw{boolean} and that the second field is a Java
\texttt{String} (which represents the \texttt{\kw{seq of char}} type).

\begin{lstlisting}[style=customJml,caption={Checking the \ac{atm}
status in the generated code.},label={lst:jmlAtmStatus}]
Tuple status = GetStatus();
//@ assert (V2J.isTup(status,2) && Utils.is_bool(V2J.~field~(status,0)) && Utils.is_(V2J.~field~(status,1),String.class));
\end{lstlisting}

\transRule{Checking of tuple types} {Let \texttt{v} be a value or
  object reference in the generated code that originates from a
  variable or pattern of the \ac{vdm} tuple type
  \texttt{T\sub{1}*...*T\sub{n}} and further define\\
  \texttt{Is(v,T\sub{1}*...*T\sub{n}) = V2J.isTup(v,n) \&\&\\
    Is(v,T\sub{1}) \&\&...\&\& Is(v,T\sub{n})}\\ To ensure that
  \texttt{v} represents a value of type \texttt{T\sub{1}*...*T\sub{n}},
  generate a \ac{jml} check to ensure that
  \texttt{Is(v,T\sub{1}*...*T\sub{n})} holds.}

\subsection{Translating union types}
\label{sec:unions}

Attempting to insert a debit card into the \ac{atm} results in the
debit card being accepted, if no card is currently inserted and it is
considered a valid card by the system. Otherwise the card is
rejected. Based on the outcome of this the \texttt{NotifyUser}
operation, shown in \autoref{lst:vdmNotifyUser}, displays a message to
inform the card holder about the current status of the session. This
operation uses a union type, formed by the three quote types
\texttt{<Accept>}, \texttt{<Busy>} and \texttt{<Reject>}, to represent
one of three outcomes of the card holder attempting to insert a debit
card into the \ac{atm}.

\begin{vdmsl}[style=customVdm,caption={Operation used to notify a
\ac{atm} user.},label={lst:vdmNotifyUser}]
NotifyUser : <Accept>|<Busy>|<Reject> ==> ()
NotifyUser (outcome) ==
if outcome = <Accept> then
  Display("Card accepted")
elseif outcome = <Busy> then
  ...
\end{vdmsl}

\noindent The code-generated version of the \texttt{NotifyUser} operation is
shown in \autoref{lst:jmlNotifyUser}. Since the \texttt{outcome}
parameter originates from the union type formed by the three quote
types, it must be checked that \texttt{outcome} equals one of the three
possible values. This check is performed immediately after entering
the \texttt{NotifyUser} method, as shown in
\autoref{lst:jmlNotifyUser}.

\begin{lstlisting}[style=customJml,caption={Code-generated version of
the \texttt{NotifyUser} operation.},label={lst:jmlNotifyUser}]
public static void NotifyUser(final Object outcome) {
 //@ assert (Utils.is_(outcome,atm.quotes.AcceptQuote.class) || Utils.is_(outcome,atm.quotes.BusyQuote.class) || Utils.is_(outcome,atm.quotes.RejectQuote.class));
 if (Utils.equals(outcome, atm.quotes.AcceptQuote.getInstance())) {
   Display("Card accepted");
 } else if (Utils.equals(outcome, atm.quotes.BusyQuote.getInstance())){
   ...
}
\end{lstlisting}

\transRule{Checking of union types} {Let \texttt{v} be a value or
  object reference in the generated code that originates from a
  variable or pattern of the \ac{vdm} union type
  \texttt{T\sub{1}|...|T\sub{n}} and further define\\
  \texttt{Is(v,T\sub{1}|...|T\sub{n}) = Is(v,T\sub{1}) ||...||\\Is(v,T\sub{n})}\\
  To ensure that \texttt{v} represents a value of type
  \texttt{T\sub{1}|...|T\sub{n}}, generate a \ac{jml}
  check to ensure that\\
  \texttt{Is(v,T\sub{1}|...|T\sub{n})} holds.}

\subsection{Translating collections}
\label{sec:col}

In the generated code the \texttt{VDMSet}, \texttt{VDMSeq} and
\texttt{VDMMap} collection classes are used as raw types. Therefore
the code-generator does not take advantage of Java generics to make
compile-time guarantees about the types of the objects a collection
stores. This approach has the advantage of making it easier to store
Java objects and values of different types in the same collection
without having to introduce additional types. Although this allows the
type system of \ac{vdm} to be represented in Java it has the
disadvantage that no compile-time guarantees can be made about the
types of the objects that a collection stores.

In the \ac{atm} example we use the \texttt{TotalBalance} function,
shown \autoref{lst:vdmTotalBalance}, to calculate the total balance
available from a set of accounts.

\begin{vdmsl}[style=customVdm,caption={Function that calculates the
total balance available from a set of accounts.},label={lst:vdmTotalBalance}]
TotalBalance : set of Account -> real
TotalBalance (acs) ==
 if acs = {} then
   0
 else
  let a in set acs
  in
    a.balance + TotalBalance(acs \ {a});
\end{vdmsl}

When the \texttt{TotalBalance} function is code-generated to
\ac{jml}-annotated Java, the code-generator adds \ac{jml} assertions
to ensure that the set of accounts is consistent with the collection
type used in \ac{vdm}. Since an \texttt{Account} record is represented
using a Java class with the same name, we have to check that every
element in the set is an instance of said Java class. As shown in
\autoref{lst:jmlTotalBalance}, this is checked using a quantified
expression. This expression uses a bound variable \texttt{i} to
iterate over all the accounts and check that each element is an
instance of the \texttt{Account} record class. Although sets are
unordered collections, the quantified expression takes advantage of
\texttt{VDMset} being implemented as an ordered collection. The
formulation of the range expression in the quantified expression
further ensures that the assertion can be checked using a tool such as
the OpenJML runtime assertion checker, i.e. the assertion is
executable.

\begin{lstlisting}[style=customJml,caption={Code-generated version of
the \texttt{TotalBalance} operation.},label={lst:jmlTotalBalance}]
/*@ pure @*/
public static Number TotalBalance(final VDMSet acs) {
 //@ assert (V2J.isSet(acs) && (\forall int i; 0 <= i && i < V2J.size(acs); Utils.is_(V2J.get(acs,i),atm.ATMtypes.Account.class)));
 if (Utils.empty(acs)) {
   Number ret_1 = 0L;
   //@ assert Utils.is_real(ret_1);
   return ret_1;
 } else { ... /*Compute sum recursively */}
}
\end{lstlisting}

The \ac{jml} translator only uses Java 7 features since OpenJML did
not support Java 8 at the time the \ac{jml} translator was
developed. Iterating over collections (as shown in
\autoref{lst:jmlTotalBalance}) may also be achieved using Java 8
features such as lambda expressions. For example, one could imagine a
method used to check collection types that would take as input two
arguments (1) the collection itself and (2) a predicate method (e.g.\
lambda expression) that would be evaluated for each of the elements in
the collection. In that way the generated \ac{jml} annotations would
not have to rely on sets implemented as ordered collections. Since
lambda expressions in Java are mostly syntactic sugar for anonymous
inner classes, lambda expressions could in principle be represented
solely using Java 7 features. However, using this approach, the
generated \ac{jml} annotations would not be concise, although this is
only a concern if a human will read them.

\transRule{Checking of sets} {Let \texttt{v} be a value or object
  reference in the generated code that originates from a variable or
  pattern of the \ac{vdm} set type \texttt{\kw{set} \kw{of} T} and
  further define\\
  \texttt{Is(v,\kw{set} \kw{of} T) = V2J.isSet(v) \&\&\\
    (\kw{\textbackslash forall} \kw{int} i; 0 <= i \&\&\\ i <
    V2J.size(v); Is(V2J.get(v,i),T))}\\
  To ensure that \texttt{v}
  represents a value of type \texttt{\kw{set} \kw{of} T}, generate a
  \ac{jml} check to ensure that \texttt{Is(v,\kw{set} \kw{of} T)}
  holds.}

The \ac{vdm} sequence types \kw{seq} and \kw{seq1} are checked in a
way similar to sets. The difference between checking the \kw{seq} and
\kw{seq1} collection types is that the \kw{seq1} type requires at
least one element to be present in the sequence. Checking a map, which
like a set is an unordered collection, takes advantage of
\texttt{VDMMap} imposing an order on the domain and range values. The
main difference between checking a map and a set is that both the
domain and range values of a map have to be checked. Checking the
injective map type \kw{inmap} is similar to checking a standard map,
except that the injectivity property must hold. We refrain from
providing examples of how to check each of the collection types in
\ac{vdm} since they are similar to what has already been
shown. Instead we summarise the rules for checking all of the
collection types in \autoref{fig:f-complete}.

\subsection{Translating named invariant types to \ac{jml}}
\label{sec:named-type-invariant}

Since the Java code-generator does not generate additional class
definitions for named invariant types, the invariant imposed on such a
type cannot be expressed as a \ac{jml} invariant. This is only
possible for a record since it translates to a class definition.

Instead, we identify places in the generated code where a named
invariant type may be violated, as described in \autoref{sec:where},
and check that the invariant holds. Also, it is worth noting that a
named invariant type, unlike a record type, does not have an explicit
type constructor. Therefore, an expression can only violate a named
invariant type if the expression is explicitly declared to be of that
type.

The \ac{atm} in our example is not capable of dispensing cents and
also imposes a limit on the amount of money that can be
withdrawn. Therefore, the amount of money can be represented as a
named invariant type. An attempt to withdraw an amount of money that
exceeds 2000 will yield a runtime error. The named invariant type used
to represent the amount withdrawn from an account is shown together
with the \texttt{Withdraw} operation in
\autoref{lst:vdmWithdrawAmount}.

\begin{vdmsl}[style=customVdm,caption={The amount to withdraw modelled
using a named invariant type.},label={lst:vdmWithdrawAmount}]
types
Amount = nat1
inv a == a < 2000;

operations
Withdraw : AccountId * Amount ==> real
Withdraw (id, amount) == ...
\end{vdmsl}

On entering the code-generated version of \texttt{Withdraw}, shown
in~\autoref{lst:jmlWithdrawdNamedTypeInv}, we assert that
\texttt{amount} meets the named invariant type \texttt{Amount}. The
assertion does two things: First it performs a dynamic type check to
ensure that \texttt{amount} is a valid domain type of \texttt{Amount}
and secondly, it checks that the invariant predicate holds. For the
example in~\autoref{lst:jmlWithdrawdNamedTypeInv} this means checking
that \texttt{amount} is of type \kw{nat1} and smaller than 2000. Note
that meeting the invariant condition does not imply compatibility with
the domain type of the named invariant type and vice versa. For
example, -1 is smaller than 2000 but it is not of type
\kw{nat1}. Likewise, 2001 is of type \kw{nat1} but it exceeds 2000 so
neither -1 nor 2001 are of type \texttt{Amount}.

\begin{lstlisting}[style=customJml,caption={Checking a named invariant
type of an operation parameter in
\ac{jml}.},label={lst:jmlWithdrawdNamedTypeInv}]
public static Number Withdraw(final Number id, final Number amount){
...
//@ assert (Utils.is_nat1(amount) && inv_ATM_Amount(amount));
 ...
}
\end{lstlisting}

The code-generated invariant method for type \texttt{Amount} is shown
in~\autoref{lst:jmlAmount}. Since the named invariant type check,
shown in \autoref{lst:jmlWithdrawdNamedTypeInv}, is evaluated from
left to right using short-circuit evaluation
semantics~\cite{McCarthy61}, the invariant method is only invoked if
the value subject to checking is compatible with the domain type of
the named invariant type. Therefore, it is safe to narrow (or cast)
the type of the argument passed to the invariant method before
performing the invariant check.

\begin{lstlisting}[style=customJml,caption={The named invariant type
method for \texttt{Amount}.},label={lst:jmlAmount}]
/*@ pure @*/
/*@ helper @*/
public static Boolean inv_ATM_Amount(final Object check_a) {
 Number a = ((Number) check_a);
 return a.longValue() < 2000L;
}
\end{lstlisting}

\transRule{Checking of named invariant types} { Let \texttt{v} be a
  value or object reference in the generated code that originates from
  a variable or pattern of the \ac{vdm}\\
  named invariant type \texttt{T} based on the domain type \texttt{D}
  and constrained
  by invariant predicate \texttt{e(p)}, i.e.\ \texttt{T} is defined as\\
  \kw{types}\\
  \texttt{T = D}\\
  \texttt{\kw{inv} p == e(p)}\\
  Then \texttt{T} has an invariant method, responsible for running the
  code-generated version of the \texttt{e(p)} check, with a signature
  defined as:\\
  \texttt{\kw{public} \kw{static} \kw{boolean} inv\_T(Object o)}\\
  Further define
  \texttt{Is(v,T) = Is(v,D) \&\& inv\_T(v)}\\
  To ensure that \texttt{v} represents a value of type \texttt{T},
  generate a \ac{jml} check to ensure that \texttt{Is(v,T)} holds.}
%
\newcounter{namedTypeInv}
\setcounter{namedTypeInv}{\arabic{transRule}}
%
Note that the invariant method \texttt{inv\_T} in rule
\arabic{namedTypeInv} defines the input parameter \texttt{o} to be of
type \texttt{Object}, thus allowing \texttt{inv\_T} to accept inputs
of any type. Therefore, \texttt{inv\_T} must narrow the type of the
input parameter \texttt{o} before performing the invariant check (see
the example in \autoref{lst:jmlAmount}). This approach has the
advantage that it allows simpler \ac{jml} checks since the argument
type does not need to be narrowed before the invariant method is
invoked. Had the input parameter of the invariant method been defined
using the smallest possible type, then the argument type would need to
be narrowed for situations where the argument is masked as a union
type. Although this would complicate the \ac{jml} checks, it would
have the advantage of allowing type narrowing to be removed from the
invariant methods.


%%% Local Variables: 
%%% mode: latex
%%% TeX-master: "../paper"
%%% End: 
